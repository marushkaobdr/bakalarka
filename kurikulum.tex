\chapter{Kurikulum}
\label{}

%\lettrine[lines=3,slope=0pt,nindent=-2pt]{P}{ojem} kurikulum 
% Tady si muzes psa poznamky
Pojem kurikulum je odvozeno z latinského slova currere tedy běžet. Slovo curriculum má významy jako běh, závodní dráha či závodní vozík. Může to tedy znamenat pohyb po určité trase, směrem k určitému cíli. Nejčastěji je dnes toto slovo používané jako curriculum vitae neboli životopis, běh života. V češtině se používá jeho přepis kurikulum. 

\vspace{5mm} % tot udela vertikalni mezeru 5mm
Ve spojení s pedagogikou se tento pojem začíná užívat v zahraničí v 60. letech 20. století. Dá se chápat jako plánovaná a nasměrovaná trasu, na níž dítě získává zkušenosti dle svých schopností a zájmů. V České republice se tento pojem užívá až koncem 80. let. Existuje mnoho jeho definic a významů podle různých pedagogických koncepcí a názorů samostatných autorů.

K rozšíření pojmu kurikulum u nás významně přispěla Eliška Walterová. Díky ní se dostal do české odborné pedagogické terminologie. Pro tuto práci se hodí dva významy podle Walterové (1994, str. 15): (1) Vzdělávací program, projekt, plán: zahrnuje škálu od programu jednotlivého kurzu nebo vyučovacího předmětu až po komplexní program vzdělávací instituce, tj. plán všech aktivit ve škole;
(2) Průběh studia a jeho obsah: charakteristika vzdělávací dráhy a obsah zkušeností, kterou žák získává v době studia.

Průcha definuje v Moderní pedagogice (2005) kurikulum jako „obsah vzdělávání, který zahrnuje veškeré zkušenosti, které žáci získávají ve škole a v činnostech ke škole se vztahujících, zejména jejich plánování, zprostředkovávání a hodnocení.

Dle terminologie se dělí kurikulum na formální, neformální a skryté. V této práci se zabývám kurikulem formálním. Formální kurikulum je komplexní projekt cílů, obsahu, prostředků a organizace vzdělávání, realizace projektovaného kurikula, způsoby kontroly a hodnocení výsledků.

Kurikulární dokumenty jsou v obou srovnávaných zemích, tedy ve Francii a České republice, pojímány odlišně. Vybrala jsem srovnatelné parametry, které se nacházejí v obou dokumentech, a těm se dále blíže věnuji. Za účelem jejich porovnávání využívám v této práci pojem kurikulum jakožto vzdělávací program či obsah vzdělávání a detailněji se věnuji vzdělávacím oblastem předškolního vzdělávání a kompetencím, které by děti měly zvládat na konci mateřské školy. Pro jasnější představu o dokumentech obou zemí zmiňuji na začátku každé kapitoly i legislativní pojetí kurikula, jeho dostupnost a stručný přehled, jak celý dokument vypadá. 

\section{FRANCOUZSKÉ KURIKULUM}
	Mateřská škola ve Francii je součástí primárního vzdělávání. Přechod z mateřské školy na školu základní je plynulý a navazující. Francouzské kurikulum předškolního vzdělávání je tedy součástí stejného dokumentu jako kurikulum školy základní. Legislativně je zakotveno ve školském zákoně (Code de l'éducation) č. 2003-339, který vešel v platnost 14. června 2003.
		
	Francouzské kurikulum je veřejně dostupný dokument, který je vydáván v podobě bulletinu Ministerstvem školství a výzkumu. Nejnovější verze je z roku 2008 Bulletin officiel (B.O.) hors série n°3 de du 19 juin 2008 – Horaires et programmes d´enseignement de l´école primaire (maternelle et élémentaire), který je ke stažení na stránkách Ministerstva školství  www.education.gouv.fr/cid33/la-presentation-des-programmes-a-l-ecole-maternelle.html nebo www.education.gouv.fr/bo/2008/hs3/default.html.

	Vychází též v knižním provedení Qu´apprend-on à l´école maternelle? – 2011-2012 Les programmes officiels od CNDP (Centre National de Documentation Pédagogique) Národní centrum pro pedagogickou dokumentaci.

	Bulletin officiel obsahuje 11 kapitol. Názvy kapitol uvádím dvojjazyčně, volně jsem je přeložila z originálu do českého jazyka.

	\begin{enumerate}[I]	% takto se dela cislovani
		\item 
		Dopis od bývalého ministra školství Xaviera Darcose k novým programům primárního vzdělávání (
		\textit{Lettre de Xavier Darcos sur le noueaux programmes pour l´école primaire.}
		)
		\item 
		Hodinová dotace v mateřské a základní škole (
		\textit{Horaires des écoles maternelles et élémentaires}
		)
		\item
		Program primárního vzdělávání (
		\textit{Programmes d´enseignement  de l´école primaire}
		)
		\item 
		Preambule (\textit{Préambule})
		\item 
		Prezentace (\textit{Présentation})
		\item 
		Program mateřské školy: malá sekce, střední sekce, velká sekce 
		(\textit{Programme de l´école maternelle:petite section, moyenne section, grande section})
		\item 
		Cyklus základního vzdělávání - Program CP a CE1 (\textit{Cycle des apprentissages fondamentaux – Pogramme du CP et du CE1})		)
		\item 
		Cyklus prohlubování – Program CE2, CM1 a CM2 (\textit{Cycle des approfondissements – Programme du CE2, du CM1 et du CM2})
		\item 
		Kritéria organizace progresivity vzdělávání v mateřské škole (\textit{Repères pour organiser la progressivité des apprentissages à l´école maternelle})
		\item 
		Cyklus základního vzdělávání – Postup pro přípravné a základní kurzy (\textit{Cycle des apprentissage fondamentaux – Progression pour le cours préparatoire et le cours élémentaire})
		\item 
		Cyklus prohlubování – Postup pro CE2, CM1 a CM2 (\textit{Cycle des approfondissements pour le cours élémentaire dèuxieme année et le cours moyen})
		\item
	\end{enumerate}


	Mateřskou školou a programem vzdělávání se zabývá šestá kapitola, které se budu dále detailněji věnovat. Níže prezentuji vlastní překlad.
	Program pro mateřské školy je rozdělen do 6 vzdělávacích oblastí:

	\begin{enumerate}[1]
		\item Osvojit si řeč 
		\item Objevovat písmo 
		\item Stát se žákem 
		\item Jednat a vyjadřovat se vlastním tělem 
		\item Objevovat svět
		\item Vnímat, cítit, představovat si, tvořit
	\end{enumerate}

	Všechny vzdělávací obsahují ještě několik dílčích oblastí.

	\textcolor{red}{Tady by mel byt ještě uvod z fr bulletinu}

	\subsection{Osvojit si řeč (S´Approprier le Langage)}
		Mluvený jazyk je v mateřské škole základním pilířem učení. Vyjadřování a pochopení se dítěte učí skrze jazyk. Učí se být pozorný ke zprávě, která je mu sdělena, pochopit ji a odpovědět na ni. V rámci komunikace s učitelem, kamarády, při společných i specificky zaměřených aktivitách se každodenně učí novým slovům. Postupně si osvojuje syntaxi francouzského jazyka.  Společné aktivity obohacují slovní zásobu a způsoby využití jazyka dítěte (dotazování se, vyprávění, vysvětlování, přemyšlení).

		\paragraph{Komunikovat, vyjadřovat se (Échanger, s´exprimer)}
		Nejdříve se děti učí komunikovat prostřednictvím dospělého v situacích, které se ho přímo týkají: jejich vlastní potřeby, objevy, otázky; naslouchá a odpovídá na žádosti. S jistotou pojmenovává objekty okolo sebe. Účastní se komunikace ve skupině, čeká, až na něj přijde řada a možnost vyjádřit se a respektuje dané téma. Dokáže nazpaměť převyprávět naučené říkanky a písně. Pozvolna rozšiřuje časovou osu, mluví o tom, co se budě dít, co zažilo, dokáže vymýšlet příběhy a převyprávět základní fakta problému. Postupně získává potřebné základy jazyka k vyjádření se: popis osob a vztahů mezi nimi, užití správného časování sloves a jak vhodným způsobem popsat dění v příběhu.

		\paragraph{Porozumět (Comprendre)}
		Více než vyjadřování je základní kapacitou dítěte porozumění, na které se dává v tomto věku velký důraz.
		Děti se učí rozlišovat otázku, slib, příkaz, odmítnutí, vysvětlení, vyprávění. Rozumí příkazům vyučujícího a všem termínům, které k tomu patří. Jsou vedené k pochopení kamaráda či dospělého, který hovoří o věcech pro dítě neznámých. Opakováním příběhů a pohádek, jak klasické tak moderní literatury, které jsou přiměřené věku dětí, je dětem umožněno porozumět složitějším a delším vyprávěním, které dokážou též převyprávět. 

		\paragraph{Zdokonalovat se ve francouzském jazyce (Progresser vers la maîtrise de la langue française)}
		Užíváním jazyka a posloucháním čtených textů se dítě učí pravidlům větné skladby a pořadí slov ve francouzské větě. Na konci mateřské školy dítě ovládá všechny slovní druhy, tvoří celé věty i krátká vyprávění a vysvětlení. Každodenní čtení a vyprávění příběhů učitelem, která obsahují nová slova, nestačí k jejich zapamatování. Nabytí slovní zásoby vyžaduje specifický přístup, pravidelné aktivity na klasifikaci a memorizování slov, opakované používání slov již naučených a vysvětlování neznámých termínů v jejich kontextu. Učitel dohlíží, aby se každý týden děti naučily nová slova k obohacení slovní zásoby. Děti se učí nejen slovíčka, která jim pomáhají k pochopení slyšeného textu, ale také slova k efektivní komunikaci ve škole a co nejpřesnější vyjádření vlastních myšlenek. Vyučující dává každému dítěti dostatek pozornosti, pomáhá mu se správnými slovy a podporuje ho. Přeformulovává pokus dítěte, aby slyšelo, jak zní správný model. Aby se děti mohly zdokonalovat v mluveném projevu, je jim vyučující příkladem správnosti vět a přesné slovní zásoby.

		Kompetence na konci mateřské školy
		- porozumět zprávě a reagovat nebo odpovědět na ni vhodným způsobem
		- s jistotou popsat objekt, osobu nebo událost z běžného života
		- srozumitelně vyjádřit otázku nebo popis
		- srozumitelně vyprávět neznámý příběh pro posluchače nebo příběh vymyšlený 
		- iniciativně se ptát na otázky a vyjadřovat svůj vlastní názor



	\subsection{OBJEVOVAT PÍSMO (DÉCOUVRIR L´ECRIT)}
		Mateřská škola připravuje děti na základní vzdělávání. Činnosti s mluveným projevem, navyšování slovní zásoby, písemná tvorba a četné poslechy vyprávěného a čteného textu učí žáky dovednostem čtení a psaní. Tři klíčové aktivity (cvičení na zvukovou stránku slov (fonémy), na základy abecedy a grafomotoriku) v mateřské škole významně podporují systematickou přípravu na čtení a psaní, která začíná v připravném vzdělávání (CP-cours préparatoir)

		\paragraph{Seznámit se s psaním (Se familiariser avec l´écrit)}
			\subparagraph{Objevování psaných podkladů (Découvrir les supports de l´écrit)}
			Děti objevují společenské užívání písemného projevu srovnáváním psaných podkladů, ve škole i mimo ni (plakáty, knihy, noviny, časopisy…). Učí se ho přesně popsat a pochopit jeho funkci. Pozorují a používají knihy, učí se orientovat na stránce i na přebalu knih. 

			\subparagraph{Objevování psaného jazyka (Découvrir la langue écrite)}
			Díky čteným textům se děti každý den seznamují s psanou francouzštinou. Aby mohly vnímat specifika psaného projevu, jsou vybírány texty jazykově kvalitní a s různými literárními žánry (pohádky, legendy, bajky, básně, říkanky). Po celou dobu mateřské školy jsou děti vedeny k vyprávění a osvojování si děl z literárního dědictví. Stávají se citlivější ke způsobům, jak vyjádřit méně známé skutečnosti. Jejich zvědavost je stimulována otázkami vyučujícího, který zdůrazňuje nová slova a slovní obraty, které poté používá i v jiných situacích.  Děti vyprávějí přečtený příběh, sdělují, co pochopily a doptávají se na nejasnosti. Jsou povzbuzovány k memorizování vět nebo krátkých úryvků textu. 

		\subparagraph{Základy psaní textu (Contribuer à l´écriture de textes)}
			Děti se účastní činností, jež přirozeně zanechávají stopu toho, co se stalo, bylo pozorováno nebo naučeno. Učí se diktovat text dospělému. Ten jim případnými otázkami pomáhá uvědomit si požadavky formy vyřčeného. Jsou též vedeni ke správnému výběru slov a syntaktické struktuře. Na konci mateřské školy dovedou děti transformovat spontánní mluvený projev na text, který dokáže dospělý zapsat podle diktátu.
			Kompetence na konci mateřské školy
			- rozlišovat zvuky
			- poznat slabiky vyřčeného slova, poznat stejnou slabiku v různých slovech
			- rozeznat a napsat většinu písmen abecedy
			- spojit hlásku s písmenem
			- pod vedením učitele napsat krátká jednoduchá slova z hlásek a písmen, které děti již znají
			- napsat své jméno

		\paragraph{Připravit na čtení a psaní (Se préparer à apprendre à lire et à écrire)}
			\subparagraph{Rozeznávání zvuku slov (Distinguer les sons de la parole)}
			Děti velmi brzo objevují radost ze hry se slovy a zvukomalebností jazyka. Nejdřív slabiky pokřikují, později si s nimi hrají (vynechávají slabiky, kombinují několik slabik dohromady v různém pořadí). Dokážou rozeznat stejné slabiky v jiných slovech a určit jejich pozici ve slově (na začátku, uprostřed, na konci)
			Postupně rozlišují zvuky hlásek a učí se operovat s nimi s dalšími jazykovými komponenty. Vyučující je pozorný k pokroku při osvojování si abstraktních hlasových aktivit.

			\subparagraph{Základy abecedy (Aborder le principe alphabétique)}
			Děti se seznamují se základy spojitosti mezi mluveným a psaným textem(pozn.autorky: Ve francouzském jazyce je rozdíl ve výslovnosti mezi psaným a mluveným projevem). Pozorováním známých věcí (datum, název příběhu nebo pohádky) nebo krátkých vět děti porozumí posloupnosti slov a faktu, že každé napsané slovo odpovídá slovu mluvenému. 

			Objevují, že každé slovo, které řeknou nebo které slyší, je složeno ze slabik a dávají si do spojitosti písmena a hlásky. Rozlišování hlásek je čím dál tím přesnější. Postupně se učí názvy všech písmen abecedy, které umí rozpoznat tiskace i psace, i přesto, že klasické pořadí písmen v abecedě ještě zůstává neznámé. U některých písmen si k jejich názvu asociují zvuk hlásky.(pozn. autorky – názvy písmen abecedy ve francouzském jazyce jsou odlišné od hlásek daných písmen ve slově). Tímto se jím dostává principům abecedy.


			\subparagraph{Základy grafomotoriky (Apprendre les gestes de l´écriture)}
			Každý den děti pozorují a reprodukují grafické motivy. Tím se učí nejefektivnějším gestům (pohybům). Vstup do světa psaní záleží na rozvinuté grafomotorice (spojování jednoduchých linií, vln apod.), ale vyžaduje i zvláštní dovednosti vnímat charakteristiky písmen. 
			Dětem ve Velké sekci, které na to již mají kapacity, je předkládáno i písmo psané. Jde o řízenou aktivitu pod dohledem vyučujícího. První zvyky, které si dítě osvojí, mají vliv na pozdější kvalitu zápisu a uvolněnost ruky při psaní. 

		Kompetence na konci mateřské školy
		- rozpoznat hlásky
		- rozlišovat slabiky vyřčených slov, poznat stejné slabiky v různých slovech
		- z krátkého výroku pospojovat mluvená slova s napsanými
		- rozpoznat a napsat většinu písmen abecedy
		- spojit hlásku s písmenem
		- pod vedením vyučujícího kopírovat psaným písmem krátká a jednoduchá slova, které dítě již zná
		- napsat psacím písmem své jméno

	\subsection{STÁT SE ŽÁKEM (DEVENIR ELEVE)}
		Cílem je dítě naučit, co ho odlišuje od ostatních a vnímat sám sebe jako osobnost. Naučit ho žít v organizované společnosti s pravidly. Chápat, co je to škola a jaké je v ní jeho místo. Stát se žákem je postupný proces, která od vyučujícího vyžaduje jak flexibilitu, tak přísnost.
		Život ve společnosti: jak se naučit pravidla společnosti a základy slušného chování (Vivre ensemble: apprendre les règles de civilité et les principes d´un comportement coforme à la morale)
		Děti objevují bohatství i nátlak skupiny, do které jsou začleňovány. Pociťují radost z přijetí, a že jsou poznány a postupně přijímají i ostatní kamarády. Kolektiv, ve kterém se děti v mateřské škole nacházejí, je vhodnou situací, kde se učí vést dialog mezi sebou, s dospělými a učí se, kdy na ně přijde řada. Toto je výborná příležitost k nácviku pravidel slušného chování, jako je pozdravení na začátku a na konci dne, odpovědět na otázku, poděkovat osobě, která nám pomohl a nepřerušovat ostatní. Zvláštní důraz je kladen na morální základy pravidel chování, jako je respektování ostatních osob a dobro bližních, povinnost přizpůsobit se pravidlům daných dospělými i respektovat, že mluví druhé dítě. 

		Spolupráce a samostatnost (Coopérer et devenir autonome)
		Účast ve hrách; kruzích či vytvořených skupinkách, které mají recitovat říkanku nebo poslouchat příběh, účast na realizaci společných projektů, apod., jsou aktivity, díky nimž děti objevují chuť společných aktivit a učí se spolupracovat. Zajímají se o ostatní a spolupracují s nimi. Přebírají zodpovědnost ve třídě a jsou iniciativní. Angažují se v projektech nebo činnostech s důrazem na své vlastní zdroje. Zakoušejí samostatnost, úsilí a vytrvalost. Chápou, co to je škola.
		Děti musí postupně pochopit, jaká jsou pravidla školní komunity, specifika školy, co se ve škole dělá, co se od nich očekává a co a proč se ve škole učí. Rozlišují rozdílné role rodičů a učitelů. 
		Postupně přijímají rytmus společných činností a dokážou odlišit uspokojení z jejich vlastních zájmů. Chápou hodnotu společných pravidel. Učí se, jak klást otázky a jak vyjednávat, aby dosáhly zadaného. Rozvíjejí si souvislosti mezi materiálními činnostmi a tím, co se danou aktivitou učí. Získávají objektivní kritéria pro evaluaci vlastních úspěchů. Na konci mateřské školy dokážou poznat vlastní chyby i chyby kamarádů. Učí se vydržet soustředit čím dál tím delší dobu. Objevují spojení mezi tím, co se učí a věcmi v běžném životě.
		Kompetence na konci mateřské školy
		- respektovat ostatní a respektovat pravidla společného života
		- poslouchat, pomáhat, spolupracovat, žádat o pomoc
		- důvěřovat sám sobě, kontrolovat vlastní emoce
		- identifikovat dospělé a jejich role
		- být samostatný v jednoduchých úkonech a hrát roli ve školních činnostech
		- říct, co se dítě naučilo

	\subsection{JEDNAT A VYJADŘOVAT SE VLASTNÍM TĚLEM (AGIR ET S´EXPRIMER AVEC SON CORPS)}
		Fyzická aktivita a zkušenosti s vlastním tělem přispívají k rozvoji motoriky, smyslů, citů a intelektu dítěte. Jsou příležitostí objevovat, vyjádřit se, jednat ve známém prostředí, později i prostředí známém méně a dovolují orientovat se v prostoru. Dítě objevuje možnosti svého těla.  V bezpečí se učí reagovat a přijímat možná rizika a využití adekvátního množství energie pro danou činnost. Vyjadřuje, co cítí, dokáže popsat činnosti a objekty, s kterými pracuje a používá je. Vyjadřuje, co má chuť dělat. Vyučující zaručuje  pestrou nabídku činností, zvyšování jejich náročnosti a dostatek možností ke sebezdokonalování. Také jim pomáhá důvěřovat sami sobě v nově nabytých dovednostech. 
		Skrze fyzickou aktivitu řízenou nebo volnou v různých prostředích rozvíjejí děti své motorické dovednosti, rovnováhu, manipulaci, házení a chytání. Hry s míčem, hry s protivníkem, hry s pravidly doplňují tyto aktivity. Děti řídí dané aktivity i jejich návaznost. Osvojují si motorické dovednosti, kdy je správně používat a jejich správné provedení. 
		Skrze činnosti s pravidly rozvíjí dovednosti adaptace a spolupráce. Díky nim chápou a přijímají pozitiva a negativa činností v kolektivu. 
		Umělecké činnosti jako kruh, tanec, pantomima pomáhají k vyjádření se gesty a k rozvoji představivosti.
		Pomocí rozličných činností nabývají děti obraz vlastního těla. Rozlišují před, za, nahoře, dole, vpravo, vlevo, blízko a daleko. Zvládají překážkové dráhy a dokážou popsat své pohyby a provedení.
		Kompetence na konci mateřské školy
		- přizpůsobit své pohyby v různých prostředích a omezeních
		- individuálně či společně spolupracovat nebo být proti
		- vyjádřit se hudebním rytmem, nástrojem, vyjádřit své pocity a emoce gesty a pohyby
		- orientovat se a pohybovat se v prostoru
		- popsat a vytvořit jednoduchou dráhu

	\subsection{OBJEVOVAT SVĚT (DECOUVRIR LE MONDE)}
		V mateřské škole dítě objevuje svět okolo něj. Učí se prostorovým a časovým omezením a jak se jim přizpůsobit. Pozoruje, klade otázky a dělá pokroky v účelnosti dotazování. Učí se přijímat jiný pohled na věc, než svůj vlastní.  Konfrontace a logické myšlenky mu dodávají chuť uvažovat nad věcmi. Naučí se počítat, třídit, uspořádávat a popsat věci jak jazykem, tak různými formami vyjadřování (obrázky, načrty). Začíná chápat rozdíly mezi živými a neživými objekty.
		Objevování objektů (Découvrir les objets)
		Děti objevují běžné technické předměty (baterka, telefon, počítač…), chápou jejich využití a funkce; k čemu slouží a jak je používat. Naučí se i znaky nebezpečných předmětů.
		Vyrábějí předměty použitím různých materiálů a vybírají si vhodné nástroje a techniky (stříhání, lepení, ohýbání, skládání, připíchnutí, složení a rozložení…)
		Objevování materiálu (Découvrir la matière)
		Znalosti o charakteristických vlastnostech materiálů získávají díky činnostem jako je stříhání, modelování, spojování běžných materiálů (dřevo, půda, papír, karton, voda, atd.)
		Uvědomují si i méně viditelnou realitu jako je vítr a začínají vnímat změny skupenství vody. 
		Objevování živého (Découvrir le vivant) 
		Děti objevují různé projevy živé přírody. Chov dobytka a hospodářství jsou významným prostředkem k objevování cyklu života od narození, přes růst, reprodukci po stárnutí až smrt.
		Objevují části těla a pět smyslů, jejich charakteristiku a funkci. Zajímají se o hygienu a zdraví a hlavně o stravu. Učí se základním pravidlům hygieny těla. 
		Jsou vnímaví k problémům životního prostředí a učí se respektovat život. 
		Objevování formy a rozměrů (Découvrir les formes et les grandeurs)
		Manipulací s různými předměty děti odhalují nejdříve jednoduché vlastnosti (malý/velký; těžký, lehký), poté dokážou rozlišovat základní kriteria. Srovnávat a třídit podle tvaru, velikosti, množství a obsahu.
		Přibližování veličinám a číslům (Découvrir les quantités et les nombres)
		Mateřská škole je klíčovým obdobím k získání povědomí o posloupnosti čísel a jejich využití v určování množství. Děti objevují a učí se chápat funkce čísel, zvláště jako prostředek k vyjádření množství a označení pořadí objektů v řadě.
		Činnosti jako rozdělování, srovnávání a třídění ovlivňuje přístup dětí k vnímání celku. Děti se postupně učí počítat nejméně do 30 a základům jednoduchých počtů.
		Čísla jsou používána v situacích, která dávají dětem smysl a jsou praktickým prostředkem k dosažení cíle: hry, činnosti ve třídě, zadané úkoly na srovnávání, spojování, řazení a rozdělování. Velikost celku a možnost reagovat na předměty jsou důležité proměnné, které vyučující přizpůsobuje kapacitám každého dítěte. 
		Konec mateřské školy je časem prvních kroků do světa počtů. 
		Tím je psaní číslic v konkrétních situacích (př. kalendář) nebo při hrách (přemísťování se po značkách s číslicemi). Děti si vytvářejí první spojení mezi ústním označením a psaným označením číslic. Jejich výkon zůstává velmi rozdílný, ale je důležité, aby se začali této dovednosti učit. Na učení se psaní číslic je dávána stejná důslednost jako na psaní písmen. 
		Orientace v čase (Se repérer dans le temps)
		Pravidelnou organizací rozvrhu, děti pozvolna vnímají časovou posloupnost dne, týdnů, měsíců. Na konci mateřské školy chápou cykličnost určitých fenoménů (roční období) a znázornění času (týden, měsíc). Pojem časová posloupnost je utvrzován v činnostech i známých příbězích. Grafické znázornění napomáhá k jejich utvrzení (obrázky, kresby).
		V malé sekci (Petit section) používají děti k orientaci v chronologii a měření času kalendáře, hodiny a přesýpací hodiny. V přípravné třídě se prohlubují tyto limitované znalosti. Popisováním příběhů, které se již staly nebo pozorováním rodinného odkazu, se děti učí blízké minulosti a s většími obtížemi i vzdálené budoucnosti.
		Tyto činnosti dávají prostor k učení se přesné slovní zásoby, která je opakovaným používáním, zvláště při rituálech, fixována.
		Orientace v prostoru (Se repérer dans l´espace)
		Po celou dobu mateřské školy se děti učí pohybovat se po prostorách a nejbližším okolí školy. Dokážou si najít své místo ve vztahu k věcem a ostatním osobám a lokalizovat věci a osoby v prostoru, což předpokládá schopnost oprostit se od svého vlastního pohledu na věc. Na konci mateřské školy rozlišují pravou a levou stranu. Děti zvládnou projít trasu podle značek a povelů (příkazy a grafické znázornění).
		Činnosti, při kterých děti musí přecházet z vertikálního plánu do horizontálního a naopak a udržovat relativní postavení předmětů nebo znázorněné prvky, jsou předmětem mimořádné pozornosti. Připravují je na orientaci v grafickém prostoru. Orientace v prostoru na stránce nebo na papíře a orientace na lince je spojena s dovednostmi čtení a psaní. 
		Kompetence na konci mateřské školy
		- rozeznat, vyjmenovat, popsat, porovnat, uspořádat, třídit materiál a předměty podle jejich kvalit a užití
		- znát projevy života zvířat i rostlin, spojit je s vyšší funkcí: narození, výživa, pohyb, reprodukce
		- vyjmenovat hlavní části těla a jejich funkce, rozlišit pět smyslů a k čemu slouží
		- umět a aktivně užívat pravidla hygieny těla, prostředí, stravování
		- rozpoznat a uvědomovat si nebezpečí
		- orientovat se ve dnech, týdnech, měsících
		- dokázat určit událost ve vztahu k ostatním událostem
		- nakreslit kruh, čtverec, trojúhelník
		- porovnat počet, vyřešit početní úkol
		- umět nazpaměť, jak jdou číslice do 30 za sebou 
		- slovně vyjádřit  množství v číslicích 
		- orientovat se v prostoru a oproti ostatním předmětům
		- orientovat se na stránce
		- užívat správná slovní vyjádření při popisu vztahu času a prostoru

	\subsection{VNÍMAT, CÍTIT, PŘEDSTAVOVAT SI, TVOŘIT}
		Mateřská škola nabízí první setkání s citem pro umění. Vizuální, hmatové, sluchové a hlasové činnosti zvyšují smyslové schopnosti dětí. Pobízí jeho představivost a obohacují jeho vyjadřovací znalosti a kapacity. Přispívají k rozvoji pozornosti a koncentrace. Poslech a pozorování jsou příležitostmi seznámit dítě s formami uměleckého vyjádření. Poznávají své emoce a vstupují do uměleckého světa.
		Tyto činnosti souvisí též s ostatními obory vzdělávání, podporují zvídavost k objevování světa, dovolují dětem vyjadřovat se pohybem, podporují vyjádření vlastních reakcí a chutí a dávají možnost výběru v rámci interakce s ostatními.
		Výkresy a prostorové kompozice (výroba předmětů) jsou oblíbenými prostředky vyjadřování.
		Děti experimentují s mnoha nástroji, které jim pomáhají tvořit výkresy. Objevují, používají předměty různé povahy a realizují obrazy. Tvoří předměty využitím malby, lepením papíru, koláží, asambláží, modelováním…
		V tomto kontextu, vyučující pomáhá dětem vyjádřit to, co vnímají a podporuje vlastní zahájení projektů a jejich realizaci. Přitom je učí používání správných slovíček. Povzbuzuje děti, aby si založily osobní sbírku předmětů s estetickou a citovou hodnotou.
		Hlas a sluch jsou prostředky komunikace, které děti velmi brzo objevují při hře se zvuky, při zpěvu a při pohybu.
		Repertoár říkanek a písní vycházející z ústní tradice, do které spadají i moderní autoři, se každým rokem obohacuje. Děti zpívají pro radost a hrají si s hlasem, hlukem a rytmy.
		Strukturované poslechové činnosti bystří pozornost, rozvíjí citlivost rozlišovat zvuky a zvukovou paměť. Děti poslouchají pro zábavu, aby mohly napodobovat, kvůli pohybu a v rámci hry. Učí se rozlišovat barvu, intenzitu, dobu, výšku tónu. Srovnávají, napodobují a určují jejich znaky. Poslouchají různá hudební díla. Hledají nové možnosti užití hudebních nástrojů. Po částech se učí rytmus a tempo. 
		Kompetence na konci mateřské školy
		- přizpůsobit svou činnost omezení materiálu (nástroje, prostředky, materiál)
		- používat obrázky jako prostředek sebevyjádření
		- realizovat výtvor podle svých představ (plán, velikost)
		- pozorovat a popsat umělecká díla a vytvářet si vlastní sbírku
		- zapamatovat si a interpretovat písně a říkanky
		- poslouchat část z hudebního díla a potom se k němu vyjádřit a hovořit o svých dojmech


	Tady bude srovnání kurikul, tabulka a popis….snad dnes stihnu






\section{ČESKÉ KURIKULUM}
	K dnešnímu dni je obsah vzdělávání v České republice řešen na dvou úrovních, na úrovni státní a na úrovni školské. Toto umožňuje nový Školský zákon č.561/2004 o předškolním, základním, středním, vyšším odborném a jiném vzděláním, který vešel v platnost 1. ledna 2005 a je výsledkem probíhající školské reformy a Národního programu pro rozvoj vzdělávání v České republice (Bílá kniha), který formuje vládní strategii v oblasti vzdělávání, odráží celospolečenské zájmy a dává konkrétní podněty k práci škol.

	Na školské úrovni si každá mateřská škola vytváří svůj vlastní školní vzdělávací plán, který vychází ze základů Rámcového vzdělávacího programu a principů v něm stanovených. Rámcový vzdělávací plán je zpracováván na úrovni státní a je dokumentem, jenž nedává návod, jak s dětmi pracovat, ale určuje principy a směr, kudy by se předškolní vzdělávání mělo ubírat. Rámcový vzdělávací program předškolního vzdělávání je závazně platný od 1. 9. 2005.

	Rámcové i školní vzdělávací programy jsou veřejné dokumenty dostupné pro celou veřejnost. Rámcový vzdělávací program pro předškolní vzdělávání je ke stažení na stránkách MŠMT ČR (Ministerstva školství, mládeže a tělovýchovy) \cite{cejpek98}
	%TODO: http://www.msmt.cz/vzdelavani/skolstvi-v-cr/skolskareforma/ramcove-vzdelavaci-programy 
	nebo na stránkách Národního ústavu pro vzdělávání \cite{cejpek98}.
	% TODO: http://www.nuv.cz/ramcove-vzdelavaci-programy.


	Rámcový vzdělávací program pro předškolní vzdělávání je rozdělen do 12 kapitol.
	1. Vymezení Rámcového vzdělávacího programu pro předškolní vzdělávání v systému kurikulárních dokumentů
	2. Předškolní vzdělávání v systému vzdělávání a jeho organizace
	3. Pojetí a cíle předškolního vzdělávání
	4. Vzdělávací obsah RVP PV
	5. Vzdělávací oblasti
	6. Vzdělávací obsah ve školním vzdělávacím programu 
	7. Podmínky předškolního vzdělávání
	8. Vzdělávání dětí se speciálními vzdělávacími potřebami a dětí mimořádně nadaných
	9. Autoevaluace mateřské školy a hodnocení dětí
	10. Zásady pro zpracování školního vzdělávacího programu
	11. Kriteria souladu rámcového a školního vzdělávacího programu
	12. Povinnost předškolního pedagoga

	Pro tuto práci a možnosti komparace s francouzským kurikulem se budu podrobněji zabývat jen některými částmi RVP PV. A to kapitolou 3 Pojetí a cíle předškolního vzdělávání a kapitolou 5 Vzdělávací oblasti. 

	6.1.1 SHRNUTÍ  POJETÍ A CÍLŮ PŘEDŠKOLNÍHO VZDĚLÁVÁNÍ

	V této kapitole se zaměřuji převážně na informace týkající se dětí a jejich vzdělávání.

	Mezi hlavní principy RVP PV patří akceptování přirozených vývojových specifiky dětí předškolního věku, vzdělávání dítěte v rozsahu jeho individuálních možností a potřeb, vytváření základů a osvojení si klíčových kompetencí (viz níže) dosažitelných v etapě předškolního vzdělávání a získávání předpokladů pro celoživotní vzdělávání.
	Předškolní vzdělávání má dítěti usnadňovat jeho další životní i vzdělávací cestu, rozvíjet jeho osobnost, tělesný rozvoj a zdraví, osobní spokojenost a pohodu a napomáhat mu v chápání okolního světa a motivovat jej k dalšímu poznávání, stejně tak jako učit dítě žít ve společnosti ostatních a přibližovat mu normy a hodnoty uznávané touto společností.

	Vhodnými metodami vzdělávání je dle RVP PV prožitkové a kooperativní učení hrou. Jde o činnosti, které jsou založeny na přímých zážitcích dítěte, které podporují dětskou zvídavost a potřebu objevovat, podněcují jeho radost z učení a jeho zájem poznávat nové věci a získávat nové zkušenosti.

	RVP řadí situační učení a spontánní sociální učení mezi významné procesy učení, které by měly být dostatečně zastoupeny. Aktivity by se měly střídat spontánní i řízené a měly by být vzájemně provázané a vyvážené. Pedagog by měl být průvodcem dítěte, probouzet v něm aktivní zájem a chuť dívat se kolem sebe, naslouchat a objevovat. Není zde ten, co „úkoluje“ a kontroluje. Didaktický styl by měl být založen na principu vzdělávací nabídky, na individuální volbě dítěte a jeho aktivní účasti.

	Důležitou složkou RVP PV jsou výstupy vzdělávacích cílů. Těmi jsou klíčové kompetence, neboli kompetence, které by děti měly ovládat na konci mateřské školy a před vstupem do školy základní. Je zde uváděno 5 kompetencí, ke každé z nich uvedu opět stručný výtah:

	kompetence k učení
	Dítě má elementární poznatky o světě lidí, kultuře, přírodě a technice, která ho obklopuje, orientuje se v řádu dění. Klade otázky a hledá na ně odpovědi, aktivně si všímá a chce porozumět jevům, které kolem sebe vidí. Soustředěně pozoruje, objevuje, experimentuje a získanou zkušenost dále uplatňuje. Učí se nejen spontánně, ale i vědomě, soustředí se na činnost a práci dokončí. Dokáže postupovat podle instrukcí, odhaduje své síly a učí se s chutí.

	kompetence k řešení problémů
	Dítě si všímá dění i problémů okolo sebe, známé situace řeší samostatně, náročnější s oporou a pomocí dospělého. Problémy řeší na základě bezprostřední zkušenosti, cestou pokusu a omylu, experimentuje, vymýšlí nová řešení, hledá varianty. K tomu všemu využívá dosavadních zkušeností, fantazii a představivost. Užívá logických, matematických a empirických postupů. Dokáže volit mezi řešením vedoucím k cíli a řešením nefunkčním. Nebojí se chybovat.

	kompetence komunikativní
	V běžných situacích dítě komunikuje bez zábran a ostychu s dětmi i s dospělými, ovládá řeč, samostatně vyjadřuje své myšlenky ve vhodně formulovaných větách. Dokáže sdělovat své prožitky a pocity a to všemi prostředky (i výtvarnými, hudebními, dramatickými). Dítě ovládá dovednosti předcházející čtení a psaní, ví, že existují jiné jazyky a průběžně rozšiřuje slovní zásobu, kterou aktivně používá. Dovede využít informativní a komunikační prostředky (telefon, knihy, počítač..)

	kompetence sociální a personální
	Dítě se samostatně rozhoduje o svých činnostech, umí si vytvořit a vyjádřit svůj názor. Uvědomuje si, že odpovídá za své jednání, projevuje citlivost a ohleduplnost k druhým, vnímá nespravedlnost a ubližování. Při společenských činnostech se domlouvá a spolupracuje, uplatňuje pravidla společenského styku, dodržuje dohodnutá pravidla a je schopné respektovat druhé a uzavírat kompromisy. Umí být tolerantní k odlišnostem druhých lidí. Dokáže se bránit násilí a ponižování.

	kompetence činnostní a občanské
	Dítě dokáže rozpoznat svoje silné a slabé stránky. Učí se plánovat, organizovat a vyhodnocovat svoje činnosti a hry. Odhaduje rizika svých nápadů a dokáže se přizpůsobovat okolnostem. Chápe, že se může svobodně rozhodovat, ale i že nese odpovědnost za svá rozhodnutí. Zajímá se o druhé a co se děje kolem. Má smysl pro povinnost ve hře, práci i učení. Uvědomuje si svá práva i práva druhých. Též si uvědomuje, že se svých chováním ovlivňuje prostředí, ve kterém žije a dbá na osobní zdravá a bezpečí své i druhých. 

	6.1.2 SHRNUTÍ Z OBSAHU V RVP A VZDĚLÁVACÍCH OBLASTÍ

	Obsah RVP je udáván pouze obecně a rámcově, slouží jako prostředek k naplňování
	vzdělávacích záměrů a dosahování vzdělávacích cílů. Vzdělávací obsah je RVP PV je rozdělen do pěti oblastí:
	1.Dítě a jeho tělo
	2.Dítě a jeho psychika
	 3.Dítě a ten druhý
	 4.Dítě a společnost
	 5.Dítě a svět
	Každá oblast obsahuje 4 části. Těmi jsou dílčí vzdělávací cíle (co pedagog u dítěte podporuje), vzdělávací nabídka (co pedagog dítěti nabízí), očekávané výstupy (co dítě na konci předškolního období zpravidla dokáže) a rizika (co ohrožuje úspěch vzdělávacích záměrů pedagoga).


	1. Dítě a jeho tělo
	Záměrem vzdělávacího úsilí v oblasti biologické je stimulovat a podporovat růst a neurosvalový vývoj dítěte, podporovat jeho fyzickou pohodu, zlepšovat jeho tělesnou zdatnost i pohybovou a zdravotní kulturu, podporovat rozvoj jeho pohybových a manipulačních dovedností, učit je sebeobslužným dovednostem a vést je k zdravým životním návykům a postojům.
	Dílčí vzdělávací cíle 
	Uvědomění si vlastního těla, zdokonalení jemné a hrubé motoriky, koordinace, ovládání pohybového aparátu, rozvoj všech smyslů, rozvoj fyzické i psychické zdatnosti, základní poznatky o těle, zdraví a zdravém životním stylu.
	Vzdělávací nabídka
	Lokomoční pohybové hry, manipulační činnosti, smyslové a psychomotorické hry, konstruktivní a grafické hry, hudební a hudebně pohybové hry, sebeobslužné činnosti, relaxační a odpočinkové činnosti, příležitosti a činnosti směřující k prevenci úrazů, nemoci, nezdravých návyků a závislostí
	Očekávané výstupy
	Správné držení těla, zvládnutí základních pohybových dovedností a prostorové orientace, koordinace lokomoce a sladění pohybu s hudbou, napodobit jednoduchý pohyb podle vzoru, ovládat dechové svalstvo, vnímat a rozlišovat pomocí všech smyslů, zvládat koordinaci ruky a oka a jemnou motoriku, zvládnout sebeosluhu a hygienické návyky, zvládat pracovní úkony, pojmenovat části těla a znát jejich funkce, rozlišovat, co prospívá zdraví, mít povědomí o péči o čistotu, o způsobech ochrany osobního zdraví a kde hledat pomoc, zacházet s předměty denní potřeby.

	2. Dítě a jeho psychika
	Záměrem vzdělávacího úsilí pedagoga v oblasti psychologické je podporovat duševní pohodu, psychickou zdatnost a odolnost dítěte, rozvoj jeho intelektu, řeči a jazyka, poznávacích procesů a funkcí, jeho citů i vůle, stejně tak i jeho sebepojetí a sebenahlížení, jeho kreativity a sebevyjádření, sitmulovat jeho osvojování a rozvoj jeho vzdělávacích dovedností a povzbuzovat je v dalším rozvoji a učení.
	Tato oblast zahrnuje tři „podoblasti“: Jazyk a řeč‘; Poznávací schopnosti a funkce, představivost a fantazie, myšlenkové operace; Sebepojetí, city a vůle.

	Jazyk a řeč
	Dílčí vzdělávací cíle
	Rozvoj řečových schopností a jazykových dovedností receptivních i produktivních, rozvoj komunikativních dovedností, osvojení si dovedností předcházejících čtení i psaní
	Vzdělávací nabídka
	Artikulační, řečové, sluchové a rytmické hry, individuální a skupinová konverzace, vyprávění, komentování zážitků, vyřizování vzkazů, poslech pohádek, sledování filmových a divadelních příběhů, vyprávění, přednes, recitace, zpěv, grafické napodobování symbolů, poznávání a rozlišování zvuků a gest, seznámení se se sdělovacími prostředky
	Očekávané výstupy
	Správně vyslovovat, ovládat dech, tempo a intonaci řeči, vyjadřovat myšlenky, nápady, pocity, vést rozhovor, domluvit se slovy i gesty, porozumět slyšenému, formulovat otázky, odpovídat, slovně reagovat, naučit se krátké texty zpaměti, sledovat a vyprávět příběh, popsat situaci, chápat slovní humor, sluchově rozlišovat začáteční a koncové slabiky a hlásky ve slovech, utvořit jednoduchý rým, rozlišovat obrazné symboly, poznat některá písmena, číslice, své jméno, zájem o knížky, hudbu, film.

	Poznávací schopnosti a funkce, představivost a fantazie, myšlenkové operace
	Dílčí vzdělávací cíle
	Rozvoj smyslového vnímání, přechod k pojmovému myšlení, rozvoj paměti, pozornosti, představivosti a fantazie, rozvoj tvořivosti, posilování poznávacích citů (zvídavost, zájem…), podpora a rozvoj zájmu o učení, osvojení si elementárních poznatků o znakových systémech, základ práce s informace.
	Vzdělávací nabídka
	Pozorování přírodních, kulturních i technických objektů a jevů, pojmenovávání jejich vlastností a charakteristických znaků, motivovaná manipulace s předměty, konkrétní operace s materiálem, smyslové hry, hry na rozvoj postřehu a vnímání, zrakové a sluchové paměti, pozornosti a různých forem paměti, námětové hry, hry podporující tvořivost, představivost a fantazii, řešení myšlenkových i praktických problémů a hledání variant řešení, činnosti k seznámení se s matematickými pojmy a jejich symbolikou, činnosti zasvěcující dítě do časových pojmů.
	Očekávané výstupy
	Vědomě využívat všech smyslů, záměrně pozorovat, všímat si, soustředit se a udržet pozornost, poznat a pojmenovat většinu toho, čím je obklopeno, přemýšlet a vést jednoduché úvahy, využívat zkušeností k učení, postupovat a učit se podle instrukcí, chápat základní číselné a matematické pojmy, souvislosti a prakticky je využívat, chápat prostorové pojmy, elementární časové pojmy, naučit se nazpaměť krátké texty, řešit problémy, myslet kreativně, nalézat nová řešení, vyjadřovat svou představivost v tvořivých činnostech.

	Sebepojetí, city, vůle
	Dílčí vzdělávací cíle
	Poznávání sebe sama, rozvoj pozitivních citů k sobě, získání relativní citové samostatnosti, rozvoj schopnosti sebeovládání vytváření citových vazebh, rozvoj schopností vyjádřit prožitky a dojmy, rozvoj mravního i estetického vnímání, získání schopnosti záměrně řídit svoje chování a ovlivňovat vlastní situaci.
	Vzdělávací nabídka
	Spontánní hra, činnosti vyvolávající spokojenost, veselí, pohodu, úkoly, v nichž může být dítě úspěšné, činnosti vyžadující samostatné vystupování, obhajování vlastních názorů, rozhodování, sebehodnocení, hry pro rozvoj vůle a sebeovládání, cvičení organizačních dovedností, estetické a tvůrčí aktivity, cvičení v projevování citů, v sebekontrole a v sebeovládání, hry na téma rodiny apod., výlety do okolí, činnosti k poznávání různých lidských vlastností, dramatické činnosti, mimické vyjadřování, činnosti vedoucí k identifikaci sebe sama a k odlišení od ostatních.
	Očekávané výstupy
	Odloučit se na určitou dobu od rodičů a blízkých, uvědomovat si svou samostatnost, zaujímat vlastní názory, rozhodovat o svých činnostech, vyjádřit souhlas i nesouhlas, uvědomovat si své možnosti a limity, přijímat pozitivní ocenění i případný neúspěch a vyrovnat se s tím, prožívat radost ze zvládnutého, vyvinout volní úsilí, soustředit se na činnost a dokončit ji, respektovat pravidla, zorganizovat hru, rozlišovat citové projevy v důvěrném a cizím prostředí, prožívat a projevovat, co cítí, snažit se ovládat afektivní chování, být citlivý k živým bytostem, přírodě i věcem, těšit se z příjemných zážitků, zachytit a vyjádřit své pocity.

	3. Dítě a ten druhý
	Záměrem vzdělávacího úsilí pedagoga v interpersonální oblasti je podporovat utváření vztahů dítěte k jinému dítěti či dospělému, posilovat, kultivovat a obohacovat jejich vzájemnou komunikaci a zajišťovat pohodu těchto vztahů.
	Dílčí vzdělávací cíle
	Seznamování se s pravidly chování k druhému, osvojení si schopností a dovedností pro navazování a rozvíjení vztahů, posilování prosociálního chování, vytváření prosociálních postojů, rozvoj komunikativních a kooperativních dovedností, ochrana osobního soukromí.
	Vzdělávací nabídka
	Běžné komunikační aktivity dítěte s druhými, sociální hry, hraní rolí, dramatické činnosti, hudební a hudebně pohybové hry, společenské hry a aktivity, kooperativní činnosti, společná setkávání a naslouchání ostatním, aktivity podporující uvědomování si vztahů mezi lidmi, činnosti na porozumění pravidlům vzájemného soužití, hry vedoucí k ohleduplnosti k druhému, ochotě rozdělit se, pomoci si, vyřešit spor, činnosti na poznávání sociálního prostředí-rodina, mateřská škola, hry a situace, kdy se dítě učí chránit soukromí a bezpečí své i druhých, četba, vyprávění a poslech příběhů s etickým obsahem a ponaučením.
	Očekávané výstupy
	Navazovat kontakty s dospělým, kterému je svěřeno do péče, komunikovat s ním, respektovat ho, porozumět projevům emocí a nálad, přirozeně komunikovat s druhým dítětem, navazovat přátelství, uvědomovat si svá práva, respektovat práva ostatních, spolupracovat s ostatními, uplatňovat své individuální potřeby a přání s ohledem na druhé, dodržovat pravidla vzájemného soužití v různých prostředích i pravidla her, respektovat potřeby jiného dítěte, dělit se s ním o věci, vycházet vstříc ostatním a pomáhat jim, bránit se projevům násilí, chovat se obezřetně při setkání si neznámými dětmi a dospělými.

	4. Dítě a společnost
	Záměrem vzdělávacího úsilí pedagoga v oblasti sociálně-kulturní je uvést dítě do společenství ostatních lidí a do pravidel soužití s ostatními, uvést je do světa materiálních i duchovních hodnot, do světa kultury a umění, pomoci dítěti osvojit si potřebné dovednosti, návyky i postoje a umožnit mu aktivně se podílet na utváření společenské pohody ve svém sociálním prostředí.
	Dílčí vzdělávací cíle
	Poznávání pravidel společenského soužití a jejich spoluvytváření, porozumění základním projevům neverbální komunikace v tomto prostředí, rozvoj schopnosti žít ve společenství ostatních lidí a přijímat základní hodnoty v tomto společenství uznávané, rozvoj základních kulturně společenských postojů, rozvoj schopnosti projevovat se autenticky a autonomně, vytvoření povědomí o morálních hodnotách, seznamování se světem lidí, kultury a umění, vytváření povědomí o jiných kulturách, rozvoj společenského i estetického vkusu.
	Vzdělávací nabídka
	Setkávání s pozitivními vzory vztahů a chování, aktivity pro adaptaci dítěte v MŠ, společenské hry a skupinové aktivity umožňující dětem se spolupodílet na jejich průběhu, přípravy a realizace společenských zábav a slavností, tvůrčí a receptivní činnosti slovesné, literární, dramatické, výtvarné apod., setkávání se s uměním mimo MŠ, návštěvy kulturních a uměleckých míst, hry na poznávání různých společenských  rolí, aktivity přibližující pravidla vzájemného styku a mravní hodnoty, hry a praktické činnosti uvádějící dítě do světa lidí, jejich občanského života a práce, aktivity přibližující svět kultury a umění a umožňující poznat rozmanitost kultur.
	Očekávané výstupy
	Uplatňovat návyky společenského chování ve styku s dospělými i dětmi, pochopit, že každý má ve společenství svou roli a podle ní se chovat, chovat se dle vlastních pohnutek, ale s ohledem na druhé, začlenit se do třídy a respektovat rozdílné vlastnosti vrstevníků, porozumět běžným neverbálním projevům citových prožitků a nálad druhých, adaptovat se ve škole a zvládat požadavky prostředí, vyjednávat s ostatními a domluvit se na společném řešení, utvořit si základní dětskou představu o pravidlech chování a společenských normách a chovat se dle toho, chovat se zdvořile, s úctou a bez předsudků, dodržovat pravidla her, jednat spravedlivě, hrát fair, odmítat společensky nežádoucí chování a chránit se pře ním, vnímat umělecké podněty a hodnotit svoje zážitky, vyjadřovat skutečnost pomocí různých výtvarných technik, vyjadřovat se prostřednictvím hudebních a hudebně pohybových činností.

	5. Dítě a svět
	Záměrem vzdělávacího úsilí pedagoga v environmentální oblasti je založit u dítěte elementární povědomí o okolním světě a jeho dění, o vlivu člověka na životní prostředí – počínaje nejbližším okolím a konče globálními problémy celosvětového dosahu – a vytvořit elementární základy pro otevřený a odpovědný postoj dítěte (člověka) k životnímu prostředí.

	Dílčí vzdělávací cíle
	Seznamování se a vytváření pozitivního vztahu k místu a prostředí, ve kterém dítě žije, poznávání jiných kultur, vytváření povědomí o širším přírodním, kulturním i technickém prostředí, pochopení, že lidská činnost může prostředí chránit, ale i ničit, osvojení si poznatků péče o okolí a spoluvytváření zdravého prostředí, rozvoj úcty k životu ve všech jeho formách, rozvoj schopnosti přizpůsobovat se podmínkám prostředí, vytvoření povědomí o vlastní sounáležitosti se světem.
	Vzdělávací nabídka
	Přirozené pozorování prostředí a života v něm, aktivity zaměřené na praktickou orientaci v obci, účast na zajímavých akcích v obci, poučení o možných nebezpečných situacích a způsobech, jak se chránit, aktivity na téma dopravy, cvičení bezpečného chování v dopravních situacích, poznávání přírodního okolí, sledování rozmanitostí a změn v přírodě, využívání encyklopedií a obrazového materiálu, kognitivní činnosti, praktické činnosti k seznámení s materiály, pozorování životních podmínek a životního prostředí, ekohry, smysluplné činnosti přispívající k péči o životní prostředí a okolní krajinu.
	Očekávané výstupy
	Bezpečně se orientovat ve známém prostředí, zvládat běžné činnosti a požadavky na dítě kladené, chovat se přiměřeně a bezpečně doma i na veřejnosti, uvědomovat si nebezpečí a jak se může chránit, osvojit si elementární poznatky o okolním prostředí, vnímat, že svět má svůj řád, že je rozmanitý a pestrý, všímat si změn v okolí a porozumět, že změny jsou přirozené a samozřejmé, mít povědomí o významu životního prostředí, pomáhat pečovat o okolní životní prostředí a rozlišovat aktivity, které mohou zdraví okolního prostředí podporovat a které poškozovat.
