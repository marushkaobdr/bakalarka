od Uhlirovy

Přes formální shodu postavení mateřské školy ve vzdělávacím systému obou sledovaných zemí je třeba poukázat na naprosto zásadní rozdíly v~cílech mateřských škol a~pohledu na dítě, které ji navštěvuje.
Předškolní výchova ve Francii tak jako ve většině románských zemí je charakteristická tím, že naplňuje cíl uvádět dítě do světa školy. Tzn. směřovat práci v~mateřské škole k~přípravě na vstup do povinné školní docházky. Tento přístup je hluboce tradičně zakořeněn, a~tak jak.....při rozboru kurikul směřuje k~získání základních kulturních technik, na nichž je ... postaven počátek primárního vzdělávání. 

Předškolní vzdělávání v~České republice nebylo takto jednoznačně orientované ve své historii, tj. příprav na školu, přeste však tento aspekt vyplynul jako nezbytnost s~přijetím dokumentu \uv{Další rozvoj výchovně vzdělávací soustavy} v~roce 1976. Tehdy byl cíl mateřských škol zúžen na přípravu pro povinné vzdělávání. děti, které neabsolvovaly mateřskou školu, byly při zápisech do základní školy ....(zarazeni) k~náhradnímu opatření, tzn. přípravných tříd, alespoň na dobu 3 měsíců, neboť dovednosti a~znalosti získané před nastupem do 1. třídy byly východiskem, na nichž 1. třída \uv{startovata}. Současná mateřská škola není vázána konceptem na školu, její koncept je mnohem širší. Příprava na život v~sobě zahrnuje také připravu na povinnou školní docházku. Ovšem v~kontextu socializace a~radostného dětství s~ostatnimi dětmi, tj. \uv{rosteme společně}. Současná předškolní výchova v~České republice neni ani školský model, ani rodinný model, ale je to smíšený model obou černobíle postavených typů.

Vzdělávací systemy jsou odlišné, i~když by se na první pohled zdálo, že mateřská škola přijímající děti od 2 do 6 let má svou stejnou pozici. Historicky byla francouzská mateřská škola vyjímána vždycky jako vzdělávací instituce. Opravdu tvoří první článěk vzdělávací soustavy (ale nepovinný), o~to je překvapivější, že cykly, které dítě v~předškolním věku prochází jsou vnímány jako nezbytný obsah na niž se váže povinná školní docházka. Tento stav se jeví jako anomálie. Přestože je nepovinná, 100\% 5ti letých dochází. Všechny rodiny, které žádají o~vstup dětí ve 3 letech jsou přijati (ne 2letí).

Tradice české mateřské školy je rovněž velmi dlouhá, ale její pozice jako vzdělávací instituce se vztahuje až ke školskému zákonu, kdy je zařazena jako první článěk vzdělávací sousty. Po roce 1948 se pozice MŠ více blížila sociálnímu zařízení, než skutečně výchovně vzdělávacímu. Mezi lety 1948 a~1989 je její vzdělávací charakter nespochybnitelný. Po roce 1989 byl krátkodobě zpochybněn vzdělávací charakter ve prospech pozice sociální, avšak na konci 90. let, zvláště pak školským zákonem v~roce 2004 se její pozice zakotvila a~posílila. 