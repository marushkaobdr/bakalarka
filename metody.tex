
\chapter{METODY}
\label{metody}

V~bakalářské práci byly využity tři výzkumné metody, a~to obsahová analýza, komparace a~pozorování. Obsahová analýza se týká kurikurálních dokumentů a cílů a~vzdělávacích oblastí předškolního vzdělávání Francie a~České republiky, které je věnována kapitola \ref{kurikulum}. Pozorování se konalo během průběžné praxe, jak ve Francii, tak v České republice a~věnuje se mu kapitola \ref{rezim}. Metody komparace je využito v~obou zmiňovaných kapitolách. 

Obsahová analýza je důležitým nástrojem poznání jednotlivých oblastí výchovy a vzdělávání. Jedná se o~velmi mladou výzkumnou metodu ze 40. let 20. století, která byla původně využívána v masmédiích a~postupně si nacházela své místo i v humanitních oborech a~v neposlední řadě i~v pedagogice. Lze ji aplikovat nekvantitativním nebo kvantitativním způsobem. V této práci se jedná o~první, nekvantitativní způsob, kdy se nejedná o~převedení kvalitativních parametrů (pojmy, slova, témata) na kvantitativní míru či numerickou hodnotu, ale o~popis a~rozbor obsahu dokumentů, tedy dokumentů kurikulárních a~jeho následné srovnání \citep{Gavora08}.

Komparace (srovnávání) je velmi používaná vědecká metoda. Umožňuje stanovit shody a~rozdíly jevů či objektů. Při komparaci se zjišťují shodné či rozdílné znaky různých předmětů, jevů či ukazatelů. 
Komparaci je možné rozdělit na dva způsoby: \citep[s.~19]{Siroky}
\begin{itemize}
\item []\uv{\textit{Srovnávání pojetí problémů, názorů, premis jako vytváření, ověřování či zdůvodňování vlastního stanoviska (postupu, úvah);}
\item []\textit{Srovnávání jako nástroj měření, zjišťování, objektivizace a~hodnocení dosažených výsledků (např. ukazatelů).}} 
\end{itemize}
%TODO jako proc to nedela to key jako referenci???
V~této práci je používán druhý způsob komparace, kdy jsou srovnávány cíle a~vzdělávací oblasti v~kurikulárních dokumentech sledovaných zemí.  

V~neposlední řadě byla provedena metoda pozorování, která spadá mezi metody kvalitativní. Vzhledem k podmínkám praxe ve Francii, kde nebylo dovoleno zapojovat se do dění vzdělávácího procesu, jedná se o~pozorování nestrukturované, při kterém se podle \citet[s.~17]{Gavora96}: \textit{„nepoužívají předem stanovené pozorovací systémy, škály anebo jiné přesné nástroje. Určeny jsou jen konkrétní události, jevy a~osoby, které se mají pozorovat.“} 

Konkrétními událostmi při tomto pozorování byl časový harmonogram a~program dne dětí a~zázemí jedné třídy mateřské školy ve Francii a~jedné třídy v České republice, o~nichž byly dělány podrobné písemné záznamy, neboli vzorky událostí (angl. specimen records). 
