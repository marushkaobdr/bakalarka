
\chapter{METODY}

Ve své práci jsem využila dvou výzkumných metod, a to obsahové analýzy a pozorování. Nejprve byla provedena analýza kurikurálních dokumentů a vzdělávacích oblastí předškolního vzdělávání Francie a České republiky a poté zpracováno pozorování z průběžné praxe, jak ve Francii, tak v České republice.

\textit{„Obsahová analýza je dôležitým nástrojom poznania jednotlivých oblastí výchovy a vzdelávania.“ }(Gavora, 2008). Jedná se o velmi mladou výzkumnou metodu ze 40.let 20.století, která byla původně využívána v masmédiích a postupně si nacházela své místo i v humanitních oborech a v neposlední řadě i v pedagogice. Lze ji uskutečňovat nekvantitativním nebo kvantitativním způsobem. V mém případě se jedná o první způsob, kdy nejde o převedení kvalitativní parametrů (pojmy, slova, témata) na kvantitativní míru či numerickou hodnotu, ale o popis obsahu kurikulárních dokumentů a jeho následné srovnání. 

Druhou výzkumnou metodou bylo pozorování, které spadá mezi metody kvalitativní. Vzhledem k podmínkám praxe, kterou jsem absolvovala ve Francii, kde nebylo dovoleno zapojovat se do dění, jedná se o pozorování nestrukturované, při kterém se podle Gavory (1996,str.17): \textit{„nepoužívají předem stanovené pozorovací systémy, škály anebo jiné přesné nástroje. Určeny jsou jen konkrétní události, jevy a osoby, které se mají pozorovat.“} 

Konkrétními událostmi při tomto pozorování byl časový harmonogram a program dne dětí a zázemí jedné třídy mateřské školy ve Francii a jedné třídy v České republice, o nichž jsem si dělala podrobné písemné záznamy, neboli vzorky událostí (angl. specimen records). 
