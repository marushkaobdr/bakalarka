
\chapter{METODY}

Ve své práci jsem využila tří výzkumných metod, a to obsahové analýzy, komparace a pozorování. Obsahová analýza se týká kurikurálních dokumentů a cílů a vzdělávacích oblastí předškolního vzdělávání Francie a České republiky, které se věnuji v kapitole \ref{kurikulum}. Pozorování jsem prováděla během průběžné praxe, jak ve Francii, tak v České republice a věnuji se mu v kapitole \ref{rezim}. Metodu komparace jsem použila u obou zmiňovaných kapitol. 

Obsahová analýza je důležitým nástrojem poznání jednotlivých oblastí výchovy a vzdělávání. Jedná se o velmi mladou výzkumnou metodu ze 40.let 20.století, která byla původně využívána v masmédiích a postupně si nacházela své místo i v humanitních oborech a v neposlední řadě i v pedagogice. Lze ji uskutečňovat nekvantitativním nebo kvantitativním způsobem. V mém případě se jedná o první nekvantitativní způsob, kdy nejde o převedení kvalitativní parametrů (pojmy, slova, témata) na kvantitativní míru či numerickou hodnotu, ale o popis a rozbor obsahu dokumentů, tj. kurikulárních dokumentů a jeho následné srovnání \citep{Gavora08}.

Komparace (srovnání) je velmi používaná vědecká metoda. Umožňuje stanovit shody a rozdíly jevů či objektů. Při komparaci se zjišťují shodné či rozdílné znaky různých předmětů, jevů nebo ukazatelů. 
Komparaci je možné rozdělit na dva způsoby:
\begin{itemize}
\item []\textit{Srovnávání pojetí problémů, názorů, premis jako vytváření, ověřování či zdůvodňování vlastního stanoviska (postupu, úvah);}
\item []\textit{Srovnávání jako nástroj měření, zjišťování, objektivizace a hodnocení dosažených výsledků (např. ukazatelů).} \citep[s.~19]{Siroky}
\end{itemize}
%TODO jako proc to nedela to key jako referenci???
V této práci jde o druhý způsob komparace, kdy srovnávám cíle a vzdělávací oblasti v kurikulárních dokumentech sledovaných zemí.  

V neposlední řadě byla provedena metoda pozorování, která spadá mezi metody kvalitativní. Vzhledem k podmínkám praxe, kterou jsem absolvovala ve Francii, kde nebylo dovoleno zapojovat se do dění, jedná se o pozorování nestrukturované, při kterém se podle \citet[s.~17]{Gavora96}: \textit{„nepoužívají předem stanovené pozorovací systémy, škály anebo jiné přesné nástroje. Určeny jsou jen konkrétní události, jevy a osoby, které se mají pozorovat.“} 

Konkrétními událostmi při tomto pozorování byl časový harmonogram a program dne dětí a zázemí jedné třídy mateřské školy ve Francii a jedné třídy v České republice, o nichž jsem si dělala podrobné písemné záznamy, neboli vzorky událostí (angl. specimen records). 
