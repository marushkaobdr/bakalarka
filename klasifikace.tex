	
\chapter{ZÁKLADNÍ INFORMACE O MATEŘSKÝCH ŠKOLÁCH SROVNÁVANÝCH ZEMÍ}
%TODO JA; nejaky kecy tady

	\section{Zařazení mateřské školy v rámci klasifikace vzdělávacího systému}

		V roce 1976 vydalo UNESCO Mezinárodní standardní klasifikaci vzdělávání ISCED (International Standard Classification of Education), která slouží ke shromažďování vzdělávacích statistik jednotlivých zemí i v mezinárodním měřítku. Klasifikace kmenových oborů vzdělávání z roku 1997 má 7 úrovní vzdělávání (0 až 6).
		Pro účely této práce je důležité si představit první dvě úrovně:

\begin{itemize}
\item [] \textbf{ISCED 0} - Vzdělávání v raném dětství (preprimární vzdělávání,mateřské školy) - programy na této úrovni mají podporovat poznávací, fyzický, sociální a emocionální rozvoj malých dětí, uvádět je do organizované výuky mimo kontext rodiny a rozvíjet jejich emocionální dovednosti nezbytné pro školní docházku a zapojení do společnosti. 
\item [] \textbf{ISCED 1} - Primární vzdělávání (základní vzdělání, základní školy včetně speciálních - 1.stupeň, zvláštní školy - 1. a 2. stupeň, pomocné školy - nižší, střední a vyšší stupeň a rehabilitační třídy) - programy na této úrovni mají žákům poskytovat základní dovednosti ve psaní, čtení a počítání a vytvářet pevný základ pro učení a porozumění jádru vědění, pro osobní a sociální rozvoj v rámci přípravy na nižší sekundarní vzdělávání. \citep{ISCED}
\end{itemize}

		Preprimární vzdělávání je méně používaný termín, který odpovídá českému termínu předškolní vzdělávání. Jde o úroveň ISCED 0, která je vymezena jako uvedení dětí raného věku do prostředí školního typu, realizovaná typicky, ne však výhradně, v mateřských školách.

		%TOTO JA stranky najit
		Podle \citet{KeyData} se v mateřských školách v evropských zemích uplatňují různé modely:
		\begin{enumerate}[1)]
		\item Školský model (school model) – preprimární vzdělávání je organizované ve třídách, v nichž jsou zařazeny děti podle věkových kategorií, tedy podobně jako ve skutečné škole. 
		\item Rodinný model (family model) – preprimární vzdělávání je organizováno ve skupinách sdružujících děti různého věku, tedy podobně jako ve skutečných rodinách. 
		\item Oba modely
		\end{enumerate}

		Francie se řadí mezi země se školským modelem, \citet{Prucha99} sem řadí i Českou republiku, dnešní trendy již ovšem postoupily a v České republice najdeme i modely rodinné.

		Z hlediska komparace je oblast předškolního vzdělávání velmi variabilní. Každá země má jiné způsoby realizace předškolní výchovy pomocí vzdělávacích institucí a jiné obsahy a cíle předškolního vzdělávání. 

		Dále tedy uvádím základní informace o organizaci a vedení mateřských škol srovnávaných zemí, tj. Francie a České republiky.

	\section{Mateřské školy ve Francii}
%TODO obrazek
		Mateřské školy ve Francii jsou státní instituce zajišťující preprimární vzdělávání. Dlouholetá tradice nahlíží na předškolní vzdělávání (école maternelle) jako na počáteční formu vzdělávání, na níž navazuje primární vzdělávání (école élémentaire). Jde o návaznost ISCED  úrovně 0 a 1. Mateřská škola poskytuje péči dětem od 2 do 6 let, je však součástí základního vzdělávání poskytující vzdělávání pro děti od 2 do 11 let.

		Vzdělání v mateřských školách odpovídá tzv. prvnímu učebnímu cyklu (cycle des apprentissages premier), rozdělenému do tří stupňů podle věku žáků: nižší stupeň (petite section) pro děti dvou až čtyřleté; střední stupeň (moyenne section) pro děti čtyř až pětileté; vyšší stupeň (grande section) pro děti pěti až šestileté.
		(Průcha, 2012). 

		Primární vzdělávání má dva cykly, prvním je mateřská škola, Grande section je již přechodem do druhého cyklu, kde jsou 2 přípravné třídy (Cours préparatoire CP1 a CP2), na ty navazují 2 třídy základního vzdělávání (Cours élémentaire CE1, CE2), a je zakončen středními třídami (Cours moyenne CM1 a CM2), poté děti přecházejí na sekundární vzdělávání na collège, které odpovídá našemu druhému stupni základních škol. 

		V roce 1886 byl vydán zákon, podle kterého jsou mateřské školy veřejné, bezplatné a sekulární instituce, a který vymezuje jejich vzdělávací funkce. 

		Školství ve Francii je od svých počátků centralizované. Od roku 1982 začala jeho decentralizace, která přerozdělila pravomoc státní administrativy a lokálních samospráv. Stát zůstává garantem vzdělávání jako veřejné služby a definuje rámec vzdělávání a kurikula. Mateřské školy jsou pod pravomocí Ministerstva školství (Ministère de l´éducation national)
%TODO odkaz
%(http://www.clovekvtisni.cz/uploads/file/1360764270-an_KA3_komparace.pdf)

	\section{Mateřské školy v České republice}
%TODO obrazek
		Mateřská škola v České republice je instituce zajišťující předškolní vzdělávání pro děti od 3 do 6 let (do 7 let v případě odkladu školní docházky), které se školským zákonem stalo legitimní součástí systému vzdělávání. Podle mezinárodní klasifikace se jedná o ISCED 0. Jedná se o organizované vzdělávání, které musí splňovat požadavky MŠMT (Ministerstva školství, mládeže a tělovýchovy). Předškolní vzdělávání v mateřské škole je veřejnou, nepovinnou a bezplatnou službou pro všechny děti. Přednostně jsou přijímány děti v posledním roce před začátkem povinné školní docházky. 

		Organizačně se mateřská škola dělí na třídy, které je možné vytvářet podle věku, a to na třídy věkově homogenní a na třídy věkově heterogenní. Do mateřských škol je možné zařazovat i děti se specifickými potřebami a vytvářet tak třídy integrované. 

		Ve veřejné sféře je zřizovatelem mateřské školy většinou obec nebo svazek obcí. V České republice existují i soukromé mateřské školy.

		Předškolní vzdělání v mateřské škole má 3 ročníky:
		
\begin{itemize}
		\item [] \textit{\uv{V prvním ročníku mateřské školy se vzdělávají děti, které v příslušném školním roce dovrší nejvýše 4 roky věku.}}
		\item [] \textit{\uv{V druhém ročníku mateřské školy se vzdělávají děti, které v příslušném školním roce dovrší nejvýše 5 let věku.}}
		\item [] \textit {\uv{Ve třetím ročníku mateřské školy se vzdělávají děti, které v příslušném školním roce dovrší 6 let věku a děti, kterým byl povolen odklad povinné školní docházky.}}
\end{itemize}
\citep[s.~71]{Organizace}