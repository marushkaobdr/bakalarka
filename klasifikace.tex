
\chapter{ZÁKLADNÍ INFORMACE O~MATEŘSKÝCH ŠKOLÁCH SROVNÁVANÝCH ZEMÍ}
%TODO JA; nejaky kecy tady

	\section{Zařazení mateřské školy v~rámci klasifikace vzdělávacího systému}

		V roce 1976 vydalo UNESCO Mezinárodní standardní klasifikaci vzdělávání ISCED (International Standard Classification of Education, dále jen ISCED), která slouží \textit{\uv{jako nástroj vhodný pro shromažďování, zpracování a zpřístupňování vzdělávacích statistik jak v~jednotlivých zemích, tak v~mezinárodním měřítku}}~\citep{ISCED2}.

\noindent
		Klasifikace kmenových oborů vzdělávání z roku 1997 má 7 úrovní vzdělávání (0~až~6).
		Pro účely této práce je důležité si představit první dvě úrovně:

\begin{itemize}
	\setlength\itemsep{-2mm}
	\item [] \textbf{ISCED 0} - Vzdělávání v~raném dětství (preprimární vzdělávání, mateřské školy). Programy na této úrovni mají podporovat poznávací, fyzický, sociální a emocionální rozvoj malých dětí, uvádět je do organizované výuky mimo kontext rodiny a rozvíjet jejich emocionální dovednosti nezbytné pro školní docházku a zapojení do společnosti. 
	\item [] \textbf{ISCED 1} - Primární vzdělávání (základní vzdělání, základní školy včetně speciálních - 1. stupeň, zvláštní školy - 1. a 2. stupeň, pomocné školy - nižší, střední a vyšší stupeň a rehabilitační třídy). Programy na této úrovni mají žákům poskytovat základní dovednosti v psaní, čtení a počítání a vytvářet pevný základ pro učení a porozumění jádru vědění, pro osobní a sociální rozvoj v~rámci přípravy na nižší sekundarní vzdělávání~\citep{ISCED}.
\end{itemize}

		Preprimární vzdělávání neboli také předškolní vzdělávání spadá do úrovně ISCED 0. Jedná se o~nepovinné vzdělávání, které uvádí děti raného věku do prostředí institucionálního zařízení. 

		Podle~\textit{Key data on education in the European Union 97} se v mateřských školách v evropských zemích uplatňují různé modely \citep[s.~57]{Prucha99}:
		\textit{\begin{enumerate}[1)]
			\setlength\itemsep{-2mm}
			\item \uv{Školský model (school model) – preprimární vzdělávání je organizované ve třídách, v nichž jsou zařazeny děti podle věkových kategorií, tedy podobně jako ve skutečné škole. 
			\item Rodinný model (family model) – preprimární vzdělávání je organizováno ve skupinách sdružujících děti různého věku, tedy podobně jako ve skutečných rodinách. 
			\item Oba modely}
		\end{enumerate}}

		Tyto modely jsou však odlišné i svými vždělávacími cíli. Školský model připravuje děti na vstup do základní školy, kdežto rodinný typ se věnuje spíše rozvoji sociálních dovedností a uvedení dětí do společnosti.

		Realizace předškolního vzdělávání se liší stát od státu. Různé jsou jak cíle tak obsah vzdělávání. Pro porozumění je tedy v~dalších kapitolách uvedena pozice mateřských škol ve vzdělávacím systému obou sledovaných zemí. 
		

	\section{Mateřské školy ve Francii}
	\label{msvefr}

		Mateřské školy (dále též zkráceně MŠ) ve Francii jsou státní instituce zajišťující preprimární vzdělávání. 
		Dlouholetá tradice nahlíží na předškolní vzdělávání (école maternelle) jako na počáteční formu vzdělávání, na níž navazuje primární vzdělávání (école élémentaire). Jde o návaznost ISCED  úrovně 0 a 1. Mateřská škola poskytuje péči dětem od 2 do 6 let, je však součástí základního vzdělávání poskytující vzdělávání pro děti od 2 do 11 let.

		Primární vzdělávání ve Francii se odehrává ve třech cyklech (viz tabulka \ref{tab:rozdeleniTridFR}). Prvním cyklem (cycle des apprentissages premiers) je mateřská škola, poslední třída mateřské školy (grande section) je již přechodem do druhého cyklu (cysle des apprentisages fondamentaux), jehož součástí je přípravná třída (cours préparatoire CP), na kterou navazuje první třída základního vzdělávání (cours élémentaire CE1). Ve třetím cyklu (cycle des approfondissements) je druhá třída základního vzdělávání (cour élémentaire CE2) a dvě střední třídy (Cours moyenne CM1 a CM2). Poté děti přecházejí na sekundární vzdělávání na collège, které odpovídá našemu druhému stupni základních škol. 

		\begin{figure} [h!]
			\center
			\includegraphics[width=1.0\linewidth]{fotky/msFR.png} \\
			\includegraphics[width=1.0\linewidth]{fotky/msVysvetlivky.png}
			\caption{ \textbf{Zařazení mateřské školy ve vzdělávacím systému Francie}
			(Převzato autorkou z: Eurydice 2014)
			}
			\label{obr:msFR}
		\end{figure}

		Vzdělání v mateřských školách odpovídá tzv. prvnímu učebnímu cyklu (cycle des apprentissages premiers) rozdělenému do tří stupňů podle věku žáků: nižší stupeň (petite section) pro děti tří až čtyřleté; střední stupeň (moyenne section) pro děti čtyř až pětileté; vyšší stupeň (grande section) pro děti pěti až šestileté
		(Průcha, 2012). 
% TODO JA: reference prucha
		Je-li v~mateřské škole dostatek místa, jsou přijímany děti již od 2 let do tzv. toute petite section. 

\begin{spacing}{1.0}
\begin{table}[h]
	\center
	\small
	\begin{center}
	\begin{tabular}{|c|ccc|}
		\hline
		\rowcolor{grey}
		\textbf{Cyklus}				& \textbf{Třída}		& \textbf{Věk}	& \textbf{Kde se odehrává}	\\
		\hline
		\hline
		\rowcolor{grey!10}
		%==================================================================================================
	\cellcolor{white} cycle des apprentissages	& toute petite section 	& 2-3 		&				\\ \rowcolor{grey!20}
	\cellcolor{white} premiers (1. cyklus)		& petite section 		& 3-4 		& jen v~MŠ 		\\ \rowcolor{grey!20}
	\cellcolor{white}							& moyenne section 		& 4-5 		& 				\\ \rowcolor{grey!20}
		\hline
		%==================================================================================================
	\cellcolor{white} cycle des apprentissages 	& grande section 		& 5-6 		& začíná v~MŠ, 		\\ \rowcolor{grey!50}
	\cellcolor{white} fondametaux (2.cyklus) 		& CP 					& 6-7 		& pokračuje na ZŠ 	\\ \rowcolor{grey!50}
	\cellcolor{white}								& CE1 					& 7-8 		& 					\\ \rowcolor{grey!50}
		\hline
		%================================================================================================+=
	\cellcolor{white} cycle des approfonissements & CE2 					& 8-9 		&					\\ \rowcolor{grey!50}
	\cellcolor{white} (3.cyklus)					& CM1 					& 9-10 		& jen v~ZŠ 			\\ \rowcolor{grey!50}
	\cellcolor{white}								& CM2 					& 10-11 	& 					\\ \rowcolor{grey!50}
		\hline
	\end{tabular}
	\end{center}
	\caption{ \textbf{Cykly primárního vzdělávání ve Francii.} Tabulka znázorňuje věkové rozdělení do tříd a učebních cyklů a barevně je označen přechod mezi mateřskou školou a školou základní. 
	}
	\label{tab:rozdeleniTridFR}
\end{table}
\end{spacing}
		Školství ve Francii bylo od svých počátků centralizované. Od roku 1982 začala jeho decentralizace, která přerozdělila pravomoc státní administrativy a lokálních samospráv. Stát zůstává garantem vzdělávání jako veřejné služby a definuje rámec vzdělávání a kurikula. Mateřské školy jsou pod pravomocí Ministerstva školství (Ministère de l´éducation national).


		V roce 1886 byl vydán zákon, podle kterého jsou mateřské školy veřejné, bezplatné a laické instituce, a který vymezuje jejich vzdělávací funkce.

	\section{Mateřské školy v~České republice}


		Mateřská škola v České republice je instituce zajišťující předškolní vzdělávání pro děti od 3 do 6 let (do 7 let v případě odkladu školní docházky), které se školským zákonem stalo legitimní součástí systému vzdělávání. Podle mezinárodní klasifikace se jedná o ISCED 0. MŠ zajišťuje organizované vzdělávání, které musí splňovat požadavky MŠMT (Ministerstva školství, mládeže a tělovýchovy). Předškolní vzdělávání v~mateřské škole je veřejnou, nepovinnou a bezplatnou službou pro všechny děti. Přednostně jsou přijímány děti v posledním roce před začátkem povinné školní docházky. 
		
		\noindent
		Ve veřejné sféře je zřizovatelem mateřské školy většinou obec nebo svazek obcí. V České republice existují i soukromé mateřské školy.
		
		\begin{figure} [t]
			\center
			\includegraphics[width=1.0\linewidth]{fotky/msCR.png} \\
			\includegraphics[width=1.0\linewidth]{fotky/msVysvetlivky.png}
			\caption{ \textbf{Zařazení mateřské školy ve vzdělávacím systému České republiky}
			(Převzato autorkou z: Eurydice 2014)
			}
			\label{obr:msCR}
		\end{figure}

\begin{spacing}{1.0}
\begin{table}[h!]
	\center
	\small
	\begin{center}
	\begin{tabular}{|c|c|c|c|}
		\hline
		\rowcolor{grey}
		\textbf{Typ skupiny} & \textbf{Ročník} & \textbf{Věk}	\\
		\hline
		\hline
		\rowcolor{grey!10}
		%==================================================================================================
		homogenní	& 1.ročník 	& 3-4 		\\ \rowcolor{grey!10}
		skupina		& 2.ročník 	& 4-5 		\\ \rowcolor{grey!10}
					& 3.ročník 	& 5-6/7		\\ \rowcolor{grey!10}
		\hline
		%==================================================================================================
		heterogenní & 1.-3.ročník 	& 3-6/7 	\\ \rowcolor{grey!10}
		skupina 	&				&			\\ \rowcolor{grey!10}
		\hline
	\end{tabular}
	\end{center}
	\caption{ \textbf{Rozdělení tříd podle věku v~České republice}
	}
	\label{tab:rozdeleniTridCR}
\end{table}
\end{spacing}
	\noindent
		Organizačně se mateřská škola dělí na třídy, které je možné vytvářet podle věku, a to na třídy věkově homogenní a na třídy věkově heterogenní (viz Tabulka~\ref{tab:rozdeleniTridCR}). Do mateřských škol je možné zařazovat i děti se specifickými potřebami a vytvářet tak třídy integrované. 		
		
		\break\noindent
		Předškolní vzdělání v mateřské škole se dělí na 3 ročníky:
	\vspace{-2mm}
	\textit{\begin{itemize}
		\setlength\itemsep{-2mm}
		\item [] \uv{V prvním ročníku mateřské školy se vzdělávají děti, které v příslušném školním roce dovrší nejvýše 4 roky věku.
		\item [] V druhém ročníku mateřské školy se vzdělávají děti, které v příslušném školním roce dovrší nejvýše 5 let věku.
		\item [] Ve třetím ročníku mateřské školy se vzdělávají děti, které v příslušném školním roce dovrší 6 let věku a děti, kterým byl povolen odklad povinné školní docházky.} \citep[s.~71]{Organizace}
	\end{itemize}}



