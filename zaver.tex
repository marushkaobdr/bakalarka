\chapter*{Závěr}
\addcontentsline{toc}{chapter}{Závěr} 
Hlavním tématem této práce byla komparace vybraných aspektů současné francouzské mateřské školy a~české mateřské školy. V~teoretické části práce byly zpracovány základní informace o~zařazení mateřských škol do vzdělávacího systému. Byly popsány podmínky péče o~předškolní děti a~přiblížili jsme čtenáři rozdílné pojetí dítěte, které se odráží i~ve vybraných srovnávaných aspektech, které byly sledovány v~části praktické. Cílem komparace bylo zjistit rozdílnost v~cílech, pojetí a~obsahu vzdělávání, stejně jako přiblížit a~porovnat časový harmonogram mateřských škol obou zemí, a~podmínky, ve kterých se předškolní vzdělávání odehrává.  

Ke komparaci byly vybrány dva aspekty: kurikulum a~režim dne. Informace o~kurikulu francouzské mateřské školy nebyly doposud českému čtenáři k~dispozici v~českém překladu. Autorka využila svých jazykových i~studiem získaných odborných znalostí k~autorskému překladu vzdělávacích oblastí legislativního dokumentu Francie. Pro potřeby této práce se nejedná o~překlad úplný, ale o~stručnější verzi nejdůležitějších sledovaných bodů. 

Autorce se jeví, že pojetí dítěte se výrazně projevuje ve všech sledovaných a~popsaných aspektech, ať již v~rozdílných ekonomických podmínkách a~výrazně rozdílné délky mateřské dovolené, tak zejména v~rozdílném uchopení cílů vzdělávání, uvedených v~legislativních dokumentech, tedy i~v~autorkou sledovaném kurikulu a~režimu dne. 

Jak je blíže popsáno ve druhé kapitole \uv{Pojetí dítěte ve Francii a~České republice}, rozdílnost v~pojetí dítěte vidí autorka zejména v~tom, že ve Francii se dívají na dítě v~mateřské škole jako na žáka, zatímco v~České republice se na něj díváme jako na dítě.   

Tyto rozdíly jsou zřetelně viditelné v~kurikulu. Hlavním vzdělávacím cílem ve francouzském dokumentu je připravit dítě na úspěšné zvládnutí role žáka v~primární škole. Dokladem toho je, že součástí francouzského kurikula je vzdělávací oblast s~názvem \uv{Stát se žákem}. Na konci mateřské školy ve Francii by děti měly umět psát velkými tiskacími písmeny. V~rámci nácviku kopírují slova z~předloh. Velký důraz je kladen i~na rozvoj jazykových dovedností dětí. Zatímco v~České republice se jedná o~přípravu na psaní, tj. uvolňování ruky, držení pera, tužky, rozlišování tvarů a~forem. V~České republice žádný cíl, který by odpovídal francouzskému pojetí \uv{stát se žákem}, není. V~České republice je dáván velký důraz na dětské prožívání, vlastní kreativitu a~socializaci, které vyplývají z~cílů RVP~PV. Dětem je dáván dostatek volného prostoru a~jsou vytvářeny i~materiální a~prostorové podmínky pro volnou hru. Hra je nedílnou součástí vzdělávacího programu v~české mateřské škole. 

Pojetí dítěte se projevuje i~v~časovém harmonogramu. Francouzský časový harmonogram kopíruje harmonogram školy primární (viz příloha tab. \ref{tabulkaMS} a~\ref{tabulkaMS2}). Děti předškolního věku tedy tráví v~mateřské škole podobnou dobu jako děti školního věku na primární škole. Rozložení aktivit během dne a~jejich délka i~obsah tomu také napovídají. V~České republice je časový harmonogram též závazný, ale jeho dodržování je mnohem flexibilnější a~dětem je dáván dostatek prostoru na vlastní prožívání, zkoumání a~experimentování. Kolik času denně  děti stráví v~mateřské škole, je v~rukou rodičů, na nich záleží, jestli budou v~mateřské škole celý den nebo třeba jen půlden. To může být důvodem, proč je hlavní část řízených aktivit zařazena v~dopoledních hodinách, tedy tak, aby se jich mohly účastnit všechny děti. 

Tyto závěry potvrzují hypotézu č.1 \ref{cile}, že kurikulum předškolní výchovy obou zemí má rozdílné cíle a~pojetí předškolního vzdělávání. Zjednodušeně řečno, ve Francii připravuje předškolní vzdělávání dítě na vstup do světa školy, kdežto v~České republice na proživotní situace. Tyto závěry současně vyvracejí hypotézu č.2 \ref{cile}, že v~obou zemích mají stejný přístup k~dítěti. Ve Francii nahlíží na dítě jako na žáka a~v~České republice jako na dítě. 

Možnou inspirací pro český vzdělávací systém by mohlo být francouzské rozdělení vzdělávání do cyklů. Poslední třída mateřské školy je zároveň první třídou druhého vzdělávacího cyklu (viz kapitola \ref{msvefr}). Toto rozdělení usnadňuje plynulý přechod z~mateřské školy na školu základní. 

Česká republika by mohla být Francii inspirací tím, že ve vzdělávacím systému respektuje neopakovatelné období dětsví. Hlavní součástí péče o~toto období je jeho zakotvení v~kurikulárním dokumentu. Jak již bylo zmíněno, české děti mají více prostoru k~volné hře, hře jako takové, vlastnímu prožívání a~experimentování, navazování a~rozvíjení sociálních vazeb. V~kurikulu jsou učitelům také dávány návody, jakými vhodnými způsoby mohou tyto oblasti rozvíjet. V~českých mateřských školách je i~prostředí a~materiálové vybavení více přizpůsobeno k~tomu, aby si děti mohly hrát.

Z~pohledu budoucí učitelky v~mateřské škole jsem si díky praxi uvědomila, že to, co jsem považovala za běžné a~normální v~českých mateřských školách, nemusí být samozřejmostí. Jsem si vědoma i~toho, že můj osobní pohled při pozorování a~srovnávání mohl snižit mou objektivitu. Tato zkušenost mě však utvrdila v~mém přesvědčení, že vzdělávání v~mateřských školách v~České republice má celou řadu pozitiv jak pro děti, tak pro profesi učitele. 