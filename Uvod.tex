\setcounter{page}{1}
\pagenumbering{arabic}
\chapter*{Úvod}
\addcontentsline{toc}{chapter}{Úvod} 
Dnešní doba 21. století je otevřena cestování a možnostem čerpat zkušenosti za hranicemi České republiky. V rámci studia na vysoké škole je možné účastnit se programů podporovaných Evropskou unií. Jedním z nich je studentský program Erasmus, díky kterému autorka studovala jeden semestr na partnerské univerzitě ve Francii. Ke studiu využila jazykových kompetencí nabytých v předešlém studiu (Jazyky pro cestovní ruch - angličtina, francouzština). V rámci studia ve Francii byla povinná dvoutýdenní stáž v mateřské škole. Autorka poté ve Francii více než rok pracovala jako chůva u několika francouzských rodin. Tyto zkušenosti ji vedly k volbě tématu bakalářské práce. 

Cílem bakalářské práce je komparace současné francouzské a české mateřské školy ve vybraných aspektech. Vybranými aspekty srovnávání jsou cíle a kurikulum předškolního vzdělávání a režim dne v mateřské škole obou srovnávaných zemí. Výběr právě těchto aspektů vychází z rozdílnosti harmonogramu dne mateřských škol, které byly zažity v praxi. Během pozorování a všímání si odlišností, vyvstaly autorce práce otázky, jestli mají obě země podobný nebo odlišný obsah vzdělávání a jakým způsobem nahlížejí na dítě. 

Teoretická část přináší obecné informace týkající se mateřských škol obou zemí, ekonomických podmínek péče o dítě předškolního věku a pohledu na dítě. První kapitola zařazuje mateřské školy v rámci mezinárodní klasifikace vzdělávacího systému. Najdeme zde hlavní principy mateřských škol a organizaci tříd a vzdělávání v mateřských školách. Druhá kapitola přibližuje rozdílnosti v pohledu na dítě obou zemí, jejichž pravdivost je dokládána jak v teoretické, tak praktické části. Ve třetí kapitole jsou popsány možnosti péče o dítě předškolního věku, jejich podmínky a ekonomické faktory. Čtvrtá kapitola obsahuje detailní analýzu legislativních dokumentů týkajících se předškolního vzdělávání sledovaných zemí a jejich srovnání, včetně autorského překladu vybraných částí originálního francouzského dokumentu.

V praktické části si dala autorka za cíl přiblížit a porovnat časové harmonogramy vybraných tříd mateřské školy v obou zemích a podmínky, v kterých se předškolní vzdělávání odehrává. Dává nahlídnout do běžného dne v mateřské škole ve Francii i České republice, z kterého lze vyčíst, jak je v daných zemích kurikulum v praxi realizováno.  