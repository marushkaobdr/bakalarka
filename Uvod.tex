\chapter{Úvod}

Dnešní doba 21. století je otevřena cestování a možnostem čerpat zkušenosti za hranicemi České republiky. V rámci studia na vysoké škole je možné se účastnit programů podporovaných Evropskou unií. Jedním z nich je studentský program Erasmus, díky kterému autorka studovala jeden semestr na partnerské univerzitě ve Francii. Ke studiu využila jazykových kompetencích nabytých v předešlém studiu (Jazyky pro cestovní ruch-angličntina, francouzština). V rámci studia ve Francii byla povinná dvoutýdenní stáž v mateřské škole. Autorka poté ve Francii více jak rok pracovala jako chůva u několika francouzkých rodin. Tyto zkušenosti vedly k volbě tématu této bakalářské práce. 

Teoretická část přináší obecné informace týkající se mateřských škol obou zemí a pohledu na dítě, stejně jako ekonomické podmínky péče o dítě předškolního věku. První kapitola zařazuje mateřské školy v rámci mezinárodní klasifikace vzdělavacího systému. Najdeme zde hlavní principy mateřských škol a organizaci tříd a vzdělavaní v mateřských školách. Druhá kapitola přibližuje rozdílnosti v pohledu na dítě obou zemí, jejichž pravdivost dokládá část praktická. Ve třetí kapitole jsou popsány možnosti péče o dítě předškolního věku, jejich podmínky a ekonomický faktory. 

Cílem výzkumného projektu bakalářské práce je komparace současné francouzské a české mateřské školy. Vybranými aspekty jsou cíle a kurikulum předškolního vzdělávání a režim dne v mateřské škole obou srovnávaných zemí. Cílem je zjistit rozdílnosti v cílech, pojetí a obsahu vzdělávání, stejně jako přiblížit a porovnat časový harmonogram mateřských škol obou zemí a podmínky, v kterých se předškolní vzdělávání odehrává. 

% zaver???