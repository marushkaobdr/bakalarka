\chapter{PÉČE O DÍTĚ PŘEŠKOLNÍHO VĚKU SROVNÁVANÝCH ZEMÍ}

	Přístup k nejmenším dětem je ovlivněn mnoha faktory. Jinak na dítě pohlíží nejbližší rodina a jinak ho vidí společnost. Možnost péče o předškolní děti je z velké části ovlivněna ekonomickými podmínkami rodiny a sociální podporou státu. Nerodinná a institucionální péče začíná tam, kde končí možnosti celodenní rodinné péče. Tento faktor je ovlivněn podmínkami mateřské a rodičovské dovolené a možnostmi další péče o děti.

	V této kapitole pracuji s podrobnou studii od \citet{DenniPece}, která vznikla v rámci projektu Nové formy denní péče o děti v České republice. Studie detailně popisuje systémy denní péče o děti, rodičovské dovolené, možnosti individuální nerodinné péče a nabídku a poptávku v obou zemích. Dále vycházím z vlastních dvouletých zkušeností chůvy a pozorování dění ve Francii.

		\section{Podmínky péče o předškolní děti ve Francii}
		Péče o děti ve Francii má dva pilíře finanční podpory – finanční podpora vyplácena přímo rodičům a finanční podpora vyplácena poskytovatelům služeb nerodinné péče. 

			\subsection{Mateřská dovolená}
				Od roku 1970 je ve Francii zavedena mateřská dovolená pro všechny zaměstnance, která je placená ze sociálního pojištění a činí 90\% hrubé mzdy. Minimální délka mateřské dovolené je 16 týdnů, tedy 6 týdnu před porodem a 10 týdnů po porodu, tato doba se mění v závislosti na době porodu, zdravotních komplikacích a počtu dětí (3 a více dětí až 26 týdnů). Minimálně je žena povinna vyčerpat 8 týdnů mateřské dovolené. Příspěvek je vyplácen, jestliže žena platila po dobu 10 měsíců pojištění a pracovala alespoň 200 hodin poslední 3 měsíce před nástupem na mateřskou dovolenou.

			\subsection{Rodičovská dovolená}
				Rodičovská dovolená byla zavedena v roce 1977. Umožňuje matkám (resp. otcům) přerušit zaměstnání po narození dítěte při zajištění možnosti návratu k práci u svého zaměstnavatele po jejím ukončení. Rodičovská dovolená trvá 6 měsíců a je možné čerpat do tří let věku dítěte. Možností je její opakované prodlužování. Rodičovská dovolená je neplacená. Příspěvky se dostávají až od druhého dítěte. 

				V roce 2004 byly všechny příspěvky sjednoceny do dávky k přijetí malého dítěte, v kterém mimo jiné je příspěvek na péči o dítě nerodičovskou osobou nebo též rodičovský příspěvek pro matky jednoho dítěte po dobu 6 měsíců. Příspěvek je možné pobírat po 2 letech přispívání do důchodového systému.

			\subsection{Péče o dítě nerodinnou osobou}
				Ve Francii je dlouhá tradice mateřských asistentek. Tyto asistentky by měly být licencované a dokázat schopnost postarat se o děti a jejich zdravý vývoj. V jeden čas smí mít v péči max. 3 děti. Jedná se o péči o děti do 3 let. Asistentka dochází buď do bydliště rodiny, nebo přijímá děti u sebe doma a je zaměstnancem rodiny, která jí vyplácí mzdu. Rodina dostává na mateřskou asistentku dotace od státu.
				Nutnost mateřských asistentek vyplývá z časného nástupu matek zpět do zaměstnání a nedostatku jeselských zařízení, o které je větší zájem, než jsou kapacitní možnosti spádových jeslí. Mateřské asistentky a dále chůvy doprovází velkou část rodin po celou dobu docházky dětí do jeslí, mateřské školy a někdy i školy základní. Asistentky a chůvy vodí děti do institucí zajišťující péči, ze kterých je také vyzvedávají a starají se o ně do příchodu rodičů.

			\subsection{Statistika návštěvnosti dětí v mateřské škole}
				Přestože se státními dotacemi na mateřskou asistentku a chůvu rodinám vrátí cca 50\% nákladů, zůstává tato služba relativně drahá. Proto většina dětí od 3 do 6 let navštěvuje mateřskou školu. Podle Education at a Glance 2013: OECD Indicators (2013) je účast předškolním vzdělávání 4 až 6letých 100\%.

		\section{Podmínky péče o předškolní děti v České republice}
			Podmínky péče v České republice jsou odlišné od podmínek Francie. Stejnými body jsou mateřské a rodičovské podmínky, jejichž čerpání se ovšem markantně liší.

			\subsection{Mateřská dovolená}
				Mateřská dovolená neboli peněžitá pomoc v mateřství a vyplácí se zaměstnankyní po dobu 28 týdnů (resp. 37 týdnů u více dětí). Podmínky pobírání tohoto příspěvku je účast na zdravotním pojištění a vypočítává se ze mzdy za posledních 12 měsíců.

			\subsection{Rodičovská dovolená}
				Rodičovskou dovolenou mohou pobírat jak matky, tak otcové a žádá se o ní s koncem mateřské dovolené nebo po narození dítěte rodičům, kterým nevznikl na mateřskou dovolenou nárok. Rodičovský příspěvek je sociální dávka, na kterou má nárok každý, kdo se účastnil na nemocenském pojištění. Celková částka činí 220 000 Kč, která se může čerpat od 19 měsíců do 4 let věku dítěte.

			\subsection{Statistika návštěvnosti dětí v mateřské škole}
				Ze stejné statistiky, kterou jsem uvedla v kapitole 4.1.4, se dá vyčíst, že návštěvnost  věku 4-6 let je poněkud nižší než ve Francii, pohybuje se mezi 85\% u 4letých až 96\% u 6letých, přesto je stále relativně vysoká. Tento rozdíl by odpovídal rozdílným podmínkám sociální podpory rodičů. V České republice je možné zůstat doma s každým dítětem až 4 roky, rodiče si tedy s více dětmi přizpůsobují čas dětí strávený v mateřské škole svým potřebám. Když jsou na rodičovské dovolené s jedním dítětem a druhé dítě může docházet třeba jen na půl den. V České republice se stále drží tradice výchovy v rodině a trávení většího času s dětmi oproti Francii. I když v posledních letech je tato tradice zastíněna trendem různých zájmových kroužků a mimoškolních aktivit, které rodičům zajistí péči o dítě, pakliže se již vrátili do práce a dítě jim nemá, kdo hlídat. 
				
				V této kapitole jsem se snažila nastínit obrázek sociálních a ekonomických podmínek rodin, které mají v péči předškolní dítě. Informace zde uvedené odkazují na kapitolu \ref{rezim}, kde se zmiňuji jak o otevírací době mateřských škole, tak o vyzvedávání dětí z nich. 
