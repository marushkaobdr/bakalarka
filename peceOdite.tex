\chapter{PÉČE O DÍTĚ PŘEŠKOLNÍHO VĚKU SROVNÁVANÝCH ZEMÍ}

	Přístup k nejmenším dětem je ovlivněn mnoha faktory. Jinak na dítě pohlíží nejbližší rodina a jinak ho vidí společnost. Možnost péče o předškolní děti je z velké části ovlivněna ekonomickými podmínkami rodiny a sociální podporou státu. Nerodinná a institucionální péče začíná tam, kde končí možnosti celodenní rodinné péče. Tento faktor je ovlivněn podmínkami mateřské a rodičovské dovolené a možnostmi další péče o děti.

	V této kapitole vycházím i z vlastních zkušeností chůvy u třech francouzských rodin. 

		\section{Podmínky péče o předškolní děti ve Francii}
		Péče o děti ve Francii má dva pilíře finanční podpory – finanční podpora vyplácena přímo rodičům a finanční podpora vyplácena poskytovatelům služeb nerodinné péče. 


			\subsection{Mateřská dovolená}
				Od roku 1970 je ve Francii zavedena mateřská dovolená pro všechny zaměstnance, která je placená ze sociálního pojištění a činí 90\% hrubé mzdy. Minimální délka mateřské dovolené je 16 týdnů, tedy 6 týdnu před porodem a 10 týdnů po porodu, tato doba se mění v závislosti na době porodu, zdravotních komplikacích a počtu dětí (3 a více dětí až 26 týdnů). Minimálně je žena povinna vyčerpat 8 týdnů mateřské dovolené. Příspěvek je vyplácen, jestliže žena platila po dobu 10 měsíců pojištění a pracovala alespoň 200 hodin poslední 3 měsíce před nástupem na mateřskou dovolenou. \citep{Dennipece}

			\subsection{Rodičovská dovolená}
				Rodičovská dovolená byla zavedena v roce 1977. Umožňuje matkám (resp. otcům) přerušit zaměstnání po narození dítěte při zajištění možnosti návratu k práci u svého zaměstnavatele po jejím ukončení. Rodičovská dovolená trvá 6 měsíců a je možné čerpat do tří let věku dítěte. Možností je její opakované prodlužování. Rodičovská dovolená je neplacená. Příspěvky se dostávají až od druhého dítěte. 

				V roce 2004 byly všechny příspěvky sjednoceny do dávky k přijetí malého dítěte, v kterém mimo jiné je příspěvek na péči o dítě nerodičovskou osobou nebo též rodičovský příspěvek pro matky jednoho dítěte po dobu 6 měsíců. Příspěvek je možné pobírat po 2 letech přispívání do důchodového systému. \citep{Dennipece}

			\subsection{Péče o dítě nerodinnou osobou}
				Ve Francii je dlouhá tradice mateřských asistentek. Tyto asistentky by měly být licencované a dokázat schopnost postarat se o děti a jejich zdravý vývoj. V jeden čas smí mít v péči max. 3 děti. Jedná se o péči o děti do 3 let. Asistentka dochází buď do bydliště rodiny, nebo přijímá děti u sebe doma a je zaměstnancem rodiny, která jí vyplácí mzdu. Rodina dostává na mateřskou asistentku dotace od státu.
				Nutnost mateřských asistentek vyplývá z časného nástupu matek zpět do zaměstnání a nedostatku jeselských zařízení, o které je větší zájem, než jsou kapacitní možnosti spádových jeslí. Mateřské asistentky a dále chůvy doprovází velkou část rodin po celou dobu docházky dětí do jeslí, mateřské školy a někdy i školy základní. Asistentky a chůvy vodí děti do institucí zajišťující péči, ze kterých je také vyzvedávají a starají se o ně do příchodu rodičů.
		\begin{spacing}{1.0}
		\begin{table}[ht]
			\small
			\begin{center}
			\begin{tabular}{|c|c|c|c|}
				%=========================================================================================
				\hline
				\rowcolor{grey}		
				\textbf{Typy}	&	\textbf{Finance} & 	\textbf{Délka} 	&	\textbf{Opatření} 	\\
				\hline
				%=========================================================================================
				\hline \rowcolor{grey!10}
				Mmateřská	&  placena ze sociálního &  min.16týdnů, 	 & 10 měsíců hrazení 		\\ \rowcolor{grey!10}
				% pokracovani prvniho radku
				dovolená 	& 	 pojištění, činí  	 & 	povinně 8 týdnů, & pojištění, 200 odpra- 	\\ \rowcolor{grey!10}
				% pokracovani prvniho radku 
				 			& 	90\% hrubé mzdy 	 &  max. 26. týdnů 	 & covaných hodin posled- 	\\ \rowcolor{grey!10}
				 			&						 & 					 & ní 3 měsíce před 		\\ \rowcolor{grey!10}
				 			&						 &					 & nástupem na mateřskou 	\\ \rowcolor{grey!10}
				 			&						 & 					 & dovolenou 				\\ \rowcolor{grey!10}
				%=========================================================================================
				\hline
				Rodičovská	& Neplacená & 6 měsíců (do tří let 		& 	zaručen návrat 	\\ \rowcolor{grey!10}
				% pokracovani druheho radku
				dovolená & (příspěvky až od & věku), možné opako-  	&  do zaměstnání	\\ \rowcolor{grey!10}
						 & druhého dítěte)  & vaně prodlužovat						&	\\ \rowcolor{grey!10}
				%=========================================================================================
				 \hline
				Péče o dítě	&	státní dotace vrací	&	dle potřeby	& Max.3 děti \\ \rowcolor{grey!10}
				% pokracovani tretiho radku
				nerodinnou 	&	cca 50\% nákladů 	&	& na 1 mateřskou 	\\ \rowcolor{grey!10}
				osobou 		& 						&	& asistentku		\\ \rowcolor{grey!10}
				%=========================================================================================
				\hline
				Mateřské 	&	bezplatné	& 2 do 6 let	& 100\% účast 4-6letých \\ \rowcolor{grey!10}
				školy 		& 	 			& věku dítěte	& 						\\ \rowcolor{grey!10}
				\hline
			\end{tabular}
			\end{center}
			\label{tab:peceFR}
			\caption{
				\textbf{Podmínky péče o předškolní děti ve Francii.}
				Tabulka shrnuje typy péče o předškolní dítě, financování státem, časovou dotaci a podmínky čerpání dané péče ve Francii.
							}
		\end{table}
		\end{spacing}

			\subsection{Statistika návštěvnosti dětí v mateřské škole}
			\label{statistika}
				Přestože se státními dotacemi na mateřskou asistentku a chůvu rodinám vrátí cca 50\% nákladů, zůstává tato služba relativně drahá. Proto většina dětí od 3 do 6 let navštěvuje mateřskou školu. Podle \cite{Eurydice} je účast předškolním vzdělávání 4 až 6letých 100\%.
			

		\section{Podmínky péče o předškolní děti v České republice}
			Podmínky péče v České republice jsou odlišné od podmínek Francie. Stejnými body jsou mateřské a rodičovské podmínky, jejichž čerpání se ovšem markantně liší.

			\subsection{Mateřská dovolená}
				Mateřská dovolená neboli peněžitá pomoc v mateřství a vyplácí se zaměstnankyní po dobu 28 týdnů (resp. 37 týdnů u více dětí). Podmínky pobírání tohoto příspěvku je účast na nemocenském pojištění a vypočítává se ze mzdy za posledních 12 měsíců. Od června 2014 činí mateřská dovolená 70\% hrubé mzdy. \citep{materska}

			\subsection{Rodičovská dovolená}
				Rodičovskou dovolenou mohou pobírat jak matky, tak otcové a žádá se o ní s koncem mateřské dovolené nebo po narození dítěte rodičům, kterým nevznikl na mateřskou dovolenou nárok. Rodičovský příspěvek je sociální dávka, na kterou má nárok každý, kdo se účastnil na zdravotním pojištění. Celková částka činí 220 000 Kč, tu lze pobírat nejméně 19 měsíců až do 4 let věku dítěte. Rodičovskou dovolenou lze zkracovat či prodlužovat každé tři měsíce do vyčerpání celé částky. \citep{rodicovska}

			\subsection{Péče o dítě nerodinnou osobou}
				V České republice nemají chůvy dlouhou tradici. Starost o děti dříve zastavávali prarodiče. Ti jsou v dnešní době často ještě sami v pracovním svazk,tudíž se o děti starat nemohou. Z tohoto důvodu začíná být ze stran rodičů o chůvy čím dál větší zájem. Tato oblast ovšem není zabezpečena legislativou, tzn. stát na chůvy nijak finančně nepřispívá, stejně jako nejsou dané podmínky na vzdělání chův či pracovní podmínky. 

		\begin{spacing}{1.0}
		\begin{table}[t]
			\small
			\begin{center}
			\begin{tabular}{|c|c|c|c|}
				\hline
				\rowcolor{grey}
				\textbf{typy}	 & \textbf{finance}		& \textbf{délka}		& \textbf{opatření}	 			\\
				\hline
				%============================================================================================
				mateřská & 70\% hrubé mzdy 			& 28, resp.37 týdnů		& nutná účast  					\\
				dovolená & za posledních	 		& 						& na nemocenském 				\\
						 & 12 měsíců				& 						& pojištění						\\
				\hline
				%============================================================================================
				rodičovská 	& celková částka  		& 19 měsíců až 4 roky	& sociální dávka pro 			\\
				dovolená 	& 220 000,- Kč			& 						& každého, kdo se účastní 		\\
							&						&						& zdravotního pojištění 		\\
				\hline
				%============================================================================================
				péče o dítě & není podporováno		& dle potřeby			&   							\\
				nerodinnou	& státem				&						&	-							\\
				osobou 		&						&						&								\\
				\hline
				%============================================================================================
				mateřské	& 	bezplatné 				& Od 3 do 6 (resp.7) let &	85\% účast 4letých 	\\	
				školy 		& 							&  						 & 96\% účast 6letých 	\\
				\hline
			\end{tabular}
			\end{center}
			\label{tab:peceCR}
			\caption{
				\textbf{Podmínky péče o předškolní děti v České republice.}
				Tabulka shrnuje typy péče o předškolní dítě, financování státem, časovou dotaci a podmínky čerpání dané péče v České republice.
			}
		\end{table}
		\end{spacing}

			\subsection{Statistika návštěvnosti dětí v mateřské škole}
				Ze statistiky \cite{Eurydice}, kterou jsem uvedla v kap. \ref{statistika}, se dá vyčíst, že návštěvnost dětí ve věku 4-6 let je v České republice poněkud nižší než ve Francii. Přestože se pohybuje mezi 85\% u 4letých až 96\% u 6letých, je stále relativně vysoká. Tento rozdíl by odpovídal rozdílným podmínkám sociální podpory rodičů. V České republice je možné zůstat doma s každým dítětem až 4 roky. Rodiče, ktekří mají více dětí si mohou přizpůsobit svým potřebám čas, který jejich děti tráví v mateřské škole.Rodič může být např. s jedním dítětem doma na mateřské dovolené a druhé dítě dochází do mateřské školy, kde pobývá každý den jen dopoledne. V České republice se stále drží tradice výchovy dětí v rodině. České rodiný tráví s dětmi více času než rodiny francouzské. V posledních letech je však tato tradice zastíněna trendem různých zájmových kroužků a mimoškolních aktivit, které rodičům zajistí péči o jejich dítě, v případě návratu do zaměstnání.  
				podmínky péče o předškolní děti v České republice	

		
				Tato kapitola je stručným přehledem sociálních a ekonomických podmínek rodin, které mají v péči předškolní dítě. Informace zde uvedené se váží i na kapitolu \ref{rezim}, kde se zmiňuji o otevírací době, a možnostech vyzvedávání dětí z mateřských škol.
