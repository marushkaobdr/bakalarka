%%% Hlavní soubor. Zde se definují základní parametry a odkazuje se na ostatní části. %%%

%% Verze pro jednostranný tisk:
% Okraje: levý 40mm, pravý 25mm, horní a dolní 25mm
% (ale pozor, LaTeX si sám přidává 1in)
% \documentclass[12pt,a4paper]{report}
% \setlength\textwidth{145mm}
% \setlength\textheight{247mm}
% \setlength\oddsidemargin{15mm}
% \setlength\evensidemargin{15mm}
% \setlength\topmargin{0mm}
% \setlength\headsep{0mm}
% \setlength\headheight{0mm}
% % \openright zařídí, aby následující text začínal na pravé straně knihy
% \let\openright=\clearpage

%% Pokud tiskneme oboustranně:
\documentclass[12pt,a4paper,singleside,openright]{report}
\setlength\textwidth{145mm}
\setlength\textheight{247mm}
\setlength\oddsidemargin{15mm}
\setlength\evensidemargin{0mm}
\setlength\topmargin{0mm}
\setlength\headsep{0mm}
\setlength\headheight{0mm}
\let\openright=\cleardoublepage
% radkovani
\renewcommand{\baselinestretch}{1.3}


%% Použité kódování znaků: obvykle latin2, cp1250 nebo utf8:
\usepackage[utf8x]{inputenc}


%% Pokud používáte csLaTeX (doporučeno):
%\usepackage{czech}
%% Pokud nikoliv:
%\usepackage[french]{babel}
\usepackage[french,english,czech]{babel}
\usepackage[T1]{fontenc}
\usepackage{lmodern}


%% Ostatní balíčky
\usepackage{color}
\usepackage{graphicx}
\usepackage{amsthm}
\usepackage{url}
\usepackage{epsfig}
\usepackage{epstopdf}
\usepackage{enumerate}
\usepackage{amsmath}
\usepackage{subfigure}
\usepackage{caption}
\usepackage{tabularx}
\usepackage{lettrine}
\usepackage{natbib}             % sazba pouzite literatury
\usepackage{setspace} 			% zmenit radkovani lokalne
\usepackage{lipsum}
\usepackage{lscape}
\usepackage{float}
\restylefloat{table}

\usepackage[table]{xcolor} 		% barevne tabulky
\definecolor{maroon}{cmyk}{0,0.87,0.68,0.3}
\definecolor{grey}{rgb}{0.8, 0.8, 0.8}

\usepackage{chngcntr}
\counterwithout{figure}{chapter}
\counterwithout{table}{chapter}

%Options: Sonny, Lenny, Glenn, Conny, Rejne, Bjarne, Bjornstrup
\usepackage[Glenn]{fncychap}

\usepackage{hyphenat}
\hyphenation{ISCED}

%% Balíček hyperref, kterým jdou vyrábět klikací odkazy v PDF,
%% ale hlavně ho používáme k uložení metadat do PDF (včetně obsahu).
%% POZOR, nezapomeňte vyplnit jméno práce a autora.
\usepackage[unicode]{hyperref}   % Musí být za všemi ostatními balíčky

\newcommand{\mujNazevPrace}{Komparace současné francouzské a~české mateřské školy ve vybraných aspektech}
\newcommand{\mujNazevPraceAJ}{The comparison of contemporary French and Czech kindergarden in some significant aspects}
\newcommand{\mujVedouci}{doc. PhDr. Jana Uhlířová, CSc.}
\newcommand{\mujKatedra}{Katedra primární pedagogiky}
\newcommand{\mujProgram}{Specializace v~pedagogice}
\newcommand{\mujObor}{Učitelství pro mateřské školy}



% POZOR, nezapomeňte vyplnit jméno práce a autora:
%%%%%%%%%%%%%%%%%%%%%%%%%%%%%%%%%%%%%%%%%%%%%%%%
\hypersetup{pdftitle=Komparace francouzské a~české mateřské školy ve vybraných aspektech}
\hypersetup{pdfauthor=Marie Obdržálková}
%%%%%%%%%%%%%%%%%%%%%%%%%%%%%%%%%%%%%%%%%%%%%%%%

%%% Drobné úpravy stylu

% Tato makra přesvědčují mírně ošklivým trikem LaTeX, aby hlavičky kapitol
% sázel příčetněji a nevynechával nad nimi spoustu místa. Směle ignorujte.
% \makeatletter
% \def\@makechapterhead#1{
%   {\parindent \z@ \raggedright \normalfont
%    \Huge\bfseries \thechapter. #1
%    \par\nobreak
%    \vskip 20\p@
% }}
% \def\@makeschapterhead#1{
%   {\parindent \z@ \raggedright \normalfont
%    \Huge\bfseries #1
%    \par\nobreak
%    \vskip 20\p@
% }}
% \makeatother

% Toto makro definuje kapitolu, která není očíslovaná, ale je uvedena v obsahu.
\def\chapwithtoc#1{
\chapter*{#1}
\addcontentsline{toc}{chapter}{#1}
}

\begin{document}

% Trochu volnější nastavení dělení slov, než je default.
\lefthyphenmin=2
\righthyphenmin=2

%%% Titulní strana práce

\pagestyle{empty}
\begin{center}

\large

Univerzita Karlova v~Praze

\medskip

Pedagogická fakulta

\vfill

{\bf\Large BAKALÁŘSKÁ PRÁCE}

\vfill

%\centerline{\mbox{\includegraphics[width=60mm]{img/logo}}}

\vfill
\vspace{5mm}

{\LARGE Marie Obdžálková}

\vspace{15mm}

% Název práce přesně podle zadání
{\LARGE\bfseries
	\mujNazevPrace
\large\bfseries \\
\vspace{8mm}
	\mujNazevPraceAJ
}

\vfill

% Název katedry nebo ústavu, kde byla práce oficiálně zadána
% (dle Organizační struktury MFF UK)
\mujKatedra

\vfill

\begin{tabular}{rl}

Vedoucí bakalářské práce: & \mujVedouci \\
\noalign{\vspace{2mm}}
Studijní program: & \mujProgram \\
\noalign{\vspace{2mm}}
Studijní obor: & \mujObor \\
\end{tabular}

\vfill

% Zde doplňte rok
Praha 2014

\end{center}

\newpage
	%%% Strana s čestným prohlášením k bakalářské práci

	\vglue 0pt plus 1fill

	\noindent
	Prohlašuji, že jsem bakalářskou práci na téma Komparace současné francouzské a~české mateřské školy ve vybraných aspektech vypracovala pod vedením vedoucího bakalářské práce samostatně za použití v práci uvedených pramenů a~literatury. Dále prohlašuji, že tato bakalářská práce nebyla využita k získání jiného nebo stejného titulu.


	\vspace{10mm}

	\hbox{\hbox to 0.5\hsize{%
	V~Praze dne 5.12.2014
	\hss}\hbox to 0.5\hsize{%
	Podpis autora
	\hss}}

	\vspace{20mm}
\newpage
	%%% Následuje vevázaný list -- kopie podepsaného "Zadání bakalářské práce".
	%%% Toto zadání NENÍ součástí elektronické verze práce, nescanovat.

	%%% Na tomto místě mohou být napsána případná poděkování (vedoucímu práce,
	%%% konzultantovi, tomu, kdo zapůjčil software, literaturu apod.)
	\openright

	\textcolor{white}{ }
	\vspace{150mm}

	\section*{Poděkování}

	Chtěla bych poděkovat doc. PhDr. Janě Uhlířové CSc. za vedení mé práce, za podnětné přípomínky a~čas, který mi věnovala.
	Dále bych chtěla poděkovat své rodině za morální podporu a~příteli za podporu technickou. 

\newpage

%%% Povinná informační strana bakalářské práce

\vbox to 0.5\vsize{
\setlength\parindent{0mm}
\setlength\parskip{3mm}
\begin{spacing}{1.0}
	\emph{Název práce:}
	\textbf{\mujNazevPrace}
	% přesně dle zadání

	\emph{Autor:}
	Marie Obržálková

	\emph{Katedra:}  % Případně Ústav:
	\mujKatedra
	% dle Organizační struktury MFF UK

	\emph{Vedoucí bakalářské práce:}
	\mujVedouci
	% dle Organizační struktury MFF UK, případně plný název pracoviště mimo MFF UK

	\emph{Abstrakt:}	
	Cílem této práce je komparace současné francouzské a~české mateřské školy ve vybraných aspektech. Těmi jsou cíle a~pojetí předškolního vzdělávání a~režim běžného dne jedné třídy mateřské školy.
	Práce má dvě části, teoretickou a~empirickou. 
	Teoretická část pojednává o~zakotvení mateřských škol ve vzdělávacím systému, pojetí dítěte, podmínkách péče o~předškolní děti a~důkladně analýzuje a~srovnává legislativní dokumenty vzdělávání Francie a~České republiky. Obsahuje i~překlad vybraných částí kurikula z~francouzského originálu. Empirická část je založena na pozorování a~srovnání průběhu dne a~podmínek, v~nichž probíhá předškolní vzdělávání. Komparace sledovaných aspektů ukazuje rozdíly v~pohledu na dítě a~v~aplikování teoretických dokumentů do praxe jak ve Francii, tak v~České republice. 

	\emph{Klíčová slova:}
	komparace, systém předškolní výchovy ve Francii, mateřská škola, vzdělávací cíle, kurikulum, režim dne, pojetí dítěte
	% 3 až 5 klíčových slov
\end{spacing}
	\vss}\nobreak\vbox to 0.49\vsize{
	\setlength\parindent{0mm}
	\setlength\parskip{3mm}

	\emph{Title:}
	% přesný překlad názvu práce v angličtině
	\textbf{\mujNazevPraceAJ}

	\emph{Author:}
	Marie Obdžálková

	\emph{Department:}
	Department of primary pedagogy

	\emph{Supervisor:}
	\mujVedouci
	% dle Organizační struktury MFF UK, případně plný název pracoviště
	% mimo MFF UK v angličtině
\begin{spacing}{1.0}
	\emph{Abstract:}
	The aim of this work is a~comparison of contemporary French and Czech kindergarden in some significant aspects. These are (i) educational goals in the kindergardens and (ii) conception of the preschool education and daily schedule of a~kindergarten class. This work consists of two major parts, the theoretical part and the empirical part. 
	The theoretical part discusses an embedding of a~kindergarden in the educational system, conception of a~child, conditions of child care in a~preschool age and, finally, it precisely analyzes legislative documents of the educational programs in France and the Czech republic. The analysis also includes a~translation of selected parts of French curriculum. The empirical part of this work is based on observation of a~daily schedule and conditions in which the preschool education takes place. The comparison of both above mentioned aspects highlights the differences between a~conception of a~child and the application of the theoretical document in practice in both countries, the Czech republic and France.
 
	% zadání bakalářské práce

	Keywords:
	comparison, preschool system in France, kindergarden, educational aims, curriculum, daily schedule, conception of a~child
	% 3 až 5 klíčových slov v angličtině
\end{spacing}

	\vss}
\newpage

%%% Strana s automaticky generovaným obsahem bakalářské práce. U matematických
%%% prací je přípustné, aby seznam tabulek a zkratek, existují-li, byl umístěn
%%% na začátku práce, místo na jejím konci.

\openright
\pagestyle{plain}
\setcounter{page}{1}  % nastaví čítač stránek znovu od jedné
%\pagenumbering{Roman} % číslování římskými číslicemi
\tableofcontents


%%% Jednotlivé kapitoly práce jsou pro přehlednost uloženy v samostatných souborech

\chapter{Úvod}

Cílem výzkumného projektu bakalářské práce je komparace francouzské a české mateřské školy. Vybranými aspekty jsou cíle a kurikulum předškolního vzdělávání a režim dne v mateřské škole obou srovnávaných zemí. 
\part{TEORETICKÁ ČÁST}
\chapter{CÍLE PRAKTICKÉ ČÁSTI}
Jak napovídá název, tato bakalářská práce se věnuje komparaci současné mateřské škoy ve Francii a v České republice. Vybranými aspekty v této práci jsou kurikulum a režim dne. Za tímto účelem byly vytčeny tyto cíle:

\begin{itemize}
	\setlength\itemsep{-2mm}
	\item [-] Analýza legislativních dokumentů věnující se předškolnímu vzdělávání ve Francii.
	\item [-] Analýza legislativních dokumentů věnující se předškolnímu vzdělávání v České republice.
	\item [-] Analýza cílů a vzdělávacích oblastí kurikul obou sledovaných zemí.
	\item [-] Pozorování průběhu dne ve francouzské mateřské škole
	\item [-] Pozorování průběhu dne v české mateřské škole. 
	\item [-] Komparace režimu dne v mateřské škole obou sledovaných zemí
\end{itemize}




\chapter{METODY}

Ve své práci jsem využila dvou výzkumných metod, a to obsahové analýzy a pozorování. Nejprve byla provedena analýza kurikurálních dokumentů a vzdělávacích oblastí předškolního vzdělávání Francie a České republiky a poté zpracováno pozorování z průběžné praxe, jak ve Francii, tak v České republice.

\textit{„Obsahová analýza je dôležitým nástrojom poznania jednotlivých oblastí výchovy a vzdelávania.“ }(Gavora, 2008). Jedná se o velmi mladou výzkumnou metodu ze 40.let 20.století, která byla původně využívána v masmédiích a postupně si nacházela své místo i v humanitních oborech a v neposlední řadě i v pedagogice. Lze ji uskutečňovat nekvantitativním nebo kvantitativním způsobem. V mém případě se jedná o první způsob, kdy nejde o převedení kvalitativní parametrů (pojmy, slova, témata) na kvantitativní míru či numerickou hodnotu, ale o popis obsahu kurikulárních dokumentů a jeho následné srovnání. 

Druhou výzkumnou metodou bylo pozorování, které spadá mezi metody kvalitativní. Vzhledem k podmínkám praxe, kterou jsem absolvovala ve Francii, kde nebylo dovoleno zapojovat se do dění, jedná se o pozorování nestrukturované, při kterém se podle Gavory (1996,str.17): \textit{„nepoužívají předem stanovené pozorovací systémy, škály anebo jiné přesné nástroje. Určeny jsou jen konkrétní události, jevy a osoby, které se mají pozorovat.“} 

Konkrétními událostmi při tomto pozorování byl časový harmonogram a program dne dětí a zázemí jedné třídy mateřské školy ve Francii a jedné třídy v České republice, o nichž jsem si dělala podrobné písemné záznamy, neboli vzorky událostí (angl. specimen records). 

	
\chapter{ZÁKLADNÍ INFORMACE O MATEŘSKÝCH ŠKOLÁCH SROVNÁVANÝCH ZEMÍ}
%TODO JA; nejaky kecy tady

	\section{Zařazení mateřské školy v rámci klasifikace vzdělávacího systému}

		V roce 1976 vydalo UNESCO Mezinárodní standardní klasifikaci vzdělávání ISCED (International Standard Classification of Education), která slouží \uv{jako nástroj vhodný pro shromažďování, zpracování a zpřístupňování vzdělávacích statistik jak v jednotlivých zemích, tak v mezinárodním měřítku"} \citep{ISCED}. Klasifikace kmenových oborů vzdělávání z roku 1997 má 7 úrovní vzdělávání (0 až 6).
		Pro účely této práce je důležité si představit první dvě úrovně:

\begin{itemize}
	\setlength\itemsep{-2mm}
	\item [] \textbf{ISCED 0} - Vzdělávání v raném dětství (preprimární vzdělávání,mateřské školy). Programy na této úrovni mají podporovat poznávací, fyzický, sociální a emocionální rozvoj malých dětí, uvádět je do organizované výuky mimo kontext rodiny a rozvíjet jejich emocionální dovednosti nezbytné pro školní docházku a zapojení do společnosti. 
	\item [] \textbf{ISCED 1} - Primární vzdělávání (základní vzdělání, základní školy včetně speciálních - 1.stupeň, zvláštní školy - 1. a 2. stupeň, pomocné školy - nižší, střední a vyšší stupeň a rehabilitační třídy). Programy na této úrovni mají žákům poskytovat základní dovednosti ve psaní, čtení a počítání a vytvářet pevný základ pro učení a porozumění jádru vědění, pro osobní a sociální rozvoj v rámci přípravy na nižší sekundarní vzdělávání. \citep{ISCED}
\end{itemize}

		Preprimární vzdělávání neboli také předškolní vzdělávání spádá do úrovně ISCED 0. Jedná se nepovinné vzdělávání, které uvádí děti raného věku do prostředí institucionálního zařízení. 

		Podle \citet{KeyData} se v mateřských školách v evropských zemích uplatňují různé modely:
		\begin{enumerate}[1)]
			\setlength\itemsep{-2mm}
			\item Školský model (school model) – preprimární vzdělávání je organizované ve třídách, v nichž jsou zařazeny děti podle věkových kategorií, tedy podobně jako ve skutečné škole. 
			\item Rodinný model (family model) – preprimární vzdělávání je organizováno ve skupinách sdružujících děti různého věku, tedy podobně jako ve skutečných rodinách. 
			\item Oba modely
		\end{enumerate}

		Tyto modely jsou však odlišné i svými vždělávacími cíli. Školský model připravuje děti na vstup do základní školy, kdežto rodinný typ se věnuje spíše rozvoji sociálních dovedností a uvedení dětí do společnosti.

		Realizace předškolního vzdělávání se liší zemi od země. Různé jsou jak cíle tak obsah vzdělávání. Pro porozumění tedy v dalších kapitolách uvádím pozici, kterou mají mateřské školy ve vzdělávacím systému obou sledovaných zemí. 
		

	\section{Mateřské školy ve Francii}
%TODO obrazek
		Mateřské školy ve Francii jsou státní instituce zajišťující preprimární vzdělávání. Dlouholetá tradice nahlíží na předškolní vzdělávání (école maternelle) jako na počáteční formu vzdělávání, na níž navazuje primární vzdělávání (école élémentaire). Jde o návaznost ISCED  úrovně 0 a 1. Mateřská škola poskytuje péči dětem od 2 do 6 let, je však součástí základního vzdělávání poskytující vzdělávání pro děti od 2 do 11 let.

		Primární vzdělávání ve Francii se odehrává ve 3 cyklech. Prvním cyklem (cycle des apprentissages premiers)je mateřská škola, poslední třída mateřské školy (grande section) je již přechodem do druhého cyklu (cysle des apprentisages fondamentaux), jehož součástí je přípravná třída (cours préparatoire CP), na kterou navazují první třída základního vzdělávání (cours élémentaire CE1). Ve třetím cyklu (cycle des approfondissements) je druhá třída základního vzdělávání (cour élémentaire CE2 a dvě střední třídy (Cours moyenne CM1 a CM2). Poté děti přecházejí na sekundární vzdělávání na collège, které odpovídá našemu druhému stupni základních škol. 

		Vzdělání v mateřských školách odpovídá tzv. prvnímu učebnímu cyklu (cycle des apprentissages premiers), rozdělenému do tří stupňů podle věku žáků: nižší stupeň (petite section) pro děti tří až čtyřleté; střední stupeň (moyenne section) pro děti čtyř až pětileté; vyšší stupeň (grande section) pro děti pěti až šestileté.
		(Průcha, 2012). 
% TODO JA: reference prucha
		Je-li v mateřské škole dostatek místa, jsou přijímany děti již od 2 let do tzv. toute petite section. 

		Školství ve Francii je od svých počátků centralizované. Od roku 1982 začala jeho decentralizace, která přerozdělila pravomoc státní administrativy a lokálních samospráv. Stát zůstává garantem vzdělávání jako veřejné služby a definuje rámec vzdělávání a kurikula. Mateřské školy jsou pod pravomocí Ministerstva školství (Ministère de l´éducation national)
%TODO odkaz
%(http://www.clovekvtisni.cz/uploads/file/1360764270-an_KA3_komparace.pdf)

		V roce 1886 byl vydán zákon, podle kterého jsou mateřské školy veřejné, bezplatné a sekulární instituce, a který vymezuje jejich vzdělávací funkce.
%TODO JA principy

\begin{spacing}{1.0}
\begin{table}[h]
	\small
	\begin{center}
	\begin{tabular}{|c|c|c|c|}
		\hline
		\rowcolor{grey}
		\textbf{Cyklus}				& \textbf{Třída}		& \textbf{Věk}	& \textbf{Kde se odehrává}	\\
		\hline
		\hline
		\rowcolor{grey!10}
		%==================================================================================================
		cycle des apprentissages	& toute petite section 	& 2-3 		&				\\ \rowcolor{grey!10}
		premiers (1. cyklus)		& petite section 		& 3-4 		& jen v MŠ 		\\ \rowcolor{grey!10}
									& moyenne section 		& 4-5 		& 				\\ \rowcolor{grey!10}
		\hline
		%==================================================================================================
		cycle des apprentissages 	& grande section 		& 5-6 		& začíná v MŠ, 		\\ \rowcolor{grey!10}
		fondametaux (2.cyklus) 		& CP 					& 6-7 		& pokračuje na ZŠ 	\\ \rowcolor{grey!10}
									& CE1 					& 7-8 		& 					\\ \rowcolor{grey!10}
		\hline
		%================================================================================================+=
		cycle des approfonissements & CE2 					& 8-9 		&					\\ \rowcolor{grey!10}
		(3.cyklus)					& CM1 					& 9-10 		& jen v ZŠ 			\\ \rowcolor{grey!10}
									& CM2 					& 10-11 	& 					\\ \rowcolor{grey!10}
		\hline
	\end{tabular}
	\end{center}
	\caption{ \textbf{Primární vzdělávání ve Francii.}
		TODO: Strucny popis tbaulky aby to nekdo pochopil bez cteni textu.
	}
	\label{tab:primarniVzdelavaniFR}
\end{table}
\end{spacing}

	\section{Mateřské školy v České republice}
%TODO obrazek
		Mateřská škola v České republice je instituce zajišťující předškolní vzdělávání pro děti od 3 do 6 let (do 7 let v případě odkladu školní docházky), které se školským zákonem stalo legitimní součástí systému vzdělávání. Podle mezinárodní klasifikace se jedná o ISCED 0. Jedná se o organizované vzdělávání, které musí splňovat požadavky MŠMT (Ministerstva školství, mládeže a tělovýchovy). Předškolní vzdělávání v mateřské škole je veřejnou, nepovinnou a bezplatnou službou pro všechny děti. Přednostně jsou přijímány děti v posledním roce před začátkem povinné školní docházky. 

		Ve veřejné sféře je zřizovatelem mateřské školy většinou obec nebo svazek obcí. V České republice existují i soukromé mateřské školy.

		Organizačně se mateřská škola dělí na třídy, které je možné vytvářet podle věku, a to na třídy věkově homogenní a na třídy věkově heterogenní. Do mateřských škol je možné zařazovat i děti se specifickými potřebami a vytvářet tak třídy integrované. 

		
		Předškolní vzdělání v mateřské škole má 3 ročníky:
		
	\begin{itemize}
		\setlength\itemsep{-2mm}
		\item [] \textit{\uv{V prvním ročníku mateřské školy se vzdělávají děti, které v příslušném školním roce dovrší nejvýše 4 roky věku.}}
		\item [] \textit{\uv{V druhém ročníku mateřské školy se vzdělávají děti, které v příslušném školním roce dovrší nejvýše 5 let věku.}}
		\item [] \textit {\uv{Ve třetím ročníku mateřské školy se vzdělávají děti, které v příslušném školním roce dovrší 6 let věku a děti, kterým byl povolen odklad povinné školní docházky.}} \citep[s.~71]{Organizace}
	\end{itemize}

%TODO JA preformulovat
Přes formální shodu postavení mateřské školy ve vzdělávacím systému obou sledovaných zemí je třeba poukázat na naprosto zásadní rozdíly v cílech mateřských škol a pohledu na dítě, které ji navštěvuje.
Předškolní výchova ve Francii tak jako ve většině románských zemí je charakteristická tím, že naplňuje cíl uvádět dítě do světa školy. Tzn. směřovat práci v mateřské škole k přípravě na vstup do povinné školní docházky. Tento přístup je hluboce tradičně zakořeněn, a tak jak.....při rozboru kurikul směřuje k získání základních kulturních technik, na nichž je ... postaven počátek primárního vzdělávání. 

Předškolní vzdělávání v České republice nebylo takto jednoznačně orientované ve své historii, tj. příprav na školu, přeste však tento aspekt vyplynul jako nezbytnost s přijetím dokumentu \uv{Další rozvoj výchovně vzdělávací soustavy} v roce 1976. Tehdy byl cíl mateřských škol zúžen na přípravu pro povinné vzdělávání. děti, které neabsolvovaly mateřskou školu, byly při zápisech do základní školy ....(zarazeni) k náhradnímu opatření, tzn. přípravných tříd, alespoň na dobu 3 měsíců, neboť dovednosti a znalosti získané před nastupem do 1. třídy byly východiskem, na nichž 1. třída \uv{startovata}. Současná mateřská škola není vázána konceptem na školu, její koncept je mnohem širší. Příprava na život v sobě zahrnuje také připravu na povinnou školní docházku. Ovšem v kontextu socializace a radostného dětství s ostatnimi dětmi, tj. \uv{rosteme společně}. Současná předškolní výchova v České republice neni ani školský model, ani rodinný model, ale je to smíšený model obou černobíle postavených typů.

Vzdělávací systemy jsou odlišné, i když by se na první pohled zdálo, že mateřská škola přijímající děti od 2 do 6 let má svou stejnou pozici. Historicky byla francouzská mateřská škola vyjímána vždycky jako vzdělávací instituce. Opravdu tvoří první článěk vzdělávací soustavy (ale nepovinný), o to je překvapivější, že cykly, které dítě v předškolním věku prochází jsou vnímány jako nezbytný obsah na niž se váže povinná školní docházka. Tento stav se jeví jako anomálie. Přestože je nepovinná, 100\% 5ti letých dochází. Všechny rodiny, které žádají o vstup dětí ve 3 letech jsou přijati (ne 2letí).

Tradice české mateřské školy je rovněž velmi dlouhá, ale její pozice jako vzdělávací instituce se vztahuje až ke školskému zákonu, kdy je zařazena jako první článěk vzdělávací sousty. Po roce 1948 se pozice MŠ více blížila sociálnímu zařízení, než skutečně výchovně vzdělávacímu. Mezi lety 1948 a 1989 je její vzdělávací charakter nespochybnitelný. Po roce 1989 byl krátkodobě zpochybněn vzdělávací charakter ve prospech pozice sociální, avšak na konci 90. let, zvláště pak školským zákonem v roce 2004 se její pozice zakotvila a posílila. 

\chapter{POJETÍ DÍTĚTE VE FRANCII A~ČESKÉ REPUBLICE}

Přestože se postavení mateřských škol ve vzdělávacím systému obou zemí formálně shoduje, v~cílech mateřských škol a pohledu na dítě se obě země zásadně odlišují. Pojetím dítěte v~rámci této práce je myšlen pohled na dítě v~kontextu vzdělávacích cílů mateřských škol. 

Hlavním cílem ve francouzském předškolním vzdělávání je uvádět dítě do světa školy, tzn. směřovat práci v~mateřské škole k~přípravě na vstup do povinné školní docházky. Toto pojetí je tradiční ve většině románských zemí. Jedná se o~školský model (viz kapitola 1.1).

 Česká republika je tímto pojetím také ovlivněna, nicméně koncept české mateřské školy je mnohem širší. Cílem není pouze příprava na roli žáka, ale zejména celková příprava na život, která v~sobě zahrnuje také přípravu na povinnou školní docházku. Důraz je v~České republice kladen zejména na socializaci a radostné dětství s~ostatními dětmi, tj. \uv{rosteme společně}. Nejedná se tedy čistě jen o~školský nebo rodinný model, ale lze o~něm hovořit jako o~modelu smíšeném. 

 Tato pojetí jsou ovlivněna historickým vývojem mateřských škol a postavením dítěte v~nich. Historicky byla francouzská mateřská škola vnímána jako vzdělávací instituce. Tvoří první článek vzdělávací soustavy (viz obr. \ref{tab:rozdeleniTridFR}). Cykly, kterými dítě v~předškolním věku prochází, jsou vnímány jako nezbytný obsah, na nějž se váže povinná školní docházka. 

Tradice české mateřské školy je také velmi dlouhá, avšak ve své historii nebyla jednoznačně orientovaná jako příprava na školu. Její pozice jako vzdělávací instituce se vztahuje až ke školskému zákonu, kdy je zařazena jako první článek vzdělávací soustavy. 

 Do roku 1948 se pozice MŠ více blížila sociálnímu zařízení než zařízení výchovně vzdělávacímu. Mezi lety 1948 a 1989 je její vzdělávací charakter nezpochybnitelný. S~přijetím dokumentu \uv{Další rozvoj výchovně vzdělávací soustavy} v~roce 1976 byl posílen aspekt přípravy na školu. Cíl mateřských škol byl zúžen na přípravu pro povinné vzdělávání. Děti, které neabsolvovaly mateřskou školu, byly při zápisech do základní školy vyzvány k~náhradnímu opatření, tzv. přípravných tříd, alespoň na dobu 3 měsíců, neboť dovednosti a znalosti získané před nástupem do první třídy byly východiskem, na němž první třída \uv{startovala}. Po roce 1989 byl krátkodobě zpochybněn vzdělávací charakter ve prospěch pozice sociální, avšak na konci devadesátých let, zvláště pak školským zákonem v~roce 2004 se vzdělávací pozice mateřské školy zakotvila a posílila (viz přednášky Dějiny předškolní pedagogiky, akademický rok 2011/2012).


\chapter{PÉČE O DÍTĚ PŘEŠKOLNÍHO VĚKU SROVNÁVANÝCH ZEMÍ}

	Přístup k nejmenším dětem je ovlivněn mnoha faktory. Jinak na dítě pohlíží nejbližší rodina a jinak ho vidí společnost. Možnost péče o předškolní děti je z velké části ovlivněna ekonomickými podmínkami rodiny a sociální podporou státu. Nerodinná a institucionální péče začíná tam, kde končí možnosti celodenní rodinné péče. Tento faktor je ovlivněn podmínkami mateřské a rodičovské dovolené a možnostmi další péče o děti.

	V této kapitole vychází autorka i z vlastních zkušeností chůvy u třech francouzských rodin v letech 2010/2012.

		\section{Podmínky péče o předškolní děti ve Francii}
		Péče o děti ve Francii má dva pilíře finanční podpory: finanční podpora vyplácena přímo rodičům a finanční podpora vyplácena poskytovatelům služeb nerodinné péče. 


			\subsection{Mateřská dovolená}
				Od roku 1970 je ve Francii zavedena mateřská dovolená pro všechny zaměstnance, která je placená ze sociálního pojištění a činí 90 \% hrubé mzdy. Minimální délka mateřské dovolené je 16 týdnů, tedy 6 týdnů před porodem a 10 týdnů po porodu, tato doba se mění v závislosti na době porodu, zdravotních komplikacích a počtu dětí (3 a více dětí až 26 týdnů). Minimálně je žena povinna vyčerpat 8 týdnů mateřské dovolené. Příspěvek je vyplácen, jestliže žena platila po dobu 10ti měsíců pojištění a pracovala alespoň 200 hodin poslední 3 měsíce před nástupem na mateřskou dovolenou. \citep{Dennipece}

			\subsection{Rodičovská dovolená}
				Rodičovská dovolená byla zavedena v roce 1977. Umožňuje matkám (resp. otcům) přerušit zaměstnání po narození dítěte při zajištění možnosti návratu k práci u svého zaměstnavatele po jejím ukončení. Rodičovská dovolená trvá 6 měsíců a je možné ji čerpat do tří let věku dítěte. Možností je její opakované prodlužování. Rodičovská dovolená je neplacená. Příspěvky se dostávají až od druhého dítěte. 

				V roce 2004 byly všechny příspěvky sjednoceny do jedné dávky, ve kterém je mimo jiné říspěvek na péči o dítě nerodičovskou osobou nebo též rodičovský příspěvek pro matky s jedním dítětem na dobu 6ti měsíců. Příspěvek je možné pobírat po 2 letech přispívání do důchodového systému. \citep{Dennipece}

			\subsection{Péče o dítě nerodinnou osobou}
				Ve Francii je dlouhá tradice mateřských asistentek. Tyto asistentky by měly být licencované a dokázat schopnost postarat se o děti a jejich zdravý vývoj. V jeden čas smí mít v péči max. 3 děti. Jedná se o péči o děti do 3 let. Asistentka dochází buď do bydliště rodiny, nebo přijímá děti u sebe doma a je zaměstnancem rodiny, která jí vyplácí mzdu. Rodina dostává na mateřskou asistentku dotace od státu.
				Nutnost mateřských asistentek vyplývá z časného nástupu matek zpět do zaměstnání a nedostatku jeselských zařízení, o která je větší zájem, než jsou kapacitní možnosti spádových jeslí. Mateřské asistentky a dále chůvy doprovází velkou část rodin po celou dobu docházky dětí do jeslí, mateřské školy a někdy i školy základní. Asistentky a chůvy vodí děti do institucí zajišťující péči, ze kterých je také vyzvedávají a starají se o ně do příchodu rodičů.
		\begin{spacing}{1.0}
		\begin{table}[ht]
			\small
			\begin{center}
			\begin{tabular}{|c|c|c|c|}
				%=========================================================================================
				\hline
				\rowcolor{grey}		
				\textbf{Typy}	&	\textbf{Finance} & 	\textbf{Délka} 	&	\textbf{Opatření} 	\\
				\hline
				\hline
				%=========================================================================================
				\rowcolor{grey!10}
				Mateřská	&  placena ze sociálního &  min. 16 týdnů, 	 & 10 měsíců hrazení 		\\ \rowcolor{grey!10}
				% pokracovani prvniho radku
				dovolená 	& 	 pojištění, činí  	 & 	povinně 8 týdnů, & pojištění, 200 odpra- 	\\ \rowcolor{grey!10}
				% pokracovani prvniho radku 
				 			& 	90 \% hrubé mzdy 	 &  max. 26 týdnů 	 & covaných hodin posled- 	\\ \rowcolor{grey!10}
				 			&						 & 					 & ní 3 měsíce před 		\\ \rowcolor{grey!10}
				 			&						 &					 & nástupem na mateřskou 	\\ \rowcolor{grey!10}
				 			&						 & 					 & dovolenou 				\\ \rowcolor{grey!10}
				%=========================================================================================
				\hline
				Rodičovská	& neplacená & 6 měsíců (do tří let 		& 	zaručen návrat 	\\ \rowcolor{grey!10}
				% pokracovani druheho radku
				dovolená & (příspěvky až od & věku), možné opako-  	&  do zaměstnání	\\ \rowcolor{grey!10}
						 & druhého dítěte)  & vaně prodlužovat						&	\\ \rowcolor{grey!10}
				%=========================================================================================
				 \hline
				Péče o dítě	&	státní dotace vrací	&	dle potřeby	& max. 3 děti \\ \rowcolor{grey!10}
				% pokracovani tretiho radku
				nerodinnou 	&	cca 50 \% nákladů 	&	& na 1 mateřskou 	\\ \rowcolor{grey!10}
				osobou 		& 						&	& asistentku		\\ \rowcolor{grey!10}
				%=========================================================================================
				\hline
				Mateřské 	&	bezplatné	& od 2 do 6 let	& 100 \% účast 4-6ti letých \\ \rowcolor{grey!10}
				školy 		& 	 			& věku dítěte	& 						\\ \rowcolor{grey!10}
				\hline
			\end{tabular}
			\end{center}
			\label{tab:peceFR}
			\caption{
				\textbf{Podmínky péče o předškolní děti ve Francii.}
				Tabulka shrnuje typy péče o předškolní dítě, financování státem, časovou dotaci a podmínky čerpání dané péče ve Francii.
							}
		\end{table}
		\end{spacing}

			\subsection{Statistika návštěvnosti dětí v mateřské škole}
			\label{statistika}
				Přestože se státními dotacemi na mateřskou asistentku a chůvu rodinám vrátí cca 50 \% nákladů, zůstává tato služba relativně drahá. Většina dětí od 3 do 6 let navštěvuje mateřskou školu. Podle \cite{Eurydice} je účast na předškolním vzdělávání 4 až 6ti letých 100 \%. Tento stav je v rámci Evropské unie unikátní. 
			

		\section{Podmínky péče o předškolní děti v České republice}
		Rodinám v České republice se též dostává finační podpory od státu. Avšak podmníky čerpání mateřské a rodičovské dovolené se markatně odlišují.

			\subsection{Mateřská dovolená}
				Mateřská dovolená neboli peněžitá pomoc v mateřství se vyplácí zaměstnankyním po dobu 28 týdnů (resp. 37 týdnů u více dětí). Podmínkou pobírání tohoto příspěvku je účast na nemocenském pojištění a vypočítává se ze mzdy za posledních 12 měsíců. Od června 2014 činí mateřská dovolená 70 \% hrubé mzdy. \citep{materska}

			\subsection{Rodičovská dovolená}
				Rodičovskou dovolenou mohou pobírat jak matky, tak otcové a žádá se o ní s koncem mateřské dovolené nebo po narození dítěte rodičům, kterým nevznikl na mateřskou dovolenou nárok. Rodičovský příspěvek je sociální dávka, na kterou má nárok každý, kdo se účastnil na zdravotním pojištění. Celková částka činí 220 000 Kč, tu lze pobírat nejméně 19 měsíců až do 4 let věku dítěte. Rodičovskou dovolenou lze zkracovat či prodlužovat každé tři měsíce do vyčerpání celé částky. \citep{rodicovska}

			\subsection{Péče o dítě nerodinnou osobou}
				V České republice nemají chůvy dlouhou tradici. Starost o děti dříve zastávali prarodiče. Ti jsou v dnešní době často ještě sami v pracovním poměru, tudíž se o děti starat nemohou. Z tohoto důvodu začíná být ze strany rodičů o chůvy čím dál větší zájem. Tato oblast ovšem není zabezpečena legislativou, tzn. stát na chůvy nijak finančně nepřispívá, stejně jako nejsou dané podmínky na vzdělání chův či pracovní podmínky. 


			\subsection{Statistika návštěvnosti dětí v mateřské škole}
		
				Ze statistiky \cite{Eurydice}, uvedené v kap. \ref{statistika}, se dá vyčíst, že návštěvnost dětí ve věku 4 - 6 let je v České republice poněkud nižší než ve Francii. Přestože se pohybuje mezi 85 \% u 4letých až 96 \% u 6letých, je stále relativně vysoká. Tento rozdíl by odpovídal rozdílným podmínkám sociální podpory rodičů. V České republice je možné zůstat doma s každým dítětem až 4 roky. Rodiče, kteří mají více dětí, si mohou podle svých potřeb upravit čas, který jejich děti v mateřské škole tráví. Rodič může být např. s jedním dítětem doma na mateřské dovolené, zatímco druhé dítě dochází do mateřské školy jen během dopoledne. V České republice se stále drží tradice výchovy dětí v rodině. České rodiny tráví s dětmi více času než rodiny francouzské. V posledních letech je však tato tradice zastíněna trendem různých zájmových kroužků a mimoškolních aktivit, které rodičům zajistí péči o jejich dítě v případě návratu do zaměstnání.  


				
	Tato kapitola je stručným přehledem sociálních a ekonomických podmínek rodin, které mají v péči předškolní dítě. Informace zde uvedené se váží i na kapitolu \ref{rezim}, kde se zmiňuji o otevírací době a možnostech vyzvedávání dětí z mateřských škol.

\begin{spacing}{1.0}
\begin{table}[h!]
	\small
	\begin{center}
	\begin{tabular}{|c|c|c|c|}
		\hline
		\rowcolor{grey}
		\textbf{Typy}	 & \textbf{Finance}		& \textbf{Délka}		& \textbf{Opatření}	 \\
		\hline
		\hline
		%============================================================================================
		Mateřská & 70 \% hrubé mzdy 		& 28, resp. 37 týdnů	& nutná účast  			\\ \rowcolor{grey!10}
		dovolená & za posledních	 		& 						& na nemocenském 		\\ \rowcolor{grey!10}
				 & 12 měsíců				& 						& pojištění				\\ \rowcolor{grey!10}
		\hline
		%============================================================================================
		Rodičovská 	& celková částka  		& 19 měsíců až 4 roky	& sociální dávka pro 	\\ \rowcolor{grey!10}
		dovolená 	& 220 000 Kč			& 					& každého, kdo se účastní	\\ \rowcolor{grey!10}
					&						&						& zdravotního pojištění \\ \rowcolor{grey!10}
		\hline
		%============================================================================================
		Péče o dítě & není podporováno		& dle potřeby			&   					\\ \rowcolor{grey!10}
		nerodinnou	& státem				&						&	-					\\ \rowcolor{grey!10}
		osobou 		&						&						&						\\ \rowcolor{grey!10}
		\hline
		%============================================================================================
		Mateřské	& 	bezplatné 			& Od 3 do 6 (resp.7) let & 85 \% účast 4letých 	\\ \rowcolor{grey!10}\
		školy 		& 						&  						 & 96 \% účast 6letých 	\\ \rowcolor{grey!10}
		\hline
	\end{tabular}
	\end{center}
	\label{tab:peceCR}
	\caption{
		\textbf{Podmínky péče o předškolní děti v České republice.}
				Tabulka shrnuje typy péče o předškolní dítě, financování státem, časovou dotaci a podmínky čerpání dané péče v České republice.
	}
\end{table}
\end{spacing}	
% HACK: Ta bulka zarovnana nahore -------
\vspace{10cm}
\textcolor{white}{.}
% ---------------------------------------



\chapter{KURIKULUM}
\label{kurikulum}
Pojem kurikulum je odvozeno z latinského slova \textit{currere} tedy běžet.  Slovo curriculum má významy jako \textit{\uv{běh, závodiště či závodní vozík}}. Může to tedy znamenat \textit{\uv{pohyb po určité trase, směrem k určitému cíli}}.\citep[s.~24]{Opravilova} Nejčastěji je dnes toto slovo používané jako curriculum vitae neboli životopis, běh života. V češtině se používá jeho přepis kurikulum.

Ve spojení s pedagogikou se tento pojem začíná užívat v zahraničí v 60. letech 20. století. Dá se chápat jako plánovaná a nasměrovaná trasa, na níž dítě získává zkušenosti dle svých schopností a zájmů. V České republice se tento pojem užívá až koncem 80. let. Existuje mnoho jeho definic a významů podle různých pedagogických koncepcí a názorů samostatných autorů.

K rozšíření pojmu kurikulum u nás významně přispěla Eliška Walterová. Díky ní se dostal do české odborné pedagogické terminologie. Pro tuto práci se hodí dva významy podle \citet[s.~15]{Walterova}:

\uv{\textit{\textbf{Vzdělávací program, projekt, plán:} 
		zahrnuje škálu od programu jednotlivého kurzu nebo vyučovacího předmětu až po komplexní program vzdělávací instituce, tj. plán všech aktivit ve škole;}

\textit{\textbf{Průběh studia a jeho obsah:}
		charakteristika vzdělávací dráhy a obsah zkušeností, kterou žák získává v době studia.}}

% TODO ja: stranky u Pruchy
\citet{Prucha} definuje kurikulum jako \uv{\textit{obsah vzdělávání, který zahrnuje veškeré zkušenosti, které žáci získávají ve škole a v činnostech ke škole se vztahujících, zejména jejich plánování, zprostředkovávání a hodnocení.}}

Dle terminologie se dělí kurikulum na \textbf{formální}, \textbf{neformální} a \textbf{skryté}. V této práci se zabývám kurikulem formálním. Formální kurikulum je komplexní projekt cílů, obsahu, prostředků a organizace vzdělávání, realizace projektovaného kurikula, způsoby kontroly a hodnocení výsledků.

Kurikulární dokumenty jsou v obou srovnávaných zemích pojímány odlišně. Ve Francii je kurikulum předškolní výchovy zahrnuto v jednom dokumentu s kurikulem školy základní (vysvětleno v kapitole \ref{frkurikulum}). V České republice je naopak předškolnímu vzdělávání věnován samostatný dokument, který se zabývá širokým spektrem otázek ohledně vzdělávání nejmenších dětí. Pro tuto práci byly vybrány srovnatelné parametry, které se nacházejí v obou dokumentech. Za účelem jejich porovnání je v této práci brán pojem kurikulum jakožto vzdělávací program či obsah vzdělávání a pozornost je věnována \textbf{vzdělávacím oblastem předškolního vzdělávání} a \textbf{kompetencím}, které by děti měly zvládat na konci mateřské školy. Pro jasnější představu o dokumentech obou zemí je na začátku každé kapitoly zmíněno i legislativní pojetí kurikula, jeho dostupnost a stručný přehled, jak celý dokument vypadá. 

	\section{Francouzské kurikulum}
	\label{frkurikulum}

		Mateřská škola ve Francii je součástí preprimárního vzdělávání, odpovídající úrovni ISCED 0. Jak je uvedeno v kapitole \ref{msvefr}, vzdělávání dětí předškolního a školního věku je řazeno do tří cyklů. Na půdě mateřské školy se odehrávají první dva cykly. Cílem preprimárního vzdělávání je připravit dítě na vstup do povinného vzdělávání na základní škole a  zaručit plynulý přechod podle individiuálních schopností každého dítěte. Francouzské kurikulum předškolního vzdělávání je tedy součástí stejného dokumentu jako kurikulum školy základní. 

		Legislativně je zakotveno ve školském zákoně (Code de l´éducation) č. 2003-339, který vešel v platnost 14. června 2003.
	
		Francouzské kurikulum je veřejně dostupný dokument, který je vydáván v podobě bulletinu Ministerstvem školství a výzkumu \citep{buletin}.

		Vychází též v knižní verzi od CNDP (Centre National de Documentation Pédagogique)\citep{CNDP}.

		Část dokumentu týkající se předškolního vzdělávání, která je v této práci předložena, nebyla ještě z francouzského jazyka přeložena a otvírá novou oblast pro českého čtenáře. Níže je prezentován vlastní překlad autorky, nejedná se o překlad oficiální. 

		Francouzský kurikulární dokument obsahuje 11 kapitol. 

	\begin{enumerate}[1]
		\setlength\itemsep{-2mm}
		\item Dopis od bývalého ministra školství Xaviera Darcose k novým programům primárního vzdělávání \textit{(Lettre de Xavier Darcos sur le noueaux programmes pour l´école primaire.)}
		\item Hodinová dotace v mateřské a základní škole \textit{(Horaires des écoles maternelles et élémentaires)}
		\item Program primárního vzdělávání \textit({Programmes d´enseignement  de l´école primaire)}
		\item Preambule \textit{(Préambule)}
		\item Prezentace \textit{(Présentation)}
		\item Program mateřské školy: nižší stupeň, střední stupeň, vyšší stupeň \textit{(Programme de l´école maternelle:petite section, moyenne section, grande section)}
		\item Cyklus základního vzdělávání - Program CP a CE1 \textit{(Cycle des apprentissages fondamentaux – Pogramme du CP et du CE1)}
		\item Cyklus prohlubování vzdělání – Program CE2, 1 a CM2 \textit{(Cycle des approfondissements – Programme du CE2, du CM1 et du CM2)}
		\item Kritéria organizace progresivity vzdělávání v mateřské škole \textit{(Repères pour organiser la progressivité des apprentissages à l´école maternelle)}
		\item Cyklus základního vzdělávání – Postup pro CP a CE1 \textit{(Cycle des apprentissage fondamentaux – Progression pour le cours préparatoire et le cours élémentaire)}
		\item Cyklus prohlubování – Postup pro CE2, CM1 a CM2 \textit{(Cycle des approfondissements pour le cours élémentaire dèuxieme année et le cours moyen)}
	\end{enumerate}


	Mateřskou školou a programem vzdělávání se zabývá šestá kapitola.

	Program pro mateřské školy je rozdělen do 6ti hlavních vzdělávacích oblastí, které jsou dále členěny do několika dílčích oblastí.
	Na konci každé vzdělávací oblasti jsou uvedeny znalosti a dovednosti, které by děti měly ovládat na konci mateřské školy. Jde o základy kompetencí, ke kterým je dítě vedeno, ale nejsou překážkou v postupu do dalšího ročníku/cyklu.


	\begin{enumerate}[1]
		\setlength\itemsep{-2mm}
		\item Osvojit si řeč 
		\item Objevovat písmo 
		\item Stát se žákem 
		\item Jednat a vyjadřovat se vlastním tělem 
		\item Objevovat svět
		\item Vnímat, cítit, představovat si, tvořit
	\end{enumerate}

		\subsection{Osvojit si řeč (S'approprier le langage)}
			Mluvený jazyk je v mateřské škole základním pilířem učení. Vyjadřování a pochopení se dítě učí skrze jazyk. Učí se být pozorný ke zprávě, která je mu sdělena, pochopit ji a odpovědět na ni. V rámci komunikace s učitelem, kamarády, při společných i specificky zaměřených aktivitách se každodenně učí novým slovům. Postupně si osvojuje syntaxi francouzského jazyka. Společné aktivity obohacují slovní zásobu a způsoby užití jazyka (dotazování se, vyprávění, vysvětlování, přemýšlení).

			\paragraph{Komunikovat, vyjadřovat se (Échanger, s´exprimer)}

			Nejdříve se děti učí komunikovat prostřednictvím dospělého v situacích, které se ho přímo týkají: jejich vlastní potřeby, objevy, otázky; naslouchá a odpovídá na žádosti. S jistotou pojmenovává objekty okolo sebe. Účastní se komunikace ve skupině, čeká, až na něj přijde řada a možnost vyjádřit se a respektuje dané téma. Dokáže nazpaměť převyprávět naučené říkanky a písně. Pozvolna rozšiřuje časovou osu, mluví o tom, co se bude dít, co zažilo, dokáže vymýšlet příběhy a převyprávět základní fakta problému. Postupně získává potřebné základy jazyka k vyjádření se: popis osob a vztahů mezi nimi, užití správného časování sloves a jak vhodným způsobem popsat dění v příběhu.

			\paragraph{Porozumět (Comprendre)}
			Více než vyjadřování je základní kapacitou dítěte porozumění, na které se dává v tomto věku velký důraz.
			Děti se učí rozlišovat otázku, slib, příkaz, odmítnutí, vysvětlení, vyprávění. Rozumí příkazům vyučujícího a všem termínům, které k tomu patří. Jsou vedené k pochopení kamaráda či dospělého, který hovoří o věcech pro dítě neznámých. Opakováním příběhů a pohádek, jak klasické tak moderní literatury, které jsou přiměřené věku dětí, je dětem umožněno porozumět složitějším a delším vyprávěním, která dokážou též převyprávět. 

			\paragraph{Zdokonalovat se ve francouzském jazyce (Progresser vers la maîtrise de la langue française)}
			Užíváním jazyka a posloucháním čtených textů se dítě učí pravidlům větné skladby a pořadí slov ve francouzské větě. Na konci mateřské školy dítě ovládá všechny slovní druhy, tvoří celé věty i krátká vyprávění a vysvětlení. Každodenní čtení a vyprávění příběhů učitelem, která obsahují nová slova, nestačí k jejich zapamatování. Nabytí slovní zásoby vyžaduje specifický přístup, pravidelné aktivity na klasifikaci a memorizování slov, opakované používání slov již naučených a vysvětlování neznámých termínů v jejich kontextu. Učitel dohlíží, aby se každý týden děti naučily nová slova k obohacení slovní zásoby. Děti se učí nejen slovíčka, která jim pomáhají k pochopení slyšeného textu, ale také slova k efektivní komunikaci ve škole a co nejpřesnějšímu vyjádření vlastních myšlenek. Vyučující věnuje každému dítěti dostatek pozornosti, pomáhá mu se správnými slovy a podporuje ho. Přeformuluje pokus dítěte, aby slyšelo, jak zní správný model. Aby se děti mohly zdokonalovat v mluveném projevu, je jim vyučující příkladem správnosti vět a přesné slovní zásoby.

			\paragraph{Znalosti a dovednosti, které by děti měly ovládat na konci MŠ}
			\begin{itemize}
				\setlength\itemsep{-2mm}
				\item[-] porozumět zprávě a reagovat nebo odpovědět na ni vhodným způsobem
				\item[-] s jistotou popsat objekt, osobu nebo událost z běžného života
				\item[-] srozumitelně vyjádřit otázku nebo popis
				\item[-] srozumitelně vyprávět pro posluchače neznámý příběh nebo příběh vymyšlený 
				\item[-] iniciativně se ptát na otázky a vyjadřovat svůj vlastní názor
			\end{itemize}


		\subsection{Objevovat písmo (Découvrir l'écrit)}
			Mateřská škola připravuje děti na základní vzdělávání. Činnosti spojené s mluveným projevem, navyšování slovní zásoby, písemná tvorba a četné poslechy vyprávěného a čteného textu učí žáky dovednostem čtení a psaní. Tři klíčové aktivity (cvičení na zvukovou stránku slov \uv{fonémy}, na základy abecedy a grafomotoriku) v mateřské škole významně podporují systematickou přípravu na čtení a psaní, která začíná v přípravném vzdělávání (CP-cours préparatoir).

			\paragraph*{I Seznámit se s psaním (Se familiariser avec l´écrit)}
				\subparagraph{Objevování psaných podkladů (Découvrir les supports de l´écrit)}
					Děti objevují užívání písemného projevu ve společnosti srovnáváním psaných podkladů, ve škole i mimo ni (plakáty, knihy, noviny, časopisy…). Učí se ho přesně popsat a pochopit jeho funkci. Objevují a používají knihy, učí se orientovat na stránce i na přebalu knih. 
				\subparagraph{Objevování psaného jazyka (Découvrir la langue écrite)}
					Díky čteným textům se děti každý den seznamují s psanou francouzštinou. Aby mohly vnímat specifika psaného projevu, jsou vybírány texty jazykově kvalitní z různých literárních žánrů (pohádky, legendy, bajky, básně, říkanky). Po celou dobu mateřské školy jsou děti vedeny k vyprávění a osvojování si děl z literárního dědictví. Stávají se citlivější ke způsobům, jak vyjádřit méně známé skutečnosti. Jejich zvědavost je stimulována otázkami vyučujícího, který zdůrazňuje nová slova a slovní obraty, které poté používá i v jiných situacích.  Děti vyprávějí přečtený příběh, sdělují, co pochopily, a doptávají se na nejasnosti. Jsou povzbuzovány k memorizování vět nebo krátkých úryvků textu. 
				\subparagraph{Základy psaní textu (Contribuer à l´écriture de textes)}
					Děti se účastní činností, jež přirozeně zanechávají stopu toho, co se stalo, bylo pozorováno nebo naučeno. Učí se diktovat text dospělému. Ten jim případnými otázkami pomáhá uvědomit si požadavky formy vyřčeného. Jsou též vedeny ke správnému výběru slov a syntaktické struktuře. Na konci mateřské školy dovedou děti transformovat spontánní mluvený projev na text, který dokáže dospělý zapsat podle diktátu.

				\subparagraph{Znalosti a dovednosti, které by děti měly ovládat na konci MŠ \hspace{3cm}}

				\begin{itemize}
					\setlength\itemsep{-2mm}
					\item[-] rozlišovat zvuky
					\item[-] poznat slabiky vyřčeného slova, poznat stejnou slabiku v různých slovech
					\item[-] rozeznat a napsat většinu písmen abecedy
					\item[-] spojit hlásku s písmenem
					\item[-] pod vedením učitele napsat krátká jednoduchá slova z hlásek a písmen, které děti již znají 	
					\item[-] napsat své jméno
				\end{itemize}


			\paragraph*{II Připravit se na čtení a psaní (Se préparer à apprendre à lire et à écrire)}
				\subparagraph{Rozeznávání zvuku slov (Distinguer les sons de la parole)}
					Děti velmi brzo objevují radost ze hry se slovy a zvukomalebností jazyka. Nejdřív slabiky pokřikují, později si s nimi hrají (vynechávají slabiky, kombinují několik slabik dohromady v různém pořadí). Dokážou rozeznat stejné slabiky v jiných slovech a určit jejich pozici ve slově (na začátku, uprostřed, na konci)
					Postupně rozlišují zvuky hlásek a učí se operovat s nimi a s dalšími jazykovými komponenty. Vyučující je pozorný k pokroku při osvojování si abstraktních hlasových aktivit.
				\subparagraph{Základy abecedy (Aborder le principe alphabétique)}
					Děti se seznamují se základy rozdílnosti mezi mluveným a psaným projevem francouzského jazyka. Pozorováním známých věcí (datum, název příběhu nebo pohádky) nebo krátkých vět děti porozumí posloupnosti slov a faktu, že každé napsané slovo odpovídá slovu mluvenému. 
					Objevují, že každé slovo, které řeknou nebo které slyší, je složeno ze slabik a dávají si do spojitosti písmena a hlásky. Rozlišování hlásek je čím dál tím přesnější. Postupně se učí názvy všech písmen abecedy, která umí rozpoznat tiskace i psace, i přesto, že klasické pořadí písmen v abecedě jim ještě zůstává neznámé. U některých písmen si k jejich názvu asociují zvuk hlásky (pozn. autorky – názvy písmen abecedy ve francouzském jazyce jsou odlišné od hlásek daných písmen ve slově). Tímto si osvojují principy abecedy.
				\subparagraph{Základy grafomotoriky (Apprendre les gestes de l´écriture)} 
					Každý den děti pozorují a reprodukují grafické motivy. Tím se učí nejefektivnějším gestům (pohybům). Vstup do světa psaní záleží na rozvinuté grafomotorice (spojování jednoduchých linií, vln apod.), ale vyžaduje i zvláštní dovednosti vnímat charakteristiky písmen. 
					Dětem ve vyšším stupni (Grande section), které na to již mají kapacity, je předkládáno i písmo psané. Jde o řízenou aktivitu pod dohledem vyučujícího. První zvyky, které si dítě osvojí, mají vliv na pozdější kvalitu zápisu a uvolněnost ruky při psaní. 
					
					\subparagraph{Znalosti a dovednosti, které by děti měly ovládat na konci MŠ}
					\begin{itemize}
						\setlength\itemsep{-2mm}
						\item[-] rozpoznat hlásky
						\item[-] rozlišovat slabiky vyřčených slov, poznat stejné slabiky v různých slovech
						\item[-] z krátkého výroku pospojovat mluvená slova s napsanými
						\item[-] rozpoznat a napsat většinu písmen abecedy
						\item[-] pojit hlásku s písmenem
						\item[-] pod vedením vyučujícího kopírovat psaným písmem krátká a jednoduchá slova, která dítě již zná
						\item[-] napsat psacím písmem své jméno
					\end{itemize}


		\subsection{Stát se žákem (Devenir élève)}
			Cílem je dítě naučit, co ho odlišuje od ostatních a vnímat sám sebe jako osobnost. Naučit ho žít v organizované společnosti s pravidly. Chápat, co je to škola a jaké je v ní jeho místo. Stát se žákem je postupný proces, který od vyučujícího vyžaduje jak flexibilitu, tak důslednost.
			\paragraph{Život ve společnosti: jak se naučit pravidla společnosti a základy slušného chování (Vivre ensemble: apprendre les règles de civilité et les principes d´un comportement coforme à la morale)}
				Děti objevují bohatství i nátlak skupiny, do které jsou začleňovány. Pociťují radost z přijetí, a že jsou poznány a postupně přijímají i ostatní kamarády. Kolektiv, ve kterém se děti v mateřské škole nacházejí, je vhodnou situací, kde se učí vést dialog mezi sebou, s dospělými a učí se, kdy na ně přijde řada. Je to výborná příležitost k nácviku pravidel slušného chování, jako je pozdravení na začátku a na konci dne, odpovědět na otázku, poděkovat osobě, která nám pomohla a nepřerušovat ostatní. Zvláštní důraz je kladen na morální základy pravidel chování, jako je respektování ostatních osob a dobro bližních, povinnost přizpůsobit se pravidlům daných dospělými i respektovat, že mluví druhé dítě. 

			\paragraph{Spolupráce a samostatnost (Coopérer et devenir autonome)}
				Účast ve hrách, kruzích či vytvořených skupinkách, které mají recitovat říkanku nebo poslouchat příběh, účast na realizaci společných projektů apod., jsou aktivity, díky kterým děti přicházejí na chuť společným aktivitám a učí se spolupracovat. Zajímají se o ostatní a spolupracují s nimi. Přebírají zodpovědnost ve třídě a jsou iniciativní. Angažují se v projektech nebo činnostech s důrazem na své vlastní zdroje. Zakoušejí samostatnost, úsilí a vytrvalost. Chápou, co to je škola.
				Děti se mají postupně naučit, jaká jsou pravidla školní komunity, specifika školy, co se ve škole dělá, co se od nich očekává a co a proč se ve škole učí. Rozlišují rozdílné role rodičů a učitelů. 
				Postupně přijímají rytmus společných činností a dokážou odlišit uspokojení z jejich vlastních zájmů. Chápou hodnotu společných pravidel. Učí se, jak klást otázky a jak vyjednávat, aby dosáhly zadaného. Rozvíjejí si souvislosti mezi materiálními činnostmi a tím, co se danou aktivitou učí. Získávají objektivní kritéria pro evaluaci vlastních úspěchů. Na konci mateřské školy umí poznat vlastní chyby i chyby kamarádů. Učí se vydržet soustředit čím dál tím delší dobu. Objevují spojení mezi tím, co se učí a věcmi v běžném životě.
			\paragraph{Znalosti a dovednosti, které by děti měly ovládat na konci MŠ}
			\begin{itemize}
				\setlength\itemsep{-2mm}
				\item[-] respektovat ostatní a respektovat pravidla společného života
				\item[-] poslouchat, pomáhat, spolupracovat, žádat o pomoc
				\item[-] důvěřovat sám sobě, kontrolovat vlastní emoce
				\item[-] identifikovat dospělé a jejich role
				\item[-] být samostatný v jednoduchých úkonech a hrát roli ve školních činnostech
				\item[-] říct, co se naučilo
				\end{itemize}

		\subsection{Jednat a vyjadřovat se vlastním tělem (Agir et s'exprimer avec son corps)}

			Fyzická aktivita a zkušenosti s vlastním tělem přispívají k rozvoji motoriky, smyslů, citů a intelektu dítěte. Jsou příležitostí objevovat, vyjádřit se, jednat ve známém prostředí, později i prostředí známém méně a dovolují orientovat se v prostoru. Dítě objevuje možnosti svého těla. V bezpečí se učí reagovat a přijímat možná rizika a využití adekvátního množství energie pro danou činnost. Vyjadřuje, co cítí, dokáže popsat činnosti a objekty, s kterými pracuje a používá je. Vyjadřuje, co má chuť dělat. Vyučující zaručuje  pestrou nabídku činností, zvyšování jejich náročnosti a dostatek možností ke sebezdokonalování. Také jim pomáhá vytvářet důvěru sám v sebe a v nově nabytých dovednostech. 
			\textbf{Skrze fyzickou aktivitu řízenou nebo volnou} v různých prostředích rozvíjejí děti své motorické dovednosti, rovnováhu, manipulaci, házení a chytání. Hry s míčem, hry s protivníkem, hry s pravidly doplňují tyto aktivity. Děti řídí dané aktivity i jejich návaznost. Osvojují si motorické dovednosti, kdy je správně používat a jejich správné provedení. 
			\textbf{Skrze činnosti s pravidly} rozvíjejí dovednosti adaptace a spolupráce. Díky nim chápou a přijímají pozitiva a negativa činností v kolektivu. 
			\textbf{Umělecké činnosti} jako kruh, tanec, pantomima pomáhají k vyjádření se gesty a k rozvoji představivosti.
			Pomocí rozličných činností nabývají děti \textbf{obraz vlastního těla}. Rozlišují před, za, nahoře, dole, vpravo, vlevo, blízko a daleko. Zvládají překážkové dráhy a dokážou popsat své pohyby a jejich provedení.
				\paragraph{Znalosti a dovednosti, které by děti měly ovládat na konci MŠ}

				\begin{itemize}
					\setlength\itemsep{-2mm}
					\item[-] přizpůsobit své pohyby v různých prostředích a omezeních
					\item[-] individuálně či společně pracovat nebo být proti
					\item[-] vyjádřit se hudebním rytmem, nástrojem, vyjádřit své pocity a emoce gesty a pohyby
					\item[-] orientovat se a pohybovat se v prostoru
					\item[-] popsat a vytvořit jednoduchou dráhu
				\end{itemize}

		\subsection{Objevovat svět (Découvrir le monde)}
			V mateřské škole dítě objevuje svět okolo něj. Učí se prostorovým a časovým omezením a jak se jim přizpůsobit. Pozoruje, klade otázky a dělá pokroky v účelnosti dotazování. Učí se přijímat jiný pohled na věc než svůj vlastní.  Konfrontace a logické myšlenky mu dodávají chuť uvažovat nad věcmi. Naučí se počítat, třídit, uspořádávat a popsat věci jak jazykem, tak různými formami vyjadřování (obrázky, načrty). Začíná chápat rozdíly mezi živými a neživými objekty.

			\paragraph{Objevování objektů (Découvrir les objets)}
				Děti objevují běžné technické předměty (baterka, telefon, počítač…), chápou jejich využití a funkce, k čemu slouží a jak je používat. Naučí se i znaky nebezpečných předmětů.
				Vyrábějí předměty použitím různých materiálů a vybírají si vhodné nástroje a techniky (stříhání, lepení, ohýbání, skládání, připíchnutí, složení a rozložení…).
			\paragraph{Objevování materiálu (Découvrir la matière)}
				Znalosti o charakteristických vlastnostech materiálů získávají díky činnostem, jako je stříhání, modelování, spojování běžných materiálů (dřevo, půda, papír, karton, voda, atd.).
				Uvědomují si i méně viditelnou realitu jako např. vítr, a začínají vnímat změny skupenství vody. 
			\paragraph{Objevování živého (Découvrir le vivant)} 
				Děti objevují různé projevy živé přírody. Chov dobytka a hospodářství jsou významným prostředkem k objevování cyklu života od narození přes růst, reprodukci, po stárnutí až smrt.
				Objevují části těla a pět smyslů, jejich charakteristiku a funkci. Zajímají se o hygienu a zdraví a hlavně o stravu. Učí se základním pravidlům hygieny těla. 
				Jsou vnímavé k problémům životního prostředí a učí se respektovat život. 
			\paragraph{Objevování formy a rozměrů (Découvrir les formes et les grandeurs)}
				Manipulací s různými předměty děti odhalují nejdříve jednoduché vlastnosti (malý/velký; těžký/lehký) a poté dokážou rozlišovat základní kritéria, srovnávat a třídit podle tvaru, velikosti, množství a obsahu.
			\paragraph{Přibližování se veličinám a číslům (Découvrir les quantités et les nombres)}
				Mateřská škola je klíčovým obdobím k získání povědomí o posloupnosti čísel a jejich využití v určování množství. Děti objevují a učí se chápat funkce čísel, zvláště jako prostředek k vyjádření množství a označení pořadí objektů v řadě.
				Činnosti jako rozdělování, srovnávání a třídění ovlivňují přístup dětí k vnímání celku. Děti se postupně učí počítat nejméně do třiceti a základům jednoduchých počtů.
				Čísla jsou používána v situacích, které dávají dětem smysl a jsou praktickým prostředkem k dosažení cíle: hry, činnosti ve třídě, zadané úkoly na srovnávání, spojování, řazení a rozdělování. Velikost celku a možnost reagovat na předměty jsou důležité proměnné, které vyučující přizpůsobuje kapacitám každého dítěte. 
				Konec mateřské školy je časem prvních kroků do světa počtů. 
				Tím je psaní číslic v konkrétních situacích (př. kalendář) nebo při hrách (přemísťování se po značkách s číslicemi). Děti si vytvářejí první spojení mezi ústním označením a psaným označením číslic. Jejich výkon zůstává velmi rozdílný, ale je důležité, aby se začali této dovednosti učit. K výuce psaní číslic se přistupuje stejně důkladně jako v případě psaní písmen.
			\paragraph{Orientace v čase (Se repérer dans le temps)}
				Pravidelnou organizací rozvrhu děti pozvolna vnímají časovou posloupnost dnů, týdnů, měsíců. Na konci mateřské školy chápou cykličnost určitých fenoménů (roční období) a znázornění času (týden, měsíc). Učení se pojmu časová posloupnost je utvrzováno v činnostech i známých příbězích. Grafické znázornění napomáhá k jejich utvrzení (obrázky, kresby).
				V nižším stupni (Petit section) používají děti k orientaci v chronologii a měření času kalendáře, hodiny a přesýpací hodiny. V přípravné třídě se tyto limitované znalosti prohlubují. Popisováním příběhů, které se již staly, nebo pozorováním rodinného odkazu, se děti učí blízké minulosti a s většími obtížemi i vzdálené budoucnosti.
				Tyto činnosti dávají prostor k učení se přesné slovní zásoby, která je opakovaným používáním, zvláště při rituálech, fixována.
			\paragraph{Orientace v prostoru (Se repérer dans l´espace)}
				Po celou dobu mateřské školy se děti učí pohybovat se po prostorách a nejbližším okolí školy. Dokážou si najít své místo ve vztahu k věcem a ostatním osobám a lokalizovat věci a osoby v prostoru, což předpokládá schopnost oprostit se od svého vlastního pohledu na věc. Na konci mateřské školy rozlišují pravou a levou stranu. Děti zvládnou projít trasu podle značek a povelů (příkazy a grafické znázornění).
				Činnosti, při kterých děti musí přecházet z vertikálního plánu do horizontálního, a naopak, a udržovat relativní postavení předmětů nebo znázorněné prvky, jsou předmětem mimořádné pozornosti. Připravují je na orientaci v grafickém prostoru. Orientace v prostoru na stránce nebo na papíře a orientace na lince je spojena s dovednostmi čtení a psaní. 
			\paragraph{Znalosti a dovednosti, které by děti měly ovládat na konci MŠ}
				\begin{itemize}
					\setlength\itemsep{-2mm}
					\item[-] rozeznat, vyjmenovat, popsat, porovnat, uspořádat, třídit materiál a předměty podle jejich kvalit a užití
					\item[-] znát projevy života zvířat i rostlin, spojit je s vyšší funkcí: narození, výživa, pohyb, reprodukce
					\item[-] vyjmenovat hlavní části těla a jejich funkce, rozlišit pět smyslů a k čemu slouží
					\item[-] umět a aktivně užívat pravidla hygieny těla, prostředí, stravování
					\item[-] rozpoznat a uvědomovat si nebezpečí
					\item[-] orientovat se ve dnech, týdnech, měsících
					\item[-] dokázat určit událost ve vztahu k ostatním událostem
					\item[-] nakreslit kruh, čtverec, trojúhelník
					\item[-] porovnat počet, vyřešit početní úkol
					\item[-] umět nazpaměť, jak jdou číslice do třiceti za sebou 
					\item[-] slovně vyjádřit  množství v číslicích 
					\item[-] orientovat se v prostoru a oproti ostatním předmětům
					\item[-] orientovat se na stránce
					\item[-] užívat správná slovní vyjádření při popisu vztahu času a prostoru
				\end{itemize}

		\subsection{Vnímat, cítit, představovat si, tvořit (Percevoir, sentir, imaginer, créer)}
			Mateřská škola nabízí první setkání s citem pro umění. Vizuální, hmatové, sluchové a hlasové činnosti zvyšují smyslové schopnosti dětí. Pobízí jeho představivost a obohacují jeho vyjadřovací znalosti a kapacity. Přispívají k rozvoji pozornosti a koncentrace. Poslech a pozorování jsou příležitostmi seznámit dítě s formami uměleckého vyjádření. Poznávají své emoce a vstupují do uměleckého světa.
			Tyto činnosti souvisejí též s ostatními obory vzdělávání, podporují zvídavost k objevování světa, dovolují dětem vyjadřovat se pohybem, podporují vyjádření vlastních reakcí a chutí a dávají možnost výběru v rámci interakce s ostatními.
			\textbf{Výkresy a prostorové kompozice (výroba předmětů) jsou oblíbenými prostředky vyjadřování.}
			Děti experimentují s mnoha nástroji, které jim pomáhají tvořit výkresy. Objevují, používají předměty různé povahy a realizují obrazy. Tvoří předměty využitím malby, lepením papíru, koláží, asambláží, modelováním apod.
			V tomto kontextu vyučující pomáhá dětem vyjádřit to, co vnímají, a podporuje vlastní zahájení projektů a jejich realizaci. Přitom je učí používání správných slov. Povzbuzuje děti, aby si založily osobní sbírku předmětů s estetickou a citovou hodnotou.
			\textbf{Hlas a sluch} jsou prostředky komunikace, které děti velmi brzo objevují při hře se zvuky, při zpěvu a při pohybu.
			Repertoár říkanek a písní vycházející z ústní tradice, do které spadají i moderní autoři, se každým rokem obohacuje. Děti zpívají pro radost a hrají si s hlasem, hlukem a rytmy.
			Strukturované poslechové činnosti bystří pozornost, rozvíjejí citlivost rozlišovat zvuky a zvukovou paměť. Děti poslouchají pro zábavu, aby mohly napodobovat, kvůli pohybu a v rámci hry. Učí se rozlišovat barvu, intenzitu, dobu, výšku tónu. Srovnávají, napodobují a určují jejich znaky. Poslouchají různá hudební díla. Hledají nové možnosti užití hudebních nástrojů. Po částech se učí rytmus a tempo. 

			\paragraph{Znalosti a dovednosti, které by děti měly ovládat na konci MŠ}
				\begin{itemize}
					\setlength\itemsep{-2mm}
					\item[-] přizpůsobit svou činnost omezení materiálu (nástroje, prostředky, materiál)
					\item[-] používat obrázky jako prostředek sebevyjádření
					\item[-] realizovat výtvor podle svých představ (plán, velikost)
					\item[-] pozorovat a popsat umělecká díla a vytvářet si vlastní sbírku
					\item[-] zapamatovat si a interpretovat písně a říkanky
					\item[-] poslouchat část z hudebního díla a poté se k němu vyjádřit a hovořit o svých dojmech
				\end{itemize}



\subsection{Shrnutí cílů a pojetí předškolního vzdělávání a role pedagoga}

	Francouzský dokument neuvádí žádný z těchto bodů jako samostatnou kapitolu. V úvodu pro všechny cykly vzdělávání je popsána role pedagoga, kterému je formálně ponechána svoboda ve volbě vzdělávacích metod, tak aby byl zajištěn rozvoj dítěte s ohledem na cíle vzdělávacího programu.

	Pedagogická svoboda učitelů jde ruku v ruce s novými způsoby hodnocení žáků, která jsou více zaměřena na evaluaci nabytých vědomostí. Tato nová koncepce pedagogické profese číní učitele  plně zodpovědné za volbu metod a strategie, které mají pomoci dětem se vzdělávat. Aby pedagog dobře plnil tuto roli, musí mít výbornou znalost cílů a obsahu vzdělávání.

	Pojetí a cíle jsou popsány v úvodu programu pro mateřské školy. Níže je uvedeno jejich shrnutí.

	Mateřská škola má za cíl adekvátními postupy pomoci každému dítěti stát se soběstačným a osvojit si znalosti a kompetence potřebné k úspěšnému zvládnutí přípravné třídy (Cours préparatoire) základního vzdělávání. Základním cílem jsou bohaté a uspořádané znalosti mluveného jazyka srozumitelného pro ostatní. V mateřské škole dítě navazuje nové vztahy s dětmi i dospělými. Procvičuje své pohybové, smyslové, citové, vztahové a intelektuální schopnosti a dovednosti. Postupně se z něj stává žák. Objevuje svět psaného jazyka. 

	Mateřská škola otevírá dítěti svět vztahů mezi lidmi a dovoluje mu prožít si takové situace, jako jsou hry, experimenty, vedenou nebo volnou tvorbu, bohaté a rozmanité úkoly, které obohacují formování jeho osobnosti a kulturního probouzení. 

	Děti mají v mateřské škole dostatek času a prostoru zvykat si, pozorovat, napodobovat, zkoušet, hledat a pracovat, aniž by byly ohroženy ztrátou zájmu. Mateřská škola podporuje v dítěti touhu učit se, nabízí bohaté a různorodé zkušenosti a obohacuje jejich pochopení. 
	Činnosti v mateřské škole musí nabízet širokou nabídku smyslových a pohybových zkušeností. Organizace dne respektuje biologický rytmus a potřeby dětí dle jejich věku. 

	Školní projekt je prostředkem zaručujícím potřebnou návaznost v přechodu ze školy mateřské na školu základní, kde \textit{grande section} je sice poslední třídou mateřské školy, ale zároveň otvírá druhý program, tj. cyklus základního vzdělávání. Stanovuje cíl, který je uzavřen a dosažen až na úrovni 2. ročníku primární školy. Akceptuje přirozená vývojová specifika dětí. Mateřská škola hraje klíčovou roli v diagnóze a prevenci deficitů a poruch komunikačních schopností.

	\section{České kurikulum}

		K dnešnímu dni je obsah vzdělávání v České republice řešen na dvou úrovních, na úrovni státní a na úrovni školské. Toto je vyžadováno novým Školským zákonem č. 561/2004 o předškolním, základním, středním, vyšším odborném a jiném vzdělávání, který vešel v platnost 1. ledna 2005 a je výsledkem probíhající školské reformy a Národního programu pro rozvoj vzdělávání v České republice (Bílá kniha), který formuje vládní strategii v oblasti vzdělávání, odráží celospolečenské zájmy a dává konkrétní podněty k práci škol.

		Na školské úrovni si každá mateřská škola vytváří svůj vlastní školní vzdělávací plán, který vychází ze základů Rámcového vzdělávacího programu a principů v něm stanovených. Rámcový vzdělávací plán je zpracováván na úrovni státní a je dokumentem, jenž určuje principy a směr, kudy by se předškolní vzdělávání mělo ubírat. Rámcový vzdělávací program předškolního vzdělávání je závazně platný od 1. 9. 2005.

		Rámcové i školní vzdělávací programy jsou veřejné dokumenty dostupné pro celou veřejnost. Rámcový vzdělávací program pro předškolní vzdělávání je ke stažení na stránkách Národního ústavu pro vzdělávání \citep{RVP} nebo na stránkách Ministerstva školství, mládeže a tělovýchovy.

		
		\begin{figure}[h!t]
			\center
			\includegraphics[width=0.8\linewidth]{fotky/rvpCR.jpg}
			\caption{\textbf{Rámcový vzdělávací program České republiky.}
				Schéma znázorňuje státní a školní úroveň zpracovávání vzdělávacích programů.
			}
			\label{obr:rvpCR}
		\end{figure}


		Rámcový vzdělávací program pro předškolní vzdělávání je rozdělen do 12 kapitol.
		\begin{enumerate}[1.]
			\setlength\itemsep{-2mm}

			\item \textit{\uv{Vymezení Rámcového vzdělávacího programu pro předškolní vzdělávání v systému kurikulárních dokumentů
			\item Předškolní vzdělávání v systému vzdělávání a jeho organizace
			\item Pojetí a cíle předškolního vzdělávání
			\item Vzdělávací obsah RVP PV
			\item Vzdělávací oblasti
			\item Vzdělávací obsah ve školním vzdělávacím programu 
			\item Podmínky předškolního vzdělávání
			\item Vzdělávání dětí se speciálními vzdělávacími potřebami a dětí mimořádně nadaných
			\item Autoevaluace mateřské školy a hodnocení dětí
			\item Zásady pro zpracování školního vzdělávacího programu
			\item Kriteria souladu rámcového a školního vzdělávacího programu
			\item Povinnost předškolního pedagoga}} \citep[s.~2]{RVP}
		\end{enumerate}

		Pro účely této práce a možnosti komparace s francouzským kurikulem se budu podrobněji zabývat jen některými částmi RVP PV. A to kapitolou 3 Pojetí a cíle předškolního vzdělávání a kapitolou 5 Vzdělávací oblasti.
		Obsah RVP je udáván pouze obecně a rámcově, slouží jako prostředek k naplňování
		vzdělávacích záměrů a dosahování vzdělávacích cílů. Vzdělávací obsah RVP PV je rozdělen do pěti oblastí:

		\begin{enumerate}[1.]
			\setlength\itemsep{-2mm}
			\item Dítě a jeho tělo
			\item Dítě a jeho psychika
			\item Dítě a ten druhý
			\item Dítě a společnost
			\item Dítě a svět
		\end{enumerate}

		Každá oblast obsahuje 4 části. Těmi jsou dílčí vzdělávací cíle (co pedagog u dítěte podporuje), vzdělávací nabídka (co pedagog dítěti nabízí), očekávané výstupy (co dítě na konci předškolního období zpravidla dokáže) a rizika (co ohrožuje úspěch vzdělávacích záměrů pedagoga).

		Mají-li se srovnávat vzdělávací oblasti a jejich cíle u obou dokumentů, je potřeba v další kapitole uvést obsahy vzdělávácích oblastí rámcového vzdělávacího programu pro předškolní vzdělávání České republiky. Z nich je níže uveden stručný výpis nejdůležitějších bodů. 

			\subsection{Dítě a jeho tělo}
				\textit{\uv{Záměrem vzdělávacího úsilí pedagoga v oblasti biologické je stimulovat a podporovat růst a neurosvalový vývoj dítěte, podporovat jeho fyzickou pohodu, zlepšovat jeho tělesnou zdatnost i pohybovou a zdravotní kulturu, podporovat rozvoj jeho pohybových a manipulačních dovedností, učit je sebeobslužným dovednostem a vést je k zdravým životním návykům a postojům.}} \citep[s.~16]{RVP}

				\paragraph{Dílčí vzdělávací cíle} 

				\begin{itemize}
				\setlength\itemsep{-2mm}
					\item[-]rozvoj jemné a hrubé motoriky
					\item[-]uvědomění si a ovládání vlastního těla
					\item[-]rozvoj všech smyslů
					\item[-]rozvoj psychické a fyzické zdatnosti
					\item[-]základní poznatky o zdraví a zdravém životním stylu
				\end{itemize}

				\paragraph{Vzdělávací nabídka}

				\begin{itemize}
				\setlength\itemsep{-2mm}
					\item[-]lokomoční a manipulační hry
					\item[-]smyslové a psychomotorické hry
					\item[-]konstruktivní a grafické hry
					\item[-]hudební a hudebně pohybové hry
					\item[-]sebeosblužné činnosti
					\item[-]relaxační a odpočinkové činnosti
					\item[-]prevence úrazů, nemoci, závislostí
				\end{itemize}

				\paragraph{Očekávané výstupy}

				\begin{itemize}
				\setlength\itemsep{-2mm}
					\item[-]správné držení těla
					\item[-]zvládnutí základních pohybových dovedností a prostorové orientace
					\item[-]koordinace lokomoce, koordinace ruky a oka a jemné motoriky
					\item[-]napodobování pohybů podle vzoru
					\item[-]ovládání dechového svalstva
					\item[-]vnímat a rozlišovat všemi smysly
					\item[-]zvládnutá sebeobsluha a hygienické návyky
					\item[-]pojmenovat části těla a jejich funkce
					\item[-]povědomí o ochraně osobního zdraví a kde hledat pomoc
				\end{itemize}

			\subsection{Dítě a jeho psychika}
				\textit{\uv{Záměrem vzdělávacího úsilí pedagoga v oblasti psychologické je podporovat duševní pohodu, psychickou zdatnost a odolnost dítěte, rozvoj jeho intelektu, řeči a jazyka, poznávacích procesů a funkcí, jeho citů i vůle, stejně tak i jeho sebepojetí a sebenahlížení, jeho kreativity a sebevyjádření, stimulovat jeho osvojování a rozvoj jeho vzdělávacích dovedností a povzbuzovat je v dalším rozvoji a učení.
				Tato oblast zahrnuje tři „podoblasti“: Jazyk a řeč; Poznávací schopnosti a funkce, představivost a fantazie, myšlenkové operace; Sebepojetí, city a vůle.}} \citep[s.~18]{RVP}
				\paragraph{I Jazyk a řeč}
				 
					\subparagraph{Dílčí vzdělávací cíle}

					\begin{itemize}
					\setlength\itemsep{-2mm}
						\item[-]rozvoj řečových schopností a jazykových dovedností  (receptivních i produktivních)
						\item[-]rozvoj komunikativních dovedností
						\item[-]osvojení si dovedností předcházejících čtení a psaní
					\end{itemize}
					
					\subparagraph{Vzdělávací nabídka}

					\begin{itemize}
					\setlength\itemsep{-2mm}
						\item[-]artikulační, řečové, sluchové a rytmické hry
						\item[-]individuální a skupinová  konverzace
						\item[-]vyprávění, komentování zážitků, vyřizování vzkazů
						\item[-]poslech pohádek, filmové a divadelní příběhy
						\item[-]vyprávění, přednes, recitace, zpěv
						\item[-]grafické napodobování symbolů
						\item[-]poznávání a rozlišování zvuku a gest
						\item[-]seznámení se se sdělovacími prostředky
					\end{itemize}

					\subparagraph{Očekávané výstupy}

					\begin{itemize}
					\setlength\itemsep{-2mm}
						\item[-]správně vyslovovat, ovládat dech, tempo a intonaci řeči
						\item[-]vyjadřovat myšlenky, nápady, pocity
						\item[-]vést rozhovor, domluvit se slovy i gesty
						\item[-]porozumět slyšenému, formulovat otázky, odpovídat, slovně reagovat
						\item[-]umět krátké texty zpaměti, sledovat a vyprávět příběh, popsat situaci, chápat slovní humor
						\item[-]sluchově rozlišovat začáteční a koncové slabiky a hlásky ve slovech, utvořit jednoduchý rým
						\item[-]poznat některá písmena, číslice, své jméno
						\item[-]zájem o knížky, hudbu, film
					\end{itemize}
				
				\paragraph{II Poznávací schopnosti a funkce, představivost a fantazie, myšlenkové operace}
				\textcolor{white}{ } 

					\subparagraph{Dílčí vzdělávací cíle}

					\begin{itemize}
					\setlength\itemsep{-2mm}
						\item[-]rozvoj smyslového vnímání
						\item[-]rozvoj paměti, pozornosti, představivosti a fantazie
						\item[-]rozvoj tvořivosti
						\item[-]posilování poznávacích citů
						\item[-]rozvoj zájmu o učení
						\item[-]osvojení si elementárních poznatků o znakových systémech
						\item[-]základ práce s informacemi
					\end{itemize}
					
					\subparagraph{Vzdělávací nabídka}

					\begin{itemize}
					\setlength\itemsep{-2mm}
						\item[-]pozorování přírodních, kulturních a technických objektů a jevů
						\item[-]pojmenování jejich vlastností a charakteristických znaků
						\item[-]motivovaná manipulace s předměty
						\item[-]konkrétní manipulace s materiálem
						\item[-]smyslové hry
						\item[-]hry na rozvoj postřehu, vnímání, zrakové a sluchové paměti, pozornosti a různých druhů paměti
						\item[-]námětové hry, hry podporující tvořivost, představivost a fantazii
						\item[-]řešení myšlenkových i praktických problémů a hledaní řešení
						\item[-]činnosti k seznámení s matematickými pojmy a jejich symbolikou
						\item[-]činnosti zasvěcující do časových pojmů
					\end{itemize}

					\subparagraph{Očekávané výstupy}

					\begin{itemize}
					\setlength\itemsep{-2mm}
						\item[-]vědomě využívat všech smyslů
						\item[-]záměrně pozorovat, všímat si, soustředit se a udržet pozornost
						\item[-]poznat a pojmenovat většinu toho, co ho obklopuje
						\item[-]přemýšlet a vést jednoduché úvahy
						\item[-]naučit se nazpaměť krátké texty
						\item[-]postupovat a učit se podle instrukcí, využívat zkušenost k učení
						\item[-]chápat základní číselné a matematické pojmy, souvislosti a prakticky je používat
						\item[-]chápat prostorové pojmy, elementární časové pojmy
						\item[-]řešit problémy, myslet kreativně, nalézat nová řešení
						\item[-]vyjadřovat představivost v tvořivých činnostech
					\end{itemize}
									
				\paragraph{III Sebepojetí, city, vůle}
				
					\subparagraph{Dílčí vzdělávací cíle}

					\begin{itemize}
					\setlength\itemsep{-2mm}
						\item[-]poznávání sebe sama, rozvoj pozitivních citů k sobě
						\item[-]získání relativní citové samostatnosti
						\item[-]rozvoj schopnosti sebeovládání
						\item[-]vytváření osobních vazeb
						\item[-]rozvoj schopností vyjádřit prožitky a dojmy
						\item[-]rozvoj mravního i estetického vnímání
						\item[-]získání schopnosti záměrně řídit svoje chování a ovlivňovat vlastní situaci
					\end{itemize}

					\subparagraph{Vzdělávací nabídka}
					
					\begin{itemize}
					\setlength\itemsep{-2mm}
						\item[-]spontánní hra
						\item[-]činnosti vyvolávající spokojenost, veselí, pohodu
						\item[-]úkoly, v nichž může být dítě úspěšné
						\item[-]činnosti vyžadující samostatné vystupování, obhajování vlastních názorů, rozhodování, sebeohodnocení
						\item[-]hry pro rozvoj vůle a sebeovládání
						\item[-]cvičení organizačních dovedností
						\item[-]estetické a tvůrčí aktivity, cvičení v projevování citů, v sebekontrole a sebeovládání
						\item[-]výlety do okolí
						\item[-]činnosti k poznávání různých lidských vlastností
						\item[-]dramatické činnosti, mimické vyjadřování
						\item[-]činnosti vedoucí k vyjádření sebe sama a k odlišení od ostatních
					\end{itemize}
					
					\subparagraph{Očekávané výstupy}

					\begin{itemize}
					\setlength\itemsep{-2mm}
						\item[-]odloučit se na určitou dobu od rodičů, uvědomovat si svou samostatnost
						\item[-]zaujímat vlastní názory, rozhodovat o svých činnostech
						\item[-]vyjádřit souhlas i nesouhlas, uvědomovat si své možnosti i limity
						\item[-]přijímat pozitivní ocenění i případný neúspěch a vyrovnat se s ním
						\item[-]vyvinout volní úsilí, soustředit se na činnost a dokončit ji
						\item[-]zorganizovat hru a respektovat pravidla
						\item[-]rozlišovat citové projevy v různých prostředích, prožívat a projevovat, co cítí
						\item[-]snažit se ovládat afektivní chování
						\item[-]být citlivý k živým bytostem, přírodě i věcem
						\item[-]těšit se z příjemných zážitků
						\item[-]zachytit a vyjádřit své pocity
					\end{itemize}

			\subsection{Dítě a ten druhý}
				\textit{\uv{Záměrem vzdělávacího úsilí pedagoga v interpersonální oblasti je podporovat utváření vztahů dítěte k jinému dítěti či dospělému, posilovat, kultivovat a obohacovat jejich vzájemnou komunikaci a zajišťovat pohodu těchto vztahů.}} \citep[s.~24]{RVP}

					\paragraph{Dílčí vzdělávací cíle}

					\begin{itemize}
					\setlength\itemsep{-2mm}
						\item[-]seznamování se s pravidly chování k druhému
						\item[-]osvojení si schopností a dovedností pro navazování a rozvíjení vztahů
						\item[-]posilování prosociálního chování
						\item[-]vytváření prosociálních postojů
						\item[-]rozvoj komunikativních a kooperativních dovedností
						\item[-]ochrana osobního soukromí
					\end{itemize}
					
					\paragraph{Vzdělávací nabídka}

					\begin{itemize}
					\setlength\itemsep{-2mm}
						\item[-]běžné komunikační aktivity dítěte s druhými
						\item[-]sociální hry, hraní rolí, dramatické činnosti
						\item[-]hudební a hudebně pohybové hry
						\item[-]aktivity podporující uvědomování si vztahů mezi lidmi
						\item[-]činnosti na porozumění pravidlům vzájemného soužití
						\item[-]hry vedoucí k ohleduplnosti k druhému, ochotě rozdělit se, pomoci si, vyřešit spor
						\item[-]činnosti na poznávání sociálního prostředí (rodina, MŠ)
						\item[-]hry, kdy se dítě učí chránit soukromí a bezpečí své i druhých
						\item[-]četba, vyprávění a poslech příběhu s etickým obsahem a ponaučením
					\end{itemize}
					
					\paragraph{Očekávané výstupy}

					\begin{itemize}
					\setlength\itemsep{-2mm}
						\item[-]navazovat kontakty s dospělým, kterému je svěřeno do péče
						\item[-]komunikovat s ním, respektovat ho, porozumět projevům emocí a nálad
						\item[-]přirozeně komunikovat s druhým dítětem, navazovat přátelství
						\item[-]uvědomovat si svá práva a respektovat práva ostatních
						\item[-]uplatňovat své individuální potřeby a přání s ohledem na druhé
						\item[-]dodržovat pravidla vzájemného soužití v různých prostředích i pravidla her
						\item[-]respektovat potřeby jiného dítěte, dělit se s ním o věci
						\item[-]vycházet vstříc ostatním a pomáhat jim
						\item[-]bránit se projevům násilí
						\item[-]chovat se obezřetně při setkáních s neznámými dětmi a dospělými
					\end{itemize}

			\subsection{Dítě a společnost}
				\textit{\uv{Záměrem vzdělávacího úsilí pedagoga v oblasti sociálně-kulturní je uvést dítě do společenství ostatních lidí a do pravidel soužití s ostatními, uvést je do světa materiálních i duchovních hodnot, do světa kultury a umění, pomoci dítěti osvojit si potřebné dovednosti, návyky i postoje a umožnit mu aktivně se podílet na utváření společenské pohody ve svém sociálním prostředí.}} \citep[s.~26]{RVP}

					\paragraph{Dílčí vzdělávací cíle}

					\begin{itemize}
					\setlength\itemsep{-2mm}
						\item[-]poznávání pravidel společenského soužití a jejich spoluvytváření
						\item[-]porozumění základním projevům neverbální komunikace v tomto prostředí
						\item[-]rozvoj schopnosti žít ve společenství ostatních lidí a přijímat základní hodnoty v tomto společenství uznávané
						\item[-]rozvoj základních kulturně společenských postojů
						\item[-]rozvoj schopnosti projevovat se autenticky a autonomně
						\item[-]vytvoření povědomí o morálních hodnotách
						\item[-]seznamování se světem lidí, kultury a umění
						\item[-]vytváření povědomí o jiných kulturách, rozvoj společenského i estetického vkusu
					\end{itemize}

					\paragraph{Vzdělávací nabídka}

					\begin{itemize}
					\setlength\itemsep{-2mm}
						\item[-]setkávání s pozitivními vzory vztahů a chování
						\item[-]aktivity pro adaptaci dítěte v MŠ
						\item[-]společenské hry a skupinové aktivity umožňující dětem se spolupodílet na jejich průběhu
						\item[-]přípravy a realizace společenských zábav a slavností
						\item[-]tvůrčí a receptivní činnosti slovesné, literární, dramatické, výtvarné apod.
						\item[-]návštěvy kulturních a uměleckých míst
						\item[-]hry na poznávání různých společenských rolí
						\item[-]aktivity přibližující pravidla vzájemného styku a mravní hodnoty
						\item[-]hry a praktické činnosti uvádějící dítě do světa lidí, jejich občanského života a práce
						\item[-]aktivity přibližující svět kultury a umění a umožňující poznat rozmanitost kultur
					\end{itemize}
					
					\paragraph{Očekávané výstupy}

					\begin{itemize}
					\setlength\itemsep{-2mm}
						\item[-]uplatňovat návyky společenského chování ve styku s dospělými i dětmi
						\item[-]pochopit, že každý má ve společenství svou roli a podle ní se chovat
						\item[-]chovat se dle vlastních pohnutek, ale s ohledem na druhé
						\item[-]začlenit se do třídy a respektovat rozdílné vlastnosti vrstevníků
						\item[-]porozumět běžným neverbálním projevům citových prožitků a nálad druhých
						\item[-]adaptovat se ve škole a zvládat požadavky prostředí
						\item[-]vyjednávat s ostatními a domluvit se na společném řešení
						\item[-]utvořit si základní dětskou představu o pravidlech chování a společenských normách a chovat se dle toho
						\item[-]jednat spravedlivě, hrát fér, dodržovat pravidla her 
						\item[-]odmítat společensky nežádoucí chování a chránit se před ním
						\item[-]vnímat umělecké podněty a hodnotit svoje zážitky
						\item[-]vyjadřovat se pomocí výtvarných technik a prostřednictvím hudeních a hudebně pohybových činností
					\end{itemize}

			\subsection{Dítě a svět}
				\textit{\uv{Záměrem vzdělávacího úsilí pedagoga v environmentální oblasti je založit u dítěte elementární povědomí o okolním světě a jeho dění, o vlivu člověka na životní prostředí – počínaje nejbližším okolím a konče globálními problémy celosvětového dosahu – a vytvořit elementární základy pro otevřený a odpovědný postoj dítěte (člověka) k životnímu prostředí.}} \citep[s.~29]{RVP}

				\paragraph{Dílčí vzdělávací cíle}

				\begin{itemize}
				\setlength\itemsep{-2mm}
					\item[-]seznamování se a vytváření si pozitivního vztahu k místu a prostředí, ve kterém dítě žije
					\item[-]poznávání jiných kultur
					\item[-]vytváření povědomí o širším přírodním, kulturním i technickém prostředí
					\item[-]pochopení, že lidská činnost může prostředí chránit, ale i ničit
					\item[-]osvojení si poznatků péče o okolí a spoluvytváření zdravého prostředí
					\item[-]rozvoj úcty k životu ve všech jeho formách
					\item[-]rozvoj schopnosti přizpůsobovat se podmínkám prostředí
					\item[-]vytvoření povědomí o vlastní sounáležitosti se světem
				\end{itemize}
				
				\paragraph{Vzdělávací nabídka}

				\begin{itemize}
				\setlength\itemsep{-2mm}
					\item[-]přirozené pozorování prostředí a života v něm
					\item[-]aktivity zaměřené na praktickou orientaci v obci
					\item[-]poučení o možných nebezpečných situacích a způsobech, jak se chránit
					\item[-]aktivity na téma dopravy, cvičení bezpečného chování v dopravních situacích
					\item[-]poznávání přírodního okolí, sledování rozmanitostí a změn v přírodě
					\item[-]využívání encyklopedií a obrazového materiálu
					\item[-]kognitivní činnosti, praktické činnosti k seznámení s materiály
					\item[-]pozorování životních podmínek a životního prostředí a okolní krajinu
				\end{itemize}
				
				\paragraph{Očekávané výstupy}

				\begin{itemize}
				\setlength\itemsep{-2mm}
					\item[-]bezpečně se orientovat ve známém prostředí
					\item[-]zvládat běžné činnosti a požadavky na dítě kladené
					\item[-]chovat se přiměřeně a bezpečně doma i na veřejnosti
					\item[-]uvědomovat si nebezpečí a jak se chránit
					\item[-]osvojit si elementární poznatky o okolním prostředí
					\item[-]vnímat, že svět má svůj řád, že je rozmanitý a pestrý
					\item[-]všímat si změn v okolí a porozumět, že změny jsou přirozené a samozřejmé
					\item[-]mít povědomí o významu životního prostředí
					\item[-]pomáhat pečovat o okolní prostředí a rozlišovat aktivity, které mohou zdraví okolního prostředí podporovat či poškozovat
				\end{itemize}

 			\subsection{Shrnutí cílů a pojetí předškolního vzdělávání a role pedagoga}

				Pojetím a cíli předškolní vzdělávání se v českém dokumentu věnuje samostatná kapitola \uv{Pojetí a cíle předškolního vzdělávání}. Níže je uvedeno shrnutí důležitých témat včetně vymezení role pedagoga a doporučení pro jeho práci, volbu vhodných metod vzdělávání a postupů. Výstupy vzdělávácích cílů předškolního vzdělávání jsou formulovány jako klíčové kompetence.			

				Mezi hlavní principy RVP PV patří akceptování přirozených vývojových specifik dětí předškolního věku, vzdělávání dítěte v rozsahu jeho individuálních možností a potřeb, vytváření základů a osvojení si klíčových kompetencí (viz níže) dosažitelných v etapě předškolního vzdělávání a získávání předpokladů pro celoživotní vzdělávání.
				Předškolní vzdělávání má dítěti usnadňovat jeho další životní i vzdělávací cestu, rozvíjet jeho osobnost, tělesný rozvoj a zdraví, osobní spokojenost a pohodu a napomáhat mu v chápání okolního světa a motivovat jej k dalšímu poznávání, stejně tak jako učit dítě žít ve společnosti ostatních a přibližovat mu normy a hodnoty uznávané touto společností.
				Vhodnými metodami vzdělávání je dle RVP PV prožitkové a kooperativní učení hrou. Jde o činnosti, které jsou založeny na přímých zážitcích dítěte, které podporují dětskou zvídavost a potřebu objevovat, podněcují jeho radost z učení a jeho zájem poznávat nové věci a získávat nové zkušenosti.
				RVP řadí situační učení a spontánní sociální učení mezi významné procesy učení, které by měly být dostatečně zastoupeny. Aktivity by se měly střídat spontánní i řízené a měly by být vzájemně provázané a vyvážené. Pedagog by měl být průvodcem dítěte, probouzet v něm aktivní zájem a chuť dívat se kolem sebe, naslouchat a objevovat. Není zde jako ten, který „úkoluje“ a kontroluje. Didaktický styl by měl být založen na principu vzdělávací nabídky, na individuální volbě dítěte a jeho aktivní účasti.
				

				Důležitou složkou RVP PV jsou výstupy vzdělávacích cílů. Těmi jsou klíčové kompetence, neboli kompetence, které by děti měly ovládat na konci mateřské školy a před vstupem do školy základní. Je zde uvedeno 5 kompetencí. Ke každé z nich uvedu opět stručný výtah:

				\paragraph{Kompetence k učení}
				\begin{itemize}
				\setlength\itemsep{-2mm}
				\item[-] elementární poznatky o světě lidí, kultuře, přírodě a technice, která dítě obklopuje
				\item[-] orientace se v řádu dění
				\item[-] klást otázky a hledat na ně odpovědi
				\item[-] aktivně si všímat a chtít porozumět jevům, které kolem sebe vidí
				\item[-] soustředěně pozorovat, objevovat, experimentovat a získanou zkušenost dále uplatňovat
				\item[-] učení probíhá spontánně, vědomě, s chutí
				\item[-] schopnost soustředit se na činnost a dokončit práci
				\item[-] postupovat podle instrukcí, odhadovat své síly 
				\end{itemize}
				

				\paragraph{Kompetence k řešení problémů}
				\begin{itemize}
				\setlength\itemsep{-2mm}
				\item[-] všímat si dění i problémů okolo sebe
				\item[-] známé situace řešit samostatně, náročnější s oporou a pomocí dospělého
				\item[-] řešit problémy  na základě bezprostřední zkušenosti, cestou pokusu a omylu experimentovat, vymýšlet nová řešení, hledat varianty
				\item[-] využívat dosavadních zkušeností, fantazii a představivost 
				\item[-] užívat logických, matematických a empirických postupů 
				\item[-] dovednost volit mezi řešením vedoucím k cíli a řešením nefunkčním
				\item[-] nebát se chybovat
				\end{itemize}
				

				\paragraph{Kompetence komunikativní}
				\begin{itemize}
				\setlength\itemsep{-2mm}
				\item[-] v běžných situacích komunikovat bez zábran a ostychu s dětmi i s dospělými
				\item[-] ovládat řeč, samostatně vyjadřovat své myšlenky ve vhodně formulovaných větách
				\item[-] dokázat sdělit své prožitky a pocity a to všemi prostředky (i výtvarnými, hudebními, dramatickými) 
				\item[-] ovládat dovednosti předcházející čtení a psaní 
				\item[-] mít povědomí o existenci jiných jazyků
				\item[-] průběžně rozšiřovat slovní zásobu a aktivně ji používat
				\item[-] využívat informativní a komunikační prostředky (telefon, knihy, počítač…)
				\end{itemize}

				\paragraph{Kompetence sociální a personální}
				\begin{itemize}
				\setlength\itemsep{-2mm}
				\item[-] samostatně se rozhodovat o svých činnostech
				\item[-] umět si vytvořit a vyjádřit svůj názor
				\item[-] uvědomovat si, že odpovídá za své jednání, projevovat citlivost a ohleduplnost k druhým, vnímat nespravedlnost a ubližování
				\item[-] domlouvat se a spolupracovat při společenských činnostech
				\item[-] uplatňovat pravidla společenského styku a dodržovat dohodnutá pravidla 
				\item[-] respektovat druhé a uzavírat kompromisy 
				\item[-] umět být tolerantní k odlišnostem druhých lidí
				\item[-] dokázat se bránit násilí a ponižování
				\end{itemize}

				\paragraph{Kompetence činnostní a občanské}
				\begin{itemize}
				\setlength\itemsep{-2mm}
				\item[-] dokázat rozpoznat svoje silné a slabé stránky
				\item[-] učit se plánovat, organizovat a vyhodnocovat svoje činnosti a hry
				\item[-] odhadovat rizika svých nápadů a dokázat se přizpůsobovat okolnostem
				\item[-] chápat, že se může svobodně rozhodovat, ale i že nese odpovědnost za svá rozhodnutí
				\item[-] zajímat se o druhé a o to, co se děje kolem
				\item[-] mít smysl pro povinnost ve hře, práci i učení
				\item[-] uvědomuje si svá práva i práva druhých
				\item[-] uvědomovat si, že svým chováním ovlivňuje prostředí, ve kterém žije, a dbát na osobní zdraví a bezpečí své i druhých
				\end{itemize}



\begin{landscape}
\begin{table}[t]
\center
\begin{tabular}{|c|c|c|l|}
\rowcolor{grey}
\hline			
				& \textbf{Francie}			& \textbf{Česká republika}	& \textbf{Rozdíly} 	\\
\hline
\hline
\rowcolor{grey!10}
Dokument	& státní, veřejně dostupný		& státní, veřejně dostupný 	& 			\\	\rowcolor{grey!50}
Rozsah		& 12 kapitol	 				& 11 kapitol	&						\\  \rowcolor{grey!10}
Předškolnímu vzdělání	& celý dokument			& 2 kapitoly	& FR: vzdělávání na půdě MŠ  zasahuje 	\\ \rowcolor{grey!10}
se věnuje				&						&				& do dvou vzdělávacích cyklů, proto  	\\ \rowcolor{grey!10}
				&								& 				& je prezentován v jednom dokumentu  	\\ \rowcolor{grey!10}
				&								&				& společně s ostatními cykly.    		\\ \rowcolor{grey!10}
				&								&				& ČR: preprimární vzdělávání odděleno 	\\ \rowcolor{grey!10}
				&								&				& od základního, věnuje se mu tedy  	\\ \rowcolor{grey!10}
				&								&				& samostatný dokument.					\\ 
\rowcolor{grey!50}
Kompetence		& na konci každé 				& hlavní výstupy  	&		\\	\rowcolor{grey!50}
				& vzdělávací oblasti 			& vzdělávacích cílů & 		\\
\rowcolor{grey!10}
Vzdělávací oblasti	& 6							& 5 					& FR: vzdělávací oblasti určují čeho 	\\	
\rowcolor{grey!10}
					& Osvojit si řeč			& Dítě a jeho tělo		& má dítě dosáhnout.					\\
\rowcolor{grey!10}
					& Objevovat písmo 			& Dítě a jeho psychika	& ČR: vzdělávací oblasti jsou sestave- \\
\rowcolor{grey!10}
					& Stát se žákem				& Dítě a ten druhý	 	& né z pozice dítěte k obsahu. 			\\
\rowcolor{grey!10}
					& Jednat a vyjadřovat se 	& Dítě a společnost 	& 	 									\\
\rowcolor{grey!10}
					& vlastním tělem			& Dítě a svět			& 	 									\\
\rowcolor{grey!10}
					& Objevovat svět			& 			 			& 	\\	
\rowcolor{grey!10}
					& Vnímat, cítit, 			& 						& 	\\
\rowcolor{grey!10}
					& představovat si, tvořit 	&						& 	\\		
\rowcolor{grey!50}
Dílčí vzdělávací cíle	& ANO					& ANO 					& 	\\	
\rowcolor{grey!10}
Vzdělávací nabídka	& Okrajově popsána	 		& Podrobně popsána	 	& Ve FR je dán větší prostor učitelům, 	\\
\rowcolor{grey!10}
					&							&						& jak budou práci koncipovat. 			\\
\rowcolor{grey!10}
					&							&						& V ČR je tento prostor užší, nabídka 	\\
\rowcolor{grey!10}
					&							&						& je přesněji určena. 					\\
\rowcolor{grey!50}
Očekávané výstupy	& Formulované jako 				& Detailně popsány pro 		& 		\\ 
\rowcolor{grey!50}
					& kompetence na konci kapitoly 	& každou vzdělávací oblast 	& 		\\
\rowcolor{grey!10}
Rizika	 			& NE						& ANO	 						& 		\\
\hline
\end{tabular}
\caption{ \textbf{Srovnání kurikul Francie a České republiky.}
}
\label{tab:srovnaniKurikul}
\end{table}
\end{landscape}

\section{Srovnání kurikul Francie a České republiky}
\label{srovnanikurikulfrcr}

	Kurikulární dokumenty obou srovnávaných zemí, tedy Francie a České republiky, mají stejné legislativní zázemí. Jsou to státní dokumenty zakotvené ve školském zákoně a oba jsou volně dostupné široké veřejnosti, jak pedagogické tak nepedagogické. 

	Rozsah obou dokumentů je téměř shodný, ale český Rámcový vzdělávací plán je zaměřen pouze na předškolní vzdělávání a podrobně se věnuje všem oblastem, včetně rizik, které by mohly ohrozit úspěch vzdělávacích záměrů pedagoga, začleňování dětí se speciálními potřebami, podmínkám předškolního vzdělávání, autoevaluaci mateřských škol i tvorbě školních vzdělávacích plánů. Jeho obsah je vyčerpávající a zahrnuje všechny okolnosti, v nichž se odehrává vývoj dítěte. Oproti tomu je francouzský dokument rozsahem kratší a věnuje se zvláště vzdělávacím oblastem a cílům vzdělávání. Poslední třída mateřské školy je zároveň první třídou druhého vzdělávacího cyklu. Jde o plynulý přechod na školu primární, tomu odpovídá i obsah vzdělávání, jak mateřské školy, tak prvních tříd školy základní. Cykly (viz kap. \ref{msvefr}) na sebe přirozeně navazují. Z tohoto důvodu je program mateřských škol prezentován ve stejném dokumentu jako program primárního vzdělávání a je mu věnována jeho poměrná část.

	Cíle v českém RVP PV jsou podrobně rozpracovány na záměry a výstupy. Formulovanými výstupy vzdělávacích cílů je 5 klíčových kompetencí, které jsou detailně popsány a k jejichž naplňování by mělo směřovat veškeré vzdělávání v mateřské škole. Ve francouzském Programu pro mateřské školy jsou kompetence zmíněné též, tzv. \uv{znalosti a dovednosti, které by děti měly ovládat na konci MŠ}, ale jsou uvedeny na konci každé vzdělávací oblasti.

	Formulačně jsou vzdělávací oblasti obou dokumentů odlišné, ale při detailnějším čtení je zřejmé, že přestože jsou vzdělávací oblasti zařazené do jiných skupin, obsahují oba dokumenty tytéž oblasti, které je u dětí potřeba systematicky rozvíjet. Nejmarkantnějšími rozdíly jsou ve francouzském programu v obsahu oblastí \textbf{Objevovat písmo} (Découvrir l´écrit) a \textbf{Stát se žákem} (Devenir élève).

	Ve Francii kladen velký důraz na nácvik psaní. Na konci mateřské školy by děti měly ovládat velká tiskací písmena. Jak je v programu uvedeno, některým dětem se předkládá již písmo psací. Děti hodně kopírují slova z předloh, jako jsou dny v týdnu a názvy měsíců v roce a učí se svému podpisu. V České republice se jedná spíše o přípravu na psaní. Uvolňování svalů ruky, správné držení pera či tužky, nácvik grafomotoriky. Děti si uvědomují tvary, učí se uvolňovat zápěstí, ale neučí se psát konkrétní písmena a nekopírují, s výjimkou vlastního křestního jména. Písmena píší, jen když to vychází z jejich vlastního zájmu. 

	Název druhé zmíněné oblasti \textbf{Stát se žákem} evokuje přípravu na základní školu, a status žáka v plném smyslu slova. Její obsah je společný s českými oblastmi \uv{Dítě a ten druhý} a \uv{Dítě a společnost}, přestože z pouhého názvu kapitol to není na první pohled patrné. Shoda obou zemí je ve formálním obsahu, ovšem v pojetí hlavních vzdělávacích cílů se obě země diametrálně liší.

	Ve Francii si mateřská škola dává za úkol dostatečně dětem vštípit dovednosti a znalosti, které povedou k úspěšnému zvládnutí základní školy. Formuje žáky a uvádí je do světa psaného jazyka a tím pádem i do světa čtení. Během pozorování byl ve Francii znatelný rozdíl v přístupu učitelů k dětem, kterému se krátce věnuji v závěru kapitoly~\ref{srovnani}. Učitel rozdával \uv{úkoly}, děti \uv{pracovaly} viz.~\ref{ateliery}. Autorka na základě kurikula a vzdělávacích cílů vyvozuje, že na dítě je ve Francii pohlíženo jako na \uv{žáka} a tak se k němu také přistupuje. 

	Mezi cíle předškolního vzdělávání v České republice patří rozvoj osobnosti dítěte, jeho samostatnosti, rozvoj učení a poznávání a osvojení si hodnot naší společnosti. Vzdělávací oblasti jsou vzájemně provázány a respektují přirozenost dítěte a jeho postupné začleňování do životního a sociálního prostředí. 

	Autorka má dojem, že oproti Francii je v českém předškolním vzdělávání více respektováno a podporováno období dětství, s jeho vývojovými potřebami a specifiky. Proto usuzuje, že v České republice je na dítě pohlíženo jako na \uv{dítě}.

\part{PRAKTICKÁ ČÁST}
\chapter{REŽIM DNE}
\label{rezim}
	Další srovnávaný aspekt této práce je režim dne. Tato část práce vychází z~vlastních zkušeností autorky, která se v~rámci studií účastnila povinných praxí jak v~české, tak ve francouzské mateřské škole. Byla použita metoda nestrukturovaného pozorování, která je shrnuta v~kapitole~\ref{metody}. Níže je prezentována vždy jen jedna třída jedné mateřské školy z~dané země. Nejde o~bohatý výzkumný vzorek, z~kterého by se daly vyvozovat obecné závěry, avšak cílem je přiblížit čtenáři organizační strukturu jednoho dne v~předškolním zařízení obou zemí. Kraus stručně definuje režim jako\textit{\uv{přesně určený rozvrh života, práce, činnosti}}~\citep[s.~700]{Kraus}.
	Pro tuto práci z~toho vyplývá, že režimem dne je míněn přesný časový rozvrh běžného dne, který se pravidelně opakuje.
	Součástí této kapitoly je i~popis materiálního zázemí obou srovnávaných škol doplněný, jelikož autorka považuje za důležité zdůraznit rozdílnosti prostředí, ve kterém se předškolní vzdělávání odehrává.
	
	\section{Režim dne ve francouzské mateřské škole}

		Praxe ve Francii probíhala od 8. 11. 2010 do 19. 11. 2010 v~mateřské škole Maintenon na adrese (3, rue des Glycines, 927 00 Colombes) na předměstí Paříže. Tato mateřská škola spolupracovala s~Académie Versailles, která je součástí Université de Cergy-Pontoise. Ta je partnerskou univerzitou Univerzity Karlovy v~rámci studentského projektu Erasmus. 


		\subsection{Průběh dne a~jeho specifika}

			Provozní doba nateřské školy byla od 8:50 do 16:45. Zajímavostí francouzských mateřských škol je čtyřdenní vyučovací týden. Děti navštěvují mateřskou školu jen v~pondělí, úterý, čtvrtek a~pátek. Ve středu děti zůstávají doma nebo mají volnočasové aktivity a~sporty. Některé školy tyto aktivity nabízejí, jiné však nikoli. Výše zmíněná mateřská škola patřila mezi ty, které aktivity nenabízejí. 
			
			Časový harmonogram je závazný. V~tabulce \ref{tab:rezimDneFR} je uveden harmonogram třídy \uv{Grande section} (další podrobnost ke třídě v~kapitolách~\ref{tridaVybaveni} a~\ref{trida}). Obdobný rozvrh byl vyvěšen v tištěné podobě na dveřích každé třídy. 

\begin{spacing}{1.0}
	\begin{table}[h!]
		\center
		\begin{tabular}{|l l|}
			\hline
			\rowcolor{grey!0}
			8:50 – 9:10 		& Příchod dětí a~jejich uvítání 						\\
								& (volná hra, dokončování prací z minulého dne, úklid) 	\\
			9:10 – 9:20			& Rituály 												\\
								& (pozdravení se, datum, počasí, představení ateliérů) 	\\
			9:20 – 10:05		& Dílny 												\\
								& (grafomotorika/psaní, matematika, čtení) 				\\
			10:05 – 10:15		& Společný kruh (úklid, básničky/říkanky) 				\\
			10:15 – 10:45		& Přestávka 											\\
			10:45 – 11:30		& Lingvistické aktivity, společné čtení 				\\
			11:30 – 11:50		& Společný kruh (říkanky, matematické hry) 				\\
			11:50 – 13:30		& Oběd 													\\
			13:30 – 13:55		& Společný kruh (zpěv, hlasová cvičení, poslech) 		\\
			13:55 – 14:30		& Ateliéry 												\\
								&(motorika, výtvarná výchova, objevování světa) 		\\
			14:30 – 15:00		& Tělocvična 											\\
			15:00 – 15:30		& Přestávka 											\\
			15:30 – 16:10		& Video nebo promítání diapozitivů 						\\
			16:10 – 16:20		& Úklid třídy, zhodnocení dne 							\\
			16:20 – 16:30		& Odchod dětí 											\\
			\hline
		\end{tabular}
		\caption{ \textbf{Časový harmonogram ve francouzské MŠ}	
		}
		\label{tab:rezimDneFR}
	\end{table}
	\end{spacing}

		\subsection{Příchod dětí do mateřské školy}
		\label{prichod}
			Mateřská škola byla ráno otevřena od 8:50 do 9:10. Na chodbě před třídou mělo každé dítě svůj háček na pověšení oblečení a~malou přihrádku na menší věci (Obr.~\ref{Obr9}). Děti se nepřezouvaly, zůstávaly celý den ve stejné obuvi, ve které přišly. Při vstupu do třídy vítala paní učitelka děti i~jejich rodiče. S každým dítětem se poté snažila navázat kontakt. Kladla dětem otázky, jak se mají, co dělaly o~víkendu apod. Každé dítě si poté našlo na stole cedulku se svým jménem a~přileplo ji na menší tabuli se suchým zipem vedle velké tabule (Obr.~\ref{Obr10}). Tímto byla zjištěna docházka dětí, která je poté součástí ranního rituálu. Dále měly děti čas na volnou hru, malování, prohlížení knížek, skládání puzzle, dokončování výtvarných prací z minulého dne.
		
		\subsection{Rituály}
		\label{ritualy}
			K~rannímu rituálu se děti usazovaly na lavičky před tabulí. Některé děti si kvůli nedostatku míst sedaly na koberec. Jedno vybrané dítě mělo za úkol spočítat kartičky se jmény děvčat a~chlapců a~kolik dětí je přítomno celkově. Tato čísla zapsalo na tabuli na určené místo. Další dítě mělo poté na starost datum, nejdříve změnilo číslici dne a~napsalo na tabuli novou. Je-li potřeba, mění toto dítě i~kartičku s~názvem měsíce. Třída poté společně přečetla celé datum (Obr.~\ref{Obr11},~\ref{Obr12}). Dále všichni společně zarecitovaly uvítací říkanku (v~této třídě se jednalo o~básničku s názvy dní a~děti přitom ukazovaly na prstech ruky jejich pořadí). Jako poslední bod rituálu vyučující dětem vysvětlila, jaké aktivity je ten den čekají a~podrobně je popsala. 

		\subsection{Pravidla chování}
		\label{pravidlaChovani}
			Zajímavostí této třídy byly obrázky s~pravidly, která se měla ve dodržovat. Pravidla byla zobrazena na červeném a~zeleném papíře velikosti A3 (viz Obr.~\ref{Obr6}) a~byly viditelné již ode dveří třídy. Na pravidla se vyučující odkazovala téměř pokaždé, když byla některá z nich porušena. Dítě, které nějakým způsobem pravidla nedodrželo, bylo vyzváno, aby ukázalo, o~které pravidlo se jedná a povědělo všem, jak by se mělo chovat. 

			\begin{spacing}{1.0}
			\begin{table}[h!]
				\center
				\begin{tabular}{|ll|ll|}
					\hline
					\rowcolor{grey!0}
				+	& Papír se vyhazuje do koše						& -	& Neprat se 			\\
					& Hlásit se 									&  	& Neběhat po třídě		\\
					& Uklízet po sobě materiál 						&	& Nestrkat se 			\\
					& Řadit se do řady 								&	& Nekřičet 				\\
					& Být potichu 									& 	& Neničit materiál 		\\
					& Udržovat stoly čisté 							& 	& Neschovávat věci 		\\
					& Říkat „Dobrý den“,							&	& Neříkat sprostá slova \\
					&  „Na shledanou“, „Děkuji“						&	& Nekrást				\\
					&												&	& Neobtěžovat kamarády 	\\
					\hline
				\end{tabular}
				\caption{ \textbf{Srnutí pravidel chování ve francouzské školce.}}
			\label{tab:pravidlaFR}
			\end{table}
			\end{spacing}

			\subsection{Společný kruh}
			V průběhu dne se konaly dvě seskupení u~tabule. Tento čas byl zaměřen na básničky, říkanky, matematické hry, zpěv, hlasová cvičení, poslech a~na učení se nových písmen nebo číslic. Co bude tématem daného dne, se odvíjelo ode dne předešlého. Například se opakovala básnička či písnička nebo se přidávala nová sloka, učila se nová číslice či písmeno, anebo se prohlížela a~četla nějaká kniha. Pokud chtělo dítě něco říci, muselo se podle pravidel třídy přihlásit a~počkat, až bude vyvoláno, podobně jako tomu je ve škole. Smí mluvit pouze jedno dítě, musí mluvit nahlas a~ostatní děti ho nesmějí vyrušovat. Některé děti se jen hlásily, protože chtěly být vyvolány, ale žádnou odpověď nevěděly. Hlášení se muselo striktně dodržovat. Oproti tomu, když se četla či prohlížela nová kniha, nechala paní učitelka děti mluvit více spontánně, aby se všechny mohly dostatečně vyjádřit.

		\subsection{Dílny}
		\label{dilny}
			Z~francouzského originálu \uv{ateliers}, v~české terminologii se dá chápat jako specializovaný koutek, dílničky nebo dílny, dále tedy jen dílny. 
			Během dílen se sedí u~stolů a~každé dítě pracuje individuálně, nesmějí si pomáhat. Dětem bylo stále připomínáno, že si \uv{nehrají}, ale \uv{pracují}. Vzhledem k~vysokému počtu byly děti rozděleny do čtyř skupinek po 6 - 7 dětech. Každá skupina pracovala na jiném úkolu. Například jedna skupina vyplňovala grafomotorické listy, druhá skupina stříhala, sestavovala a~lepila, třetí stavěla ze stavebnic a~poslední měla prematematické činnosti. Každý den se témata dílen předala další skupině, takže na konci týdne všechny děti pracovaly na všech aktivitách. Tento způsob práce vyžaduje od vyučující nasazení, spolupráci a~pozornost. Ta postupně obchází všechny stolky a~pomáhá těm dětem, které to potřebují. Finální práce si děti sami podepisovaly. V~připravených kelímcích byly kartičky s~názvy dnů a~měsíců, podle kterých děti opisovaly nebo kopírovaly datum na své práce. Hotové práce si poté děti lepily do sešitů, které tak byly základem jejich portfolia. Jednou až dvakrát za půl roku byly poskytnuty rodičům, aby se mohli podívat, na čem děti pracují a~jaké dělají pokroky (Obr.~\ref{Obr15}). Odpolední dílny měly již odpočinkovější nádech. K~dispozici byly čtyři počítače s~prematematickými hrami, které byly u~dětí velmi oblíbené. Dále byly v~nabídce stavebnice Lego, tématická výtvarná činnost či stříhání a~opětovné skládání částí lidského těla. Při výtvarné aktivitě byl dětem ukázán vzor, podle kterého měly malovat. Vyučující děti hodně korigovala, aby byl výtvor vzoru co nejpodobnější.


		\subsection{Pobyt venku}
		\label{prestavka}
			Děti měly během vyučování vyhrazeny dva třicetiminutové bloky na pobyt venku. Ven se chodilo na dvůr, kde se sešly najednou všechny třídy, nad kterými měly dozor vždy minimálně dvě učitelky. Každá učitelka měla dozor dvakrát do týdne. Čas pobytu venku byl flexibilní a~přizpůsoboval se aktuálnímu počasí. Ven se však chodilo i~za mírného deště. Dvůr školy má na jedné straně přístřešek, kde se děti mohly při špatném počasí schovat. Děti měly k dispozici tříkolky a~odstrkovadla. Používat je však mohla pouze třída, jejíž učitelka měla zrovna službu na dozor. K dispozici byly i~míče. Děti se převážně honily, povídaly si ve dvojicích až trojicích, některé děti jen postávaly. Praxe se konala během podzimu, na zemi bylo spadané listí, a~tak si některé děti hrály s listím, jiné dostaly koště a pomáhaly listí zametat. Několik dětí postávalo pod přístřeškem a~čekalo, až přestávka skončí, protože jim byla zima z důvodu nedostatečného oblečení a~nechtěly si kvůli tomu hrát. Přestože je na dvoře k dispozici prolézačka, děti na ni kvůli špatnému počasí nesměly (Obr.~\ref{Obr16}, \ref{Obr17}). 
			Konec pobytu na dvoře se oznamoval zazvoněním na zvoneček. Děti se pak řadily ke dveřím své třídy, kde si je vyzvedla jejich vyučující. 

		\subsection{Strava a~pitný režim}
			Mateřská škola měla svou vlastní jídelnu. Jídlo se dováželo a~v~jídelně se pouze ohřívalo. Stravování v jídelně nebylo povinné. Tradičně si ve Francii rodiče odvádějí děti na oběd domů a~do školky je vracejí ve 13:30. V~této mateřské škole však většina dětí využívala možnosti poskytované stravy. 
			Následující popis je specifický jen pro tuto mateřskou školu. Oběd byl pro děti jediná strava během dne. Svačina se v této škole nepodávala vůbec a~při pobytu venku bylo zakázáno cokoliv konzumovat. Dříve si děti nosily svačiny z~domova, ale z~důvodu údajné závisti dětí byl vedením mateřské školy vydán zákaz jakéhokoliv nošení potravin do školy. 
			Příjem tekutin byl povolen během celého dne. U~umyvadla měly děti připravené kelímky a~kdykoliv požádaly, mohly si samy natočit vodu z vodovodu a~napít se. Učitelka občas děti upozornila, že se mohou napít, ale nebyl zde kladen větší důraz na dodržování pitného režimu.

		\subsection{Odpolední klid}
		\label{spani}
			Po obědě chodily děti opět na dvůr, kde se o~ně starali dva vychovatelé, většinou studenti volnočasových aktivit či budoucí učitelé sportu.

		\subsection{Odchod dětí z~mateřské školy}
			Rodiče si své děti vyzvedávali u~dveří třídy a~vyučující osobně volal dítě, které má odcházet. Bez vědomí vyučujícího nesmělo žádné dítě odejít. U~dveří ze školy stál pan ředitel, zdravil všechny rodiče a~dohlížel na bezpečnost odchodů. 

		\subsection{Popis budovy mateřské školy}

			Mateřská škola byla součástí velké budovy, kde se nacházela i~škola základní. Děti měly přístup na betonový dvůr, který měl jen na menších částech speciální měkký povrch. K~dispozici zde byly dvě prolézačky a~pískoviště. Děti měly možnost na dvoře jezdit na odstrkovadlech, tříkolkách a~hrát si s~míči. 
			Mateřská škola měla osm tříd, z~toho šest se nacházelo v přízemí budovy a~dvě byly v prvním patře, v prostorách základní školy. Dále se zde nacházela knihovna, tělocvična, dvě ložnice na spaní, sborovna a~jídelna. 

		\subsection{Popis třídy a~materiálního vybavení}
		\label{tridaVybaveni}
			Třída, ve které se praxe konala, byla menší místnost s~položenou dlažbou. Před tabulí byl menší koberec. Místnost měla dvoje dvěře, jedny vedly na školní chodbu a~druhé přímo na dvůr.
			Na jedné straně třídy visela na zdi tabule (Obr.~\ref{Obr1}), na které byly přilepené různé cedulky s návody, jak se píší číslice 1 - 10, menší cedulky s~číslicemi od 1 do 30, cedulky s~názvy dnů v týdnu a~měsíců v~roce. V~rohu tabule visel popis dílen a~v~dolní části se psal počet dětí a~aktuální datum . Před tabulí byly do kruhu postaveny tři lavice na sezení. Po obou stranách tabule byly umístěny stolky s dětskými počítači (Obr.~\ref{Obr2}). Uprostřed třídy bylo postaveno pět stolů, každý se šesti židličkami. Na protilehlé straně třídy se nacházel čtenářský koutek s křesílkem, umyvadla, police na výtvarný materiál, bílá tabule, stůl pro vyučujícího, dětská kuchyňka s popisky věcí, které patří do kuchyně (Obr.~\ref{Obr3},~\ref{Obr4},~\ref{Obr5}). Na této straně byla na zdi pověšená písmena abecedy velikosti A3 (Obr.~\ref{Obr6}). Čtenářský koutek i~kuchyňka byly od třídy odděleny různými skříňkami a~poličkami, které sloužily k uskladnění didaktických pomůcek a~her či sešitů dětí (Obr.~\ref{Obr7}).

			Třída byla materiálně velmi dobře vybavena. K~dispozici zde bylo mnoho didaktických pomůcek. Ve třídě se nacházelo vše potřebné, ovšem pro volnou hru a~volný pohyb dětí mnoho místa nezbývalo. K~té se dal využít menší prostor s kobercem před tabulí, dětská kuchyňka nebo prostor před vchodem na dvůr a~ulička napříč třídou. Celkový prostor třídy se zdál pro 29 dětí stísněně.

			Děti měly na výběr různé stolní hry, puzzle a~knihy. Hraček, tak jak je známe z~českých školek, jako jsou panenky, plyšáci, autíčka apod., se zde nacházelo minimum. Pokud si děti donesly nějakou hračku z~domova, musely ji na začátku hodiny odložit na poličku a~vyzvednout si ji mohly až na konci dne. 

		\subsection{Počet dětí a~pedagogů ve třídě}
		\label{trida}

			Praxe se konala ve třídě “Grand section“ (viz tabulka~\ref{tab:rozdeleniTridFR}). Jedná se o~třídu homogenní, tzn. třídu, kterou navštěvují děti stejného věku. Ve třídě bylo zapsáno 29 dětí, mateřskou školu jich v průměru navštěvovalo 25. Děti trávily v mateřské škole celý den. Na tuto třídu byla jedna paní učitelka a~to od pondělí do pátku po celou otevírací dobu mateřské školy a~s nikým se nestřídala. 

		\subsection{Tělocvična}
			Tato mateřská škola měla velmi prostornou tělocvičnu rozdělenou na dvě části. První část sloužila jako sklad náčiní a~v~druhé části se konala výuka. Tělocvična byla společná i~pro základní školu. Nabídka náčiní byla pestrá a~bohatá. Tělovýchovná chvilka byla podle rozvrhu zařazena do programu každý den na 30 minut. Tato chvilka byla však zařazena hned po dílnách, a~tak se pravidelně stávalo, že se kvůli prodloužení aktivit děti do tělocvičny vůbec nedostaly a~navštívily ji v~průměru jednou až dvakrát za týden. 

		\subsection{Hygienické zázemí}
		\label{zachody}
			Pro všechny třídy na patře se nachází jedna místnost se záchody, mušlemi a~kruhovou fontánou, která slouží jako umyvadlo. Toalety jsou odděleny přepážkou (Obr.~\ref{Obr19}). Když děti potřebují, dovolí se paní učitelky a~na záchod chodí samy. U~menších dětí je doprovází asistent, je-li ve třídě. 
	

%!!!!!!!!!!!!!!!!!!!!!!!!!!!!!!!!!!!!!!!!!!!!!!!!!!!!!!!!!!!!!!!!!!!!!!!!!!!!!!!!!!!!!!!!!!!!!!!!!!!!!!!


\section{Režim dne v~české mateřské škole}

		Praxe v~České republice se konala ve fakultní mateřské škole Sluníčko pod střechou při Pedagogické fakultě UK (Mohylová 1964, 155 00 Praha 5) v~období od 19. 10. 2008 do 23. 10. 2008. 

	\subsection{Průběh dne a~jeho specifika}

			Mateřská škola je otevřena od 6:30 do 18:00 každý všední den. Časový harmonogram (viz tabulka~\ref{tab:rezimDneCR})
			je spíše orientační, lze ho přizpůsobit momentální situaci i~individuálním potřebám dětí. Mezi 6:30 a~7:30 se děti scházely v~jedné společné třídě, poté si je vyučující odvedly do kmenových tříd. Večer do zavírací doby se děti, které odcházejí pozdě domů, opět scházely v~jedné třídě.

		\begin{spacing}{1.0}
		\begin{table}[h!]
			\center
			\begin{tabular}{|l l|}
				\rowcolor{white}
				\hline
			6:30 – 9:00				& Scházení dětí, ranní hry a~činnosti dle volby dětí 	\\ 
									& Individuální práce, individuální péče o~děti\\
									& Jazykové chvilky, ranní kruh 	\\
									& Ranní cvičení, popř. relaxační cvičení, joga \\
			9:00 – 9:20				& Hygiena, svačina	\\
			9:20 – 11:45			& Plnění činností rozvíjející smyslové, manipulačně-	\\
									& technické, sebeobslužné, tělesné, estetické \\
									& a~mravní stránky osobnosti dítěte (záměrné i~spon- \\
									& tánní učení) ve skupinách i~individuálně \\
									& Pobyt venku					\\
			11:45 – 13:00			& Hygiena, oběd, příprava na odpočinek					\\
			13:00 – 14:30			& Odpočinek dle věku dětí						\\
									& Náhradní nespací aktivity dle věku dětí 				\\
									& Péče speciálně pedagogická, logopedická				\\ 
									& Nadstandardní aktivity 							 \\
			14:30 – 15:00			& Tělovýchovná chvilka, hygiena, svačina 			\\
									& Nadstandardní aktivity 		\\
			15:00 – 18:00			& Odpolední hry a~zájmové činnosti, \\
									& Individuální práce s~dětmi\\
			\hline
			\end{tabular}
			\caption{ \textbf{Časový harmonogram v~české MŠ}
			}
			\label{tab:rezimDneCR}
		\end{table}
		\end{spacing}

		\subsection{Příchod dětí do mateřské školy}
			
			Děti byly přijímány od 6:30 do 8:00. Rodiče pomohly dětem s~přezutím a~převlečením. V~šatně mělo každé dítě svoji skříňku a~přihrádku označenou specifickým symbolem. Rodiče jsou zodpovědní za osobní předání dítěte vyučujícímu. Při příchodu se vyučující s~dítětem pozdravil a~pomohl mu najít aktivitu nebo nechal dítě volně si vybrat, co bude dělat. Pozdní příchody nebo absence se hlásí telefonicky předem nebo do 8 hodin. Příchod mimo domluvený čas je možné dohodnout s~vyučujícím a~přizpůsobit potřebám rodičů.


		\subsection{Volná hra}

			Volnou hru nebo také volné činnosti si dítě může volit samo. Je na dítěti samotném, čím se zaměstná, jestli si bude hrát samo, nebo někoho přizve, či se k někomu připojí. Projevuje svou vlastní aktivitu a~učitel zde hraje roli podpůrnou a~motivační, ale neurčuje, co má dítě dělat. 

		\subsection{Ranní kruh}

			Ranní kruh se konal chvíli poté, co se ve třídě sešly všechny děti. Děti i~vyučující si sedli vedle sebe do kruhu na koberec. Tvar kruhu zajišťuje všem rovnou pozici. Vyučujísí si s~dětmi povídala o~tom, co zažily o~víkendu nebo si připomněly nějakou básničku. Na konci kruhu vysvětlila motivačním způsobem, co děti čeká za aktivitu. 

		\subsection{Řízené činnosti}
		 	V~době před svačinou se v~této mateřské škole věnoval čas spíše tělesným chvilkám a~hrám. Po svačině se konaly aktivity rozvíjející smyslové, manipulačně technické, estetické dovednosti apod. Jednalo se jak o~vyplňování grafomotorických listů, tak výtvarné aktivity, skládání puzzle, stříhání, lepení atd. 

		\subsection{Pobyt venku}
			Ve zmíněné mateřské škole je pobytu venku věnován minimální čas dvě hodiny. Doba strávená venku je ovlivněna meteorologickými podmínkami. Děti měly možnost využívat celý prostor rozlehlé zahrady, která byla bohatě vybavena. Nacházely se zde 4 pískoviště, prolézačky, houpačky, dřevěný domek, menší tabule pro kresbu křídou, zahradní domek. Děti měly k~dispozici též hračky na písek a~dětské zahradní nářadí. Mohly se starat i~o~menší bylinkovou zahrádku. Tato školka měla možnost využívat i~nedalekého místního sportovního hřiště. Odchod zpět do tříd a~řazení dětí u~dveří pavilonů byl využit k~jednoduchým opakováním probírané látky nebo řešením jednoduchých problémů. Dítě, které vědělo odpověď, bylo puštěno dovnitř a~vyučující měla více času na jejich spočítání.

		\subsection{Strava a~pitný režim}
			Mateřská škola měla svou vlastní kuchyň, která donášela dětem jídlo přímo do tříd. Bylo zde dbáno na vyváženou stravu, jednu svačinu dopoledne, oběd a~jednu svačinu odpoledne. Stravovací režim byl tedy nedílnou součástí programu dne. Na pitný režim byl dáván důraz. Po celý den bylo v~konvi připravené pití a~děti se mohly kdykoliv napít. Pití se dětem podávalo i~ke každému jídlu.

		\subsection{Odpolední klid}
			Děti, které neodcházejí domů již po obědě, dodržují odpolední klid. Do třídy se scházely i~děti z~jiné třídy. Některé děti spaly na připravených matracích s~pokrývkou. Před spaním se jim četla kniha. Děti, které spát nechtěly, dodržovaly klidový režim, mohly ležet a~číst si nebo se věnovaly jiným klidovým aktivitám, jako je kreslení, puzzle, pexeso a~další. Předškoláci mají v tuto dobu předškolní přípravu. 


		\subsection{Odchod dětí z~mateřské školy}
			Rodiče si mohly děti vyzvedávat po obědě od 12:30 do 13:00 a~odpoledne od 14:30 kdykoliv do zavírací doby. Rodičům byl dán velký prostor, kdy si mohou pro svoje děti přijít, podle jejich potřeb a~možností. Rodiče si děti vyzvedávají ve třídě nebo na zahradě. 

		\subsection{Popis budovy mateřské školy}

			Tuto mateřskou školu tvořily tři propojené jednopatrové pavilony, v~každém pavilonu se nacházely dvě třídy. V~prostředním pavilonu se nacházela vlastní kuchyně. K~objektu patřila velká zahrada, na které se nacházely 4 menší pískoviště, dřevěné domečky a~prolézačky se skluzavkou, kolotoč, moderní kruhová houpačka pro více dětí a~lavičky. Na zahradě se nacházel i~zahradní domek, kde se uskladňovalo dětské zahradní nářadí, hračky na zahradu a~pískoviště, míče apod. Byla zde k~nalezení i~tabule a~křídy na psaní či kreslení. V~jedné části zahrady se společnými silami pěstovaly bylinky. Zvláštností bylo mlhoviště, místo, kde se při velkém vedru rozprašovala voda. 

%TODO FOTKYYYYYY !!!!!!!!!!!!!!!!!!

		\subsection{Popis třídy a~materiálního vybavení}

			Před vstupem do třídy se nacházela prostorná šatna, kde mělo každé dítě svoji skříňku se značkou. Do třídy se vchází přes hygienické zázemí. Třída samotná byla prostorná a~rozdělená na dvě části. V~první části byly kruhové stolky s~různě vysokými židličkami podle potřeb dětí. Podél zdí byly postaveny zavírací skříně na materiál a~didaktické pomůcky a~otevřené police. V~této části třídy se nacházel i~stůl pro vyučujícího, magnetická tabule, tabule na výtvarné práce dětí a~klavír. Na podlaze bylo položeno linoleum. Druhá část třídy byla menší a~na zemi byl položen koberec. Na stěně byly zavěšeny žebřiny a~po obvodu byly krabice s~hračkami a~stavebnicemi. Z~této části se dalo vstoupit do kabinetu na pomůcky.

		\subsection{Počet dětí a~pedagogické zastoupení}

			V~této třídě bylo zapsáno 27 dětí, v~době praxe vzhledem k~podzimnímu počasí docházelo v~průměru 22. V~průběhu dne se střídaly dvě učitelky. Jedna měla ranní a~druhá odpolední službu.
			Jednalo se též o~homogenní třídu, byly zde děti ve věku 4 - 5 let.
		
		\subsection{Tělocvična}
			Tělocvična byla samostatná místnost plně vybavena tělocvičným nářadím i~moderními pomůckami. Na zemi byl položen koberec. Z~důvodu malého a~prostorem nevyhovujícího kabinetu na pomůcky,  byly míče zavěšeny v~síti u~stropu.

		\subsection{Hygienické zázemí}
			Hygienické zázemí se nacházelo hned za vstupem ze šatny. Jednalo se o~menší příjemný a~barevně laděný prostor. Záchody pro menší děti byly společné, větší děti je měly oddělené přepážkou pro větší soukromí. Nacházelo se zde dostatek umyvadel a~dbalo se na čištění zubů po obědě. Záchody byly vždy společné pro dvě třídy. 

\section{Srovnání režimu mateřské školy ve Francii a~České republice} 
\label{srovnani} 

\begin{table}[h]
	\center
	\begin{tabular}{|l|l|l|}
	\hline
	\rowcolor{grey}
								& \textbf{Francie}				& \textbf{Česká republika}	\\
	\hline
	\hline
	%=========================================================================
\rowcolor{grey!10}	 příchod dětí do MŠ			& 8:50 - 9:10				& 6:30 - 8:00			\\ 
\rowcolor{grey!50}	 čas na volnou hru 			& 20min 					&30min - 2h 	\\ 
\rowcolor{grey!10}	 čas na rituály 			&10min 						&10-15min \\
\rowcolor{grey!50}	 čas na řízené aktivity		&5h30 						&2-3h   \\ 
\rowcolor{grey!10}	 čas na pobyt venku     	&2x30min během dne			&min.2h 	\\ 
\rowcolor{grey!10}								&a 1h po obědě				& \\ 
\rowcolor{grey!50}	 čas na stravu				&40min						&1h \\
\rowcolor{grey!10}	 čas na odpolední klid 	 	&1h 						&1h30 	\\
\rowcolor{grey!50}	 odchod dětí z~MŠ			&16:30						&12:30-13:00, 14:30-18:00	\\														 
	 \hline
	  
	%=========================================================================
	\end{tabular}
	
	\caption{ \textbf{Srovnání časové dotace na aktivity ve Francii a~České republice} Tabulka znázorňuje kolik času mají obě MŠ vyhrazeno na dané aktivity.
	}
	\label{srovnanirezimdne}
\end{table}


	Na začátku kapitoly již bylo zmíněno, že se nejedná o~kvantitativní studii. Dvě třídy mateřské školy, jedna francouzská a~jedna česká, reprezentují dvě kazuistiky. Proto nelze stanovovat obecné závěry, ale tyto kazuistiky umožňují odhalovat shody a~rozdíly. To je pak dostačující při použití komparativní metody. Přestože tyto kazuistiky reprezentují jednotlivé příklady praxe, nejsou reprezentativním vzorek pro celou sledovanou zemi. Následující srovnání lze považovat za nahlédnutí do systému obou zemí a~přiblížení organizační struktury a~zázemí mateřských škol.

	Tabulka~\ref{srovnanirezimdne} shrnuje časový harmonogram režimu dne mateřských škol. Již první bod, příchod dětí do MŠ, ukazuje rozdíly. Ve Francii platí tento čas ve všech mateřských školách, může se lišit v~odstupu několika minut. Čas příchodu do MŠ kopíruje čas příchodu a~začátek vyučování primární školy. Oproti tomu v~České republice si čas příchodu upravuje každá mateřská škola zvlášť. Některé školy, jako zmiňovaná česká MŠ, vycházejí vstříc pracujícím rodičům a~přijímají děti již od ranních hodin.

	Dalším sledovaným aspektem je čas věnovaný volné hře. Po příchodu do MŠ mají děti ve Francii přibližně 20 minut volného času. Tento čas se zdál autorce nedostatečný, většina dětí se sotva stihla \uv{rozkoukat} a~začít si samostatně nebo s~kamarády hrát a~už musely jít \uv{pracovat}, tj. podřídit se řízenné činnosti. V~České republice čas na volnou hru vyplývá z~příchodu dětí. Děti, které jsou v~MŠ od brzkých ranních hodin mají více času na volnou hru. Naopak děti, které přicházejí později, jsou na tom podobně jako děti v~uváděné francouzské MŠ. 

	Další čas pro volnou hru dětí je při pobytu venku. Celková doba strávená dětmi venku je v~obou zemích téměř totožná, přestože je v~režimu dne zařazena jindy a~v~jiných časových blocích (vit tabulka \ref{srovnanirezimdne}).

	Čas věnovaný ranním rituálům byl v~obou zemích totožný. 

	Z~tabulky \ref{srovnanirezimdne} je patrné, že celkový čas na řízené aktivity ve Francii je výrazně vyšší než v~České republice. Hlavní náplní francouzských mateřských škol jsou řízené aktivity. Čas vyhrazený na tyto aktivity v~České republice je výrazně kratší a~to proto, že hlavní část řízených aktivit probíhá dopoledne tak, aby se jich mohly zúčastnit všechny děti včetně těch, které odcházejí domů již po obědě. Děti, které zůstávají i~odpoledne, mají řízených aktivit více. 

	Mezi řízené aktivity patří i~ranní rituály, které jsou uvedeny zvlášť, protože se jedná o~pravidelnou opakovanou činnost, která se koná v~obou zemích. Ve Francii jsou přímo nazývany \uv{rituály}, v~České republice je tento rituál nazýván \uv{ranní kruh}. Čas věnovaný těmto rituálům je v~obou zemích totožný.

	Děti v~obou mateřských školách pobývají venku stejný čas, v~každé zemi je však pobyt venku rozdělen do různě dlouhých bloků během dne. 

	Žádná z~mateřských škol neměla po obědě řízené aktivity. Francouzské děti měly cca 1 hodinu volna, kterou trávily venku a~děti v~České republice chodily spát nebo se věnovaly klidovým aktivitám. 

	Odchod dětí ze třídy je ve Francii pro všechny stejný. Toto je platné pro všechny mateřské školy v~zemi. V~České republice si rodiče mohou vyzvednout děti buď po obědě nebo po klidové části dne až do zavírací doby. Otevírací doba, stejně jako doba pro vyzvedávání, je upravována každou mateřskou školou.

	Celkový čas předškolního vzdělávání ve francouzské MŠ kopíroval 24 hodinovou týdenní dotaci výuky na primární škole. Časový harmonogram je schématicky vyjádřen v~příloze tab. \ref{tabulkaMS} a~\ref{tabulkaMS2} a~je uveden pro dokreslení tohoto tvrzení.

%///////////////////////////////////////////////////////////////////////////////////////////////////

\begin{table}[h]
	\center
	\begin{tabular}{|l|l|l|}
	\hline
	\rowcolor{grey}
								& \textbf{Francie}				& \textbf{Česká republika}	\\
	\hline
	\hline
	%=========================================================================
\rowcolor{grey!10}	 aktivity při volné hře	&puzzle, malování,čtení 	&stavebnice, hračky, kostky,\\ 
\rowcolor{grey!10}	 						&							&kočárky, malování, čtení \\ 
\rowcolor{grey!50}	 řízené aktivity  		&\uv{práce}-grafomotorika, 	&\uv{hra}-výtvarné, hudební, \\ 
\rowcolor{grey!50}	 						&stříhání, lepení, skládání,&pohybové, námětové hry, \\ 
\rowcolor{grey!50}	 						&pregramatické činnosti 	&grafomotorika, \\
\rowcolor{grey!10}	 aktivity při rituálech &docházka, datum, říkanka, 	&ranní povídání, vyprávění,\\ 
\rowcolor{grey!10}	 						&vysvětlení dílen			& motivace k~další činnosti\\ 
\rowcolor{grey!50}   venkovní aktivity 		& běhání, stání, míčové		& prolézačky, houpačky, \\
\rowcolor{grey!50}							&hra, jízda na kolech, 		&běhání, hrabání listí,	\\
\rowcolor{grey!50}							&hrabání listí				&starání se o~zahrádku	\\
	 \hline
	%=========================================================================
	\end{tabular}
	
	\caption{ \textbf{Srovnání aktivit realizovaných v~MŠ Francie a~České republiky}
	}
	\label{srovnaniaktivit}
\end{table}
	
	Tabulka~\ref{srovnaniaktivit} znázorňuje aktivity, které jsou rozděleny podle hlavních částí režimu dne. Nejedná se o~podrobný a~vyčerpávající vzorek, ale o~názorný přehled činností jednoho běžného dne. 

	Přestože je volné hře věnován v~obou zemích téměř stejný čas, děti v~České republice mají oproti Francii mnohem širší výběr deskových her, stavebnic, panenek, plyšových hraček, námětových koutků apod. To podporuje názor prezetovaný v~kapitole~\ref{srovnanikurikulfrcr}, že v~České republice je na dítě nahlíženo jako na \uv{dítě} a~jeho dětství je podporováno jak možností výběru herních prvků, tak i~většími prostory, kde se hra odehrává. 

	Rituály v~České republice neboli ranní kruh měl spíše ráz příjemné chvilky, kdy si všichni povídali a~fungoval jako úvod dne. Ve Francii se už během každodenního rituálu pracuje, zjištuje se docházka, píše se datum a~vysvětlují se charakteristiky dílen.

	Obsahová stránka řízených aktivit je v~mnohém podobná. Nejmarkantnější rozdíl byl v~práci s~dětmi. V~České republice si děti \uv{hrají} nebo se \uv{učí něčemu novému}. Velká část vzdělávacích aktivit má podobu hry, zatímco ve Francii děti \uv{pracují} a~plní úkoly, učí se, ale nehrají si. Ve Francii je kladen velký důraz na rozvoj jazykových schopností, který vyplývá z~rozdílnosti mluvené a~psané formy jazyka a~na rozvoj psaní. Děti se učí tiskacím písmenům a~svému podpisu. Toto dokládá tvrzení z~kapitoly~\ref{srovnanikurikulfrcr}, že ve Francii je na dítě nahlíženo jako na žáka.

	I~přes časovou shodu pobytu venku, se využití této doby v~obou zemích výrazně liší. Francouzské děti měly k~dispozici mnohem méně herních prvků. V~České republice měla pozorovaná mateřská škola velmi dobře vybavenou zahradu. Ne všechny mateřské školy disponují takto velkými a~vybavenými zahradami, přesto je zvykem, že mateřské školy mají zahradu vybavenou herními prvky. Ve Francii je oproti tomu velmi časté, že mateřské školy zahrady nemají vůbec a~děti si hrají na dvorech. Tento popis venkovních prostor dokládá, že volné hře a~jejím podmínkám je v~České republice věnována větší pozornost.



		
	



\chapter*{Závěr}
\addcontentsline{toc}{chapter}{Závěr} 
Hlavním tématem této práce byla komparace vybraných aspektů současné francouzské mateřské školy a české mateřské školy. V teoretické části práce byly zpracovány základní informace o zařazení mateřských škol do vzdělávacího systému. Byly popsány podmínky péče o předškolní děti, a přiblížili jsme čtenáři rozdílné pojetí dítěte, které se odráží i ve vybraných srovnávaných aspektech, které byly sledovány v praktické části. Cílem komparace bylo zjistit rozdílnost v cílech, pojetí a obsahu vzdělávání, stejně jako přiblížit a porovnat časový harmonogram mateřských škol obou zemí, a podmínky, ve kterých se předškolní vzdělávání odehrává.  

Ke komparaci byly vybrány dva aspekty: kurikulum a režim dne. Informace o kurikulu francouzské mateřské školy nebyly doposud českému čtenáři k dispozici v českém překladu. Autorka využila svých jazykových i studiem získaných odborných znalostí k autorskému překladu vzdělávacích oblastí legislativního dokumentu Francie. Pro potřeby této práce se nejedná o úplný překlad, ale o stručnější verzi nejdůležitějších sledovaných bodů. 

Autorce se jeví, že pojetí dítěte se výrazně projevuje ve všech sledovaných a popsaných aspektech. Ať již vycházíme z rozdílných ekonomických podmínek a výrazně rozdílné délky mateřské dovolené, tak zejména v rozdílném uchopení cílů vzdělávání, uvedených v legislativních dokumentech, tedy i v autorkou sledovaném kurikulu a režimu dne. 

Jak je blíže popsáno v 2. kapitole \uv{Pojetí dítěte ve Francii a České republice}, rozdílnost v pojetí dítěte vidí autorka zejména v tom, že ve Francii se dívají na dítě v mateřské škole jako na žáka, zatímco v České republice se na něj díváme jako na dítě.   

Tyto rozdíly jsou zřetelně viditelné v kurikulu. Hlavním vzdělávacím cílem ve francouzském dokumentu je připravit dítě na úspěšné zvládnutí role žáka v primární škole. Dokladem toho je, že součástí francouzského kurikula je vzdělávací oblast s názvem \uv{Stát se žákem}. Na konci mateřské školy ve Francii by děti měly umět psát velkými tiskacími písmeny. V rámci nácviku kopírují slova z předloh. Velký důraz je kladen i na rozvoj jazykových dovedností dětí. Zatímco v České republice se jedná o přípravu na psaní, tj. uvolňování ruky, držení pera, tužky, rozlišování tvarů a forem. V České republice žádný cíl, který by odpovídal francouzskému pojetí \uv{stát se žákem}, není. V České republice je dáván velký důraz na dětské prožívání, vlastní kreativitu a socializaci, které vyplývají z cílŮ RVP PV. Dětem je dáván dostatek volného prostoru a jsou vytvářeny i materiální a prostorové podmínky pro volnou hru. Hra je nedílnou součástí vzdělávacího programu v české mateřské škole. 

Pojetí dítěte se projevuje i v časovém harmonogramu. Francouzský časový harmonogram kopíruje harmonogram školy primární (viz příloha \ref{!!!!}. Děti předškolního věku tedy tráví v mateřské škole podobný čas jako děti školního věku na primární škole. Rozložení aktivit během dne a jejich délka i obsah tomu také napovídají. V České republice je časový harmonogram také daný, ale jeho dodržování je mnohem flexibilnější a dětem je dáván dostatek prostoru na vlastní prožívání, zkoumání a experimentování. Kolik času denně  děti stráví v mateřské škole je v rukou rodičů, na nich záleží, jestli budou v mateřské škole celý den nebo třeba jen půlden. To může být kulturním důvodem, proč je hlavní část řízených aktivit zařazena v dopoledních hodinách, tedy tak, aby se jich mohly účastnit všechny děti. 

Tyto závěry potvrzují hypotézu č.1, že kurikulum předškolní výchovy obou zemí má rozdílné cíle a pojetí předškolního vzdělávání. Zjednodušeně lze říci, že ve Francii připravuje předškolní vzdělávání dítě na školu, kdežto v České republice na život. Tyto závěry současně vyvrací hypotézu č.2, že v obou zemích mají stejný přístup k dítěti. Ve Francii nahlíží na dítě jako na žáka a v České republice jako na dítě. 

Možnou inspirací pro český vzdělávací systém by mohlo být francouzské rozdělení vzdělávání do cyklů. Poslední třída mateřské školy je zároveň první třídou druhého vzdělávacího cyklu viz \ref{!!!!!}. Toto rozdělení usnadňuje plynulý přechod z mateřské školy na školu základní. 

Česká republika by mohla být Francii inspirací tím, že ve vzdělávacím systému respektuje neopakovatelné období dětsví. Hlavní součástí péče o toto období je jeho zakotvení v kurikulárním dokumentu. Jak již bylo zmíněno, české děti mají více prostoru k volné hře, hře jako takové, vlastnímu prožívání a experimentování, navazování a rozvíjení sociálních vazeb. V kurikulu jsou učitelům také dávány návody, jakými vhodnými způsoby mohou tyto oblasti rozvíjet. V českých mateřských školách je i prostředí a materiálové vybavení více přizpůsobeno k tomu, aby si děti mohly hrát.

Z pohledu budoucí učitelky v mateřské škole, jsem si díky praxi uvědomila, že to, co jsem považovala za běžné a normální v českých mateřských školách, nemusí být samozřejmostí. Jsem si vědoma i toho, že můj osobní pohled při pozorování a srovnávání mohl zabarvit mé závěry. Tato zkušenost mě však utvrdila v mém přesvědčení, že vzdělávání v mateřských školách v České republice má celou řadu pozitiv jak pro děti, tak pro profesi učitele. 

%%% Seznam použité literatury
%\include{literatura}
\nocite{Eurydice2}
\nocite{Eurydice3}
\nocite{EAJE}
\nocite{fredu}
\nocite{Prucha12}
\addcontentsline{toc}{chapter}{Literatura}
\bibliographystyle{csplainnat}
\bibliography{reference}


%%% Tabulky v bakalářské práci, existují-li.
%\chapwithtoc{Seznam tabulek}

%%% Použité zkratky v bakalářské práci, existují-li, včetně jejich vysvětlení.
%\chapwithtoc{Seznam použitých zkratek}

%%% Přílohy k bakalářské práci, existují-li (různé dodatky jako výpisy programů,
%%% diagramy apod.). Každá příloha musí být alespoň jednou odkazována z vlastního
%%% textu práce. Přílohy se číslují.
%wtf
\addcontentsline{toc}{chapter}{Seznam obrázků} 
\listoffigures
\addcontentsline{toc}{chapter}{Seznam tabulek}
\listoftables 
\chapwithtoc{Přílohy}
\setcounter{page}{1}  % nastaví čítač stránek znovu od jedné
\pagenumbering{Roman} % číslování římskými číslicemi
	\begin{figure}[tb]
		\centering
		\includegraphics[height = 0.35\textheight]{./fotky/Obr1.jpg}
		\caption{
			Tabule s návody, číslicemi, datem a popisem dílen, prostor pro ranní kruh (viz.~\ref{sec:tridaVybaveni}).
		}
		\label{Obr1}
	\end{figure}

	\begin{figure}[tb]
		\centering
		\includegraphics[height = 0.35\textheight]{./fotky/Obr2.jpg}
		\caption{
			Stolky s počítači a vchod na dvůr (viz.~\ref{sec:tridaVybaveni}).
		}
		\label{Obr2}
	\end{figure}

	\begin{figure}[tb]
		\centering
		\includegraphics[height = 0.35\textheight]{./fotky/Obr3.jpg}
		\caption{
			Literární koutek s křesílkem (viz.~\ref{sec:tridaVybaveni}).
		}
		\label{Obr3}
	\end{figure}

	\begin{figure}[tb]
		\centering
		\includegraphics[height = 0.35\textheight]{./fotky/Obr4.jpg}
		\caption{
			Umyvadlo a police na materiál (viz.~\ref{sec:tridaVybaveni}).
		}
		\label{Obr4}
	\end{figure}

	\begin{figure}[tb]
		\centering
		\includegraphics[height = 0.35\textheight]{./fotky/Obr5.jpg}
		\caption{
			Stůl pro učitelku, dětská kuchyňka a poličky na didaktické pomůcky (viz.~\ref{sec:tridaVybaveni}).
		}
		\label{Obr5}
	\end{figure}

	\begin{figure}[tb]
		\centering
		\includegraphics[height = 0.35\textheight]{./fotky/Obr6.jpg}
		\caption{
			Výzdoba třídy písmeny abecedy (viz.~\ref{sec:tridaVybaveni}).
		}
		\label{Obr6}
	\end{figure}


	\begin{figure}[tb]
		\centering
		\includegraphics[height = 0.35\textheight]{./fotky/Obr7.jpg}
		\caption{
			Police na materiál a pomůcky (viz.~\ref{sec:tridaVybaveni}).
		}
		\label{Obr7}
	\end{figure}

	\begin{figure}[tb]
		\centering
		\includegraphics[width = 0.45\linewidth]{./fotky/Obr8a.jpg}
		\includegraphics[width = 0.45\linewidth]{./fotky/Obr8b.jpg}
		\caption{
			Vyvěšená pravidla třídy (viz.~\ref{pravidlaChovani}).
		}
		\label{Obr8}
	\end{figure}

	\begin{figure}[tb]
		\centering
		\includegraphics[height = 0.35\textheight]{./fotky/Obr9.jpg}
		\caption{
			Šatna (viz.~\ref{prichod}).
		}
		\label{Obr9}
	\end{figure}



	\begin{figure}[tb]
		\centering
		\includegraphics[height=0.35\textheight]{./fotky/Obr10.jpg}
		\caption{
			Prezence a rozdělení do skupin na dílny (viz.~\ref{prichod}).
		}
		\label{Obr10}
	\end{figure}

	\begin{figure}[tb]
		\centering
		\includegraphics[height=0.35\textheight]{./fotky/Obr11.jpg}
		\caption{
			Detailní pohled na tabuli, psaní data, psaní číslic, charakteristika dílen s týdenní nabídkou  (viz.~\ref{ritualy}).
		}
		\label{Obr11}
	\end{figure}

	\begin{figure}[tb]
		\centering
		\includegraphics[height=0.35\textheight]{./fotky/Obr12.jpg}
		\caption{
			Kalendář, měsíce v roce a roční období (viz.~\ref{ritualy}).y
		}
		\label{Obr12}
	\end{figure}

	\begin{figure}[tb]
		\centering
		\includegraphics[height=0.35\textheight]{./fotky/Obr13.jpg}
		\caption{
			Stoly na práci při dílnách (viz.~\ref{ateliery}).
		}
		\label{Obr13}
	\end{figure}
	\begin{figure}[tb]
		\centering
		\includegraphics[height=0.35\textheight]{./fotky/Obr14.jpg}
		\caption{
			Pohled na třídu (viz.~\ref{ateliery}).
		}
		\label{Obr14}
	\end{figure}

	\begin{figure}[tb]
		\centering
		\includegraphics[height=0.35\textheight]{./fotky/Obr15.jpg}
		\caption{
			Sešit, kam si dětí lepí své práce = základ portfolia (viz.~\ref{ateliery}).
		}
		\label{Obr15}
	\end{figure}

	\begin{figure}[tb]
		\centering
		\includegraphics[height=0.35\textheight]{./fotky/Obr16.jpg}
		\caption{
			Pohled na dvůr s prolézačkou (viz.~\ref{prestavka}).
		}
		\label{Obr16}
	\end{figure}

	\begin{figure}[tb]
		\centering
		\includegraphics[height=0.35\textheight]{./fotky/Obr17.jpg}
		\caption{
			Pohled na dvůr, odstrkovadla a domek pro děti (viz.~\ref{prestavka}).
		}
		\label{Obr17}
	\end{figure}

	\begin{figure}[tb]
		\centering
		\includegraphics[height=0.35\textheight]{./fotky/Obr18.jpg}
		\caption{
			Místnost na spaní (viz.~\ref{spani}).
		}
		\label{Obr18}
	\end{figure}
	
	\begin{figure}[tb]
		\centering
		\includegraphics[height=0.35\textheight]{./fotky/Obr19.jpg}
		\caption{
			Hygienické zázemí (viz.~\ref{zachody}).
		}
		\label{Obr19}
	\end{figure}
\begin{landscape}
\begin{table}
	\shorthandoff{-}
	\begin{tabular}{
	|p{10mm}
	|p{10mm}
	|p{23mm}
	|p{8mm}
	|p{15mm}
	|p{27mm}
	%--------------------------
	|p{18mm} %mezera
	|p{30mm}
	|p{15mm}
	|p{15mm}
	|p{23mm}
	|p{7mm}
	|p{0mm}}
		\multicolumn{1}{l}{\hspace{-5mm}8h20} & 
		\multicolumn{1}{l}{\hspace{-5mm}8h40} &  	% 20min
		\multicolumn{1}{l}{\hspace{-5mm}9h00} &  	% 20min 	
		\multicolumn{1}{l}{\hspace{-5mm}9h45} & 	% 45min
		\multicolumn{1}{l}{\hspace{-8mm}10h00} & 	% 15min
		\multicolumn{1}{l}{\hspace{-8mm}10h30} &  	% 30min
		\multicolumn{1}{l}{\hspace{-8mm}11h30} &  	% 60min
		%-------------------------------------------------------
		\multicolumn{1}{l}{\hspace{-8mm}13h30} &  	% mezera
		\multicolumn{1}{l}{\hspace{-8mm}14h30} &  	% 60min
		\multicolumn{1}{l}{\hspace{-8mm}15h00} & 	% 30min
		\multicolumn{1}{l}{\hspace{-8mm}15h30} &  	% 30min
		\multicolumn{1}{l}{\hspace{-8mm}16h15} &  	% 45min
		\multicolumn{1}{l}{\hspace{-8mm}16h30}  \\ 	% 15min
		\cline{1-6}\cline{8-12}
		 & & & & \cellcolor{gray!50} & & & & &  \cellcolor{gray!50}& & &
		 \multicolumn{1}{l}{} \\ [15mm]
		 \cline{1-6}\cline{8-12}
	\end{tabular}
	\hfill
	\caption{\textbf{A day by the very young pupil's.}}
\end{table}
% \end{landscape}

% \begin{landscape}
\begin{table}
	\shorthandoff{-}
	\begin{tabular}{
	 p{1mm} 	% invisible column
	|p{6mm}
	|p{20mm}
	|p{25mm}
	|p{7mm}
	|p{30mm}
	%---------------------------------
	|p{1mm} 	% mezera
	|p{5mm} 	% obed
	|p{1mm} 	% mezera
	%----------------------------------
	|p{23mm}
	|p{30mm}
	|p{8mm}
	|p{23mm}
	|p{8mm}
	|p{8mm}}
		\multicolumn{1}{l}{} &
		\multicolumn{1}{l}{\hspace{-5mm}8h30} & 	%
		\multicolumn{1}{l}{\hspace{-5mm}8h40} &  	% 10in
		\multicolumn{1}{l}{\hspace{-5mm}9h15} & 	% 35min
		\multicolumn{1}{l}{\hspace{-9mm}10h00} &  	% 45min
		\multicolumn{1}{l}{\hspace{-7mm}10h15} &  	% 15min
		%--------------------------------------------------
		\multicolumn{1}{l}{\hspace{-8mm}11h30} &  	% 75min
		\multicolumn{1}{l}{\hspace{0mm}12h00} &
		\multicolumn{1}{l}{} &
		%--------------------------------------------------
		\multicolumn{1}{l}{\hspace{-6mm}13h30} & 	%
		\multicolumn{1}{l}{\hspace{-8mm}14h15} & 	% 45min
		\multicolumn{1}{l}{\hspace{-8mm}15h15} &  	% 60min
		\multicolumn{1}{l}{\hspace{-8mm}15h30} & 	% 15min
		\multicolumn{1}{l}{\hspace{-8mm}16h15} &  	% 45min
		\multicolumn{1}{l}{\hspace{-8mm}16h30}  \\ 	% 15min
		\cline{2-6}\cline{8-8}\cline{10-14}
		 & & & & \cellcolor{gray!50} & & & & & & &  \cellcolor{gray!50}& & &
		\multicolumn{1}{l}{} \\ [15mm]
		\cline{2-6}\cline{8-8}\cline{10-14}
	\end{tabular}
	\hfill
	\caption{\textbf{A day at the primary school.}}
\end{table}
\end{landscape}
\openright
\end{document}
