\chapter{CÍLE}

Cílem bakalářské práce je zjistit rozdílnosti v~cílech, pojetí a obsahu vzdělávání, stejně jako přiblížit a porovnat časový harmonogram mateřských škol obou zemí a podmínky, ve kterých se předškolní vzdělávání odehrává. 

Za tímto účelem byly vytčeny tyto \textbf{dílčí cíle}:

\begin{itemize}
	\setlength\itemsep{-2mm}
	\item [-] Analýza legislativních dokumentů věnující se předškolnímu vzdělávání ve Francii.
	\item [-] Analýza legislativních dokumentů věnující se předškolnímu vzdělávání v~České republice.
	\item [-] Analýza cílů, pojetí a vzdělávacích oblastí kurikul obou sledovaných zemí.
	\item [-] Pozorování průběhu dne ve francouzské mateřské škole.
	\item [-] Pozorování průběhu dne v~české mateřské škole. 
	\item [-] Komparace režimu dne v~mateřské škole obou sledovaných zemí.
\end{itemize}


Na základě otázek,které autorce vyvstaly při pozorování, byly stanoveny dvě \textbf{hypotézy}:

\begin{itemize}
\item[-] Hypotéza č. 1 - Kurikulum pro předškolní vzdělávání má rozdílné cíle a pojetí.
\item[-] Hypotéza č. 2 - V~obou zemích mají stejný přístup k~dítěti. 
\end{itemize}
