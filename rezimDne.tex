\chapter{REŽIM DNE}
\label{rezim}
	Režimem dne je v tomto případě myšlený časový rozvrh aktivit během dne. Čili kdy a jaké činnosti se v průběhu dne konají. Spadá sem i příchod a odchod dětí z mateřské školy. Tato kapitola vychází z volného pozorování během povinných praxí v rámci studia. Věnuji se zde i zázemí tříd daných mateřských škol obou srovnávaných zemí, které je na první pohled odlišné a které mi přišlo též důležité, protože dává představu o tom, kde a za jakých podmínek se den v mateřské škole odehrává. 

	\section{Režim dne ve francouzské mateřské škole}

		Možnost věnovat se pozorování průběhu dne ve francouzské mateřské škole jsem měla v rámci povinné praxe během výměnného studijního programu Erasmus. Stáž se konala ve státní mateřské škole na předměstí Paříže, která spolupracuje s Academie Versaille pod Université de Cergy-Pontoise. Mateřskou školu jsem navštěvovala každý den po dobu dvou týdnů od 9h do 16h30, tedy po celou otevírací dobu. Mateřská škola byla otevřena každý pracovní den kromě středy. 

		\subsection{Zázemí mateřské školy}

			Mateřská škola je součástí velké budovy, kde se nachází i škola základní. Děti mají přístup na dvůr, který má na většině plochy betonový povrch a na části speciální měkký povrch, dvě prolézačky, pískoviště. Děti mají na dvoře k dispozici odstrkovadla, tříkolky a míče. 
			Mateřská škola se nachází v přízemí budovy a má 8 tříd, z čehož dvě jsou v prvním patře, v prostorách 	základní školy. Dále se zde nachází knihovna, tělocvična, dvě ložnice na spaní, sborovna a jídelna. 
			Kapacita mateřské školy v roce 2010 byla 250 dětí, maximální kapacita dětí ve třídě je 30 dětí. Tato kapacita byla ve většině tříd naplněna.

		\subsection{Třída a její vybavení}
		\label{sec:tridaVybaveni}
			Třída, ve které jsem absolvovala praxi, byla nevelká místnost, plně zaplněná nábytkem a pomůckami. Na zemi byla dlažba, jen na malé části před tabulí byl položen koberec. Místnost měla dvoje dveře. Jedny vedly ze školní chodby, druhé přímo na dvůr školy. Na jedné straně třídy dominovala na zdi pověšená tabule, na které byly přilepené různé cedulky s návody, jak se píší číslice 1-9, další číslice až do 30, cedulky se dny v týdnu, měsíci, datem a popisem ateliéru (Obr.~\ref{Obr1}). Před tabulí byly do kruhu postaveny 3 lavice na sezení. Po obou stranách tabule byly umístěny stolky s dětskými počítači (Obr.~\ref{Obr2}). Uprostřed třídy bylo postaveno 5 stolů, každý s 6 židličkami. Na protilehlé straně třídy se nacházel čtenářský koutek s křesílkem, umyvadla, police na výtvarný materiál, stůl pro paní učitelku, dětská kuchyňka s popisky věcí, které patří do kuchyně (Obr.~\ref{Obr3},~\ref{Obr4},~\ref{Obr5}). Na této straně byla na zdi pověšená písmena abecedy velikost A3 (Obr.~\ref{Obr6}). Čtenářský koutek i kuchyňka byly od třídy odděleny různými skříňkami a poličkami, které sloužily k uskladnění didaktických pomůcek a her či sešitů dětí (Obr.~\ref{Obr7}).


			Třída byla velmi dobře vybavena, co se týče materiálu i didaktických pomůcek. Do relativně malého prostoru se vešlo všechno potřebné. Nicméně pro volnou hru a volný pohyb dětí příliš místa nezbývalo. Na volnou hru se dal využít menší prostor s kobercem před tabulí a v dětské kuchyňce. Dále už zbývala jen volnější plocha přede dveřmi na dvůr a ulička napříč třídou. Tento prostor byl pro 29 dětí velmi malý. 
			Děti mají na výběr různé stolní hry, puzzle, knihy. Hraček se tu nacházelo minimum, pár plyšáků, nějaké autíčko. Pokud si děti donesly nějakou hračku z domova, musely ji na začátku hodiny odložit na poličku, kde na ně čekala až do odchodu ze třídy. 

		\subsection{Počet dětí a pedagogické zastoupení}

			Měla jsem možnost být ve třídě “Grand section“, kde jsou děti ve věku 5 let, poslední rok před nástupem na základní školu. Ve třídě bylo zapsáno 29 dětí, mateřskou školu jich v průměru navštěvovalo 25. Děti trávily v mateřské škole celý den. Na tuto třídu byla jedna paní učitelka a to od pondělí do pátku po celou otevírací dobu mateřské školy. S nikým se nestřídala. 

		\subsection{Pravidla chování}
		\label{pravidlaChovani}
			Při pohledu na třídu upoutá pozornost červený a zelený papír velikosti A3, na kterém byly přilepeny obrázky s pravidly, které by se měly ve třídě dodržovat. Na pravidla se vyučující odkazovala téměř pokaždé, když byla některá z nich porušena. Většinou dítě, které nějakým způsobem pravidla nedodrželo, bylo vyzváno, aby ukázalo, o které pravidlo jde a povědělo všem, jak by se mělo chovat. 

% TODO: do dvou sloupcu
+	Papír se vyhazuje do koše
	Hlásit se
	Uklízet po sobě materiál
	Řadit se do řady
	Být potichu
	Udržovat stoly čisté
Říkat „Dobrý den“, „Na shledanou“, „Děkuji“

- 	Neprat se
	Neběhat po třídě
	Nestrkat se
	Nekřičet
	Neničit materiál
	Neschovávat věci
	Neříkat sprostá slova
	Nekrást
	Neobtěžovat kamarády
(Obr.~\ref{Obr8})

		\subsection{Průběh dne a jeho specifika}

			Časový harmonogram visí v tištěné podobě na dveřích každé třídy a je závazný. Vyučující se snaží dané časy striktně dodržovat.

% TODO: do tabulky, aby to bylo hezky
8:50 – 9:10 		Příchod dětí a jejich uvítání (volná hra, dokončování prací z minulého dne, úklid)
9:10 – 9:20		Rituály (pozdravení se, datum, počasí, představení ateliérů)
9:30 – 10:05		Ateliéry (grafomotorika/psaní, matematika, čtení)
10:05 – 10:15			Kruh (úklid, básničky/říkanky)
10:15 – 10:45			Přestávka
10:45 – 11:30			Lingvistické aktivity, společné čtení
11:30 – 11:50			Společný kruh (říkanky, matematické hry)
11:50 – 13:30			Oběd
13:30 – 13:55			Společný kruh (zpěv, hlasová cvičení, poslech)
13:55 – 14:30			Ateliéry (motorika, výtvarná výchova, objevování světa)
14:30 – 15:00			Tělocvična
15:00 – 15:30			Přestávka
15:30 – 16:10			Video nebo promítání diapozitivů
16:10 – 16:20			Úklid třídy, zhodnocení dne
16:20 – 16:30			Odchod dětí

			Zajímavostí francouzských mateřských škol je 4 denní týden. Děti navštěvují mateřskou školu jen v pondělí, úterý, čtvrtek a pátek. Ve středu jsou děti doma nebo mají volnočasové aktivity a sporty. Některé školy tyto aktivity nabízejí, jiné ne. 

		\subsection{Příchod dětí do třídy}
		\label{prichod}
			Brána školy je otevřena od 8h50 do 9h10 pro všechny žáky školy. Na chodbě před třídou má každé dítě svůj háček na pověšení oblečení a malou přihrádku na menší věci (Obr.~\ref{Obr9}). Děti se nepřezouvají, zůstávají celý den ve stejné obuvi, ve které přišly. Při vstupu do třídy vítá paní učitelka děti i jejich rodiče. S každým dítětem se snaží navázat kontakt, zeptá se ho, jak se má, apod. Každé dítě si poté najde na stole cedulku se svým jménem a přilepí ji na menší tabuli se suchým zipem vedle velké tabule (Obr.~\ref{Obr10}). Tímto je připravena docházka dětí, která je poté součástí ranního rituálu. Dále mají děti čas na volnou hru, malování, prohlížení knížek, skládání puzzle, ukončování výtvarných prací z minulého dne. Z rozvrhu dne je patrné, že na volnou hru je zde vyčleněn pouze čas, než se do třídy dostaví všechny děti. 
		

		\subsection{Rituály}
		\label{ritualy}
			Ke každodenním rituálům se děti scházejí před tabulí a sedají si na lavičky okolo koberce. Některé děti si z nedostatku míst sedají na koberec. Jedno vybrané dítě počítá kartičky se jmény děvčat, chlapců a kolik dětí je dohromady, poté tato čísla zapíše na tabuli. Další dítě má na starosti datum, nejdříve změní číslici dne a napíše na tabuli novou, a pokud se změnil měsíc, vymění kartičku s nápisem správného měsíce, poté datum přečte a celá třída po něm opakuje (Obr.~\ref{Obr11},~\ref{Obr12}). Dále všichni společně zarecitují uvítací říkanku (ta se liší třídu od třídy, v tomto případě to byla básnička s názvy dní a děti přitom ukazovaly na prstech ruky pořadí dnů). Dokud jsou všichni pohromadě, paní učitelka vysvětlí, jaké ateliéry děti ten den čekají. 
		

		\subsection{Ateliéry}
		\label{ateliery}
			Během ateliéru se “nehraje“, ale pracuje. Dětem je to stále připomínáno. Při ateliérech děti sedí u stolů (Obr.~\ref{Obr13},~\ref{Obr14}) a každé dítě pracuje individuálně, nepomáhají si. Vzhledem k vysokému počtu dětí ve třídě jsou ateliéry 4 a děti jsou rozděleny do skupin po 6-7. Každá skupina má na práci jinou činnost. Pro příklad, jedna skupina má grafomotorické listy, jiná skupina stříhá, sestavuje a lepí, třetí staví ze stavebnic, poslední má základy matematiky apod. 4 ateliéry jsou z toho důvodu, že skupiny se každý den vymění a na konci týdne tedy každé dítě projde všemi 4 aktivitami. Tyto ateliéry vyžadují vysokou pozornost a spolupráci paní učitelky. Ta obchází stolky a pomáhá dětem, které to potřebují. 
			Finální práce si děti musí samy podepsat. V různých kelímcích mají i názvy dnů a měsíců, aby mohly napsat i správně datum. Poté si své práce samy lepí do svých sešitů. Tyto sešity fungují jako ukázka toho, co děti dělaly a jak se zdokonalují. Jednou až dvakrát za půl roku jsou poskytnuty rodičů domů k prohlédnutí (Obr.~\ref{Obr15}). 
			Odpolední ateliéry jsou již jednoduššího rázu, více odpočinkové. V této třídě byly 4 počítače s předmatematickými hrami, které byly u dětí velmi oblíbené. Dále byly v nabídce stavebnice Lego, tematická výtvarná činnost či stříhání a opětovné skládání částí lidského těla. Při výtvarné aktivitě byl dětem představen vzor obrázku, podle kterého měly děti nakreslit svůj obrázek. Paní učitelka děti hodně korigovala, aby se daný výtvor vzoru podobal.

		\subsection{Společný kruh}
			V průběhu dne se konají seskupení dětí okolo tabule. Tento čas je zaměřen na básničky, říkanky, matematické hry, zpěv, hlasová cvičení, poslech a na učení se nových písmen nebo číslic. O čem se budou ten daný den bavit, záleží na dni předešlém. Opakuje se básnička či písnička, přidává se nová sloka, učí se nová číslice či písmeno nebo se prohlíží a čte nějaká kniha. Když chce dítě něco říci, musí dodržovat pravidla třídy, v tomto případě se tedy hlásí a musí počkat, až ho paní učitelka vyvolá, teprve poté může mluvit, stejně jako ve škole. Smí mluvit pouze jedno dítě, musí mluvit nahlas a ostatní děti ho nesmí vyrušovat. A tak se stávalo, že se některé děti jen hlásily, protože chtěly být vyvolané, ale žádnou odpověď nevěděly. Hlášení se muselo striktně dodržovat, na druhou stranu, když se četla či prohlížela nová kniha, nechala paní učitelka děti mluvit více spontánně, aby se každý mohl vyjádřit.

		\subsection{Přestávka a pobyt venku}
		\label{prestavka}
			Děti mají během vyučování dvě třicetiminutové přestávky, které tráví na dvoře, jednu během dopoledne a jednu odpoledne. Ven chodí všechny děti ve stejný čas, tedy 6 tříd a nad nimi mají dozor minimálně dvě učitelky. Každá učitelka má dozor dvakrát do týdne. Přestávka může být lehce prodloužena při pěkném počasí a zkrácena při velmi špatném počasí. Dvůr má na jedné straně přístřešek, což je vlastně střecha nad příjezdem do školy, kde se děti mohou při špatném počasí schovat. Když neprší moc, jsou venku po celou dobu přestávky. Děti mají k dispozici tříkolky a odstrkovadla. Na těch může jezdit jen ta třída, jejíž učitelka má zrovna službu na dozor. Dále mají k dispozici míče. Děti se převážně honily, povídaly si ve dvojicích až trojicích, některé děti jen postávaly. V této škole jsem byla na podzim, bylo spadané listí, některé děti si hráli s listím, jiné dostaly koště a pomáhaly listí shrabat. Několik dětí postávalo pod přístřeškem a čekalo, až přestávka skončí, protože jim byla zima z důvodu nedostatečného oblečení a nechtěly si kvůli tomu hrát. Přestože je na dvoře k dispozici prolézačka, děti na ní nesměly kvůli špatnému počasí (Obr.~\ref{Obr16},~\ref{Obr17}). 
			Konec pobytu na dvoře se oznamoval zazvoněním na zvoneček. Děti se pak řadily ke dveřím své třídy, kde si je vyzvedla paní učitelka. 

		\subsection{Strava a pitný režim}
			Součástí objektu je i prostorná jídelna společná pro všechny třídy. Jídelna nemá vlastní kuchyň, jídlo se nechává dovážet z městské kuchyně a zde se jen ohřívá. Stravovat se v jídelně není povinnost. Rodiče si též své dítě mohou vzít na oběd domů a vrátit ho do školky až ke konci přestávky na oběd, která je hodinu a půl. To se ale stávalo jen výjimečně.
			Oběd je jediná strava během dne. Svačina se v této škole nepodává vůbec a o přestávce je zakázáno jíst. Nejdříve padlo rozhodnutí, že kvůli možným potravinovým alergiím, budou svačinu dětem rodiče dávat s sebou, ale děti si prý záviděly a tak vydalo vedení školy definitivní zákaz svačin s tím, že děti do oběda vydrží. Toto se bude asi lišit škola od školy.
			Děti mohou pít během celého dne, u umyvadla mají připravené kelímky a kdykoliv dítě požádá, může si samo natočit vodu z vodovodu a napít se. Učitelka občas děti upozorní, že se mohou napít, ale není zde kladen větší důraz na dodržování pitného režimu.

		\subsection{Odpočívání/spaní}
		\label{spani}
			Po obědě děti chodí opět na dvůr, kde se o ně tentokrát starají dva animátoři, většinou studenti volnočasových aktivit či budoucí učitelé sportu.
			Menší děti chodí po obědě spát. Na spaní jsou vyhrazeny dvě speciální místnosti, kde jsou dětem na zem rozložena lehátka s přikrývkou. Místnost byla menší a tak byla lehátka rozložena těsně vedle sebe. Okno je během odpočinku zatemněno a světla zhasnuta (Obr.~\ref{Obr18}).
			Každé dítě má svůj koš, kam si dává boty a oblečení. Děti spaly jen ve spodním prádle. Při převlékání si děti sedaly na zem na chodbě. 

		\subsection{Tělocvična}
			Tato školka má velmi prostornou tělocvičnu rozdělenou na dvě poloviny, v jedné polovině se cvičilo a v druhé bylo uskladněno náčiní. Nabídka náčiní byla pestrá a bohatá. Tělovýchovná chvilka je podle rozvrhu zařazena do programu každý den na 30 minut. Tato chvilka je však zařazena hned po ateliérech a tak se pravidelně stávalo, že se kvůli prodloužení aktivity děti do tělocvičny vůbec nedostaly a navštívily ji v průměru jednou až dvakrát za týden. 

		\subsection{Hygienické zázemí}
		\label{zachody}
			Pro všechny třídy na patře se nachází jedna místnost se záchody, mušlemi a kruhovou fontánou, která slouží jako umyvadlo. Toalety jsou odděleny přepážkou (Obr.~\ref{Obr19}). Když děti potřebují, dovolí se paní učitelky a na záchod chodí samy. U menších dětí je doprovází asistent, je-li ve třídě. 
		

		\subsection{Odchod dětí z mateřské školy}
			Děti si rodiče vyzvedávají u dveří třídy a paní učitelka osobně volá dítě, které k nim patří. Bez vědomí paní učitelky nesmí žádné dítě odejít. U dveří ze školy ven stojí pan ředitel a všechny rodiče zdraví a drží dozor nad odchody. 

	\section{Režim dne v české mateřské škole}
		Aby byly zachovány podobné podmínky, vybrala jsem si pro srovnání státní mateřskou školu v Praze spolupracující s pedagogickou fakultou Univerzity Karlovy, ve které jsem též byla na průběžné dvoutýdenní praxi. Tato mateřská školka je otevřena od 7 do 17 hodin v pracovní dny od pondělí do pátku.

		\subsection{Zázemí mateřské školy}

			Tato mateřská škola stojí v samostatné větší dvoupatrové budově s rozlehlou zahradou, která má dvě části, jednu s dřevěnou věží, s betonovými cestičkami pro jízdu na koloběžkách a odstrkovadlech a s menší okrasnou zahrádkou s jezírkem. Na druhé části zahrady je pískoviště, skluzavka a houpačky. Součástí budovy je i vlastní kuchyň a tělocvična, která se používá i jako divadelní sál. Nedílnou složkou je i Speciální pedagogické centrum, se kterým tato škola blízce spolupracuje. Mateřskou školu v současné době navštěvuje 114 dětí, které jsou rozděleny do 4 tříd a jedné speciální třídy pro děti se speciálními vzdělávacími potřebami. 

		\subsection{Třída a její vybavení}
			Třída, ve které jsem praxi absolvovala, měla dvě větší místnosti nebo části. V jedné části je stůl pro učitele, stolky s židlemi k jídlu a další 4 stolky pro práci dětí. Podél zdí jsou poličky s nabídkou stolních a logických her, je zde i malý koutek za záclonou, když děti potřebují chvilku o samotě. Nachází se zde i malý koutek s dětskou kuchyňkou a stolkem. Přechod do druhé místnosti je ohraničen lavičkou, na podlaze je po celé délce položen koberec. Jsou zde police a krabice na panenky, látky, stavebnice, kout s velkou dřevěnou stavebnicí, molitanové kvádry, žebřiny, klavír a další. Na konci místnosti jsou dva kumbály na pomůcky a za zástěnou jsou schované matrace na spaní a přikrývky. Před vstupem do třídy je šatna, kde má každé dítě svoji skříňku na náhradní oblečení, oblečení na ven a přezůvky, či boty na zahradu. Do třídy se jde přes místnost s umyvadly, kde má každé dítě svůj ručník. V koutě se dokonce nachází zmenšenina truhlářského ponku. Dále tu jsou samozřejmě i toalety, na jednu třídu jsou 4 dětské toalety oddělené přepážkami. 

		\subsection{Počet dětí a pedagogické zastoupení}
			V každé třídě je 25-26 dětí, do kterých jsou integrované děti se specifickými potřebami. Na každou třídu jsou dva učitelé a jeden asistent. V této třídě byl jeden pan učitel a dvě učitelky, které si dělily jeden úvazek. Dále je v každé třídě jeden asistent. Na dopoledne je vždy přítomen jeden učitel a asistent, druhý učitel přichází o něco později a zůstává na odpolední program. Třída je heterogenní, tzn., že ve třídě jsou děti ve věku od 3 do 6 let. 

		\subsection{Pravidla chování}
			Pravidla v této třídě nejsou nikde vyvěšena. Jsou zde chápána jako opatření, která napomáhají organizaci. Důležitým pravidlem je zazvonění na zvonek, při kterém se musí všichni ztišit a dávat pozor, co se bude dít. Ostatní pravidla jsou spíše připomínána ve chvíli, kdy nastane nějaký konflikt a vždy je dětem jasně vysvětleno, proč by se zrovna taková pravidla měla dodržovat. Dbá se i na to, aby určitá pravidla byla připomenuta dětmi samotnými. Tudíž ne z pozice učitel-dítě, ale z pozice rovný s rovným.
			Hodně se zde pracuje se vzájemnou důvěrou, nechají děti dělat různé stavby při volné hře i mimo místa tomu určena, ovšem za určitých podmínek a vzájemně si důvěřují, že obě strany dohodu dodrží. Občas dostanou starší děti za úkol dohlédnout na ty mladší. Toto předávání funkcí a důležitosti na starší děti bylo pozitivně  a zodpovědně přijímáno.

		\subsection{Průběh dne a jeho specifika}

			Od 7 do 8 a od 16 do 17 hodin jsou děti sdruženy jen do jedné třídy kvůli malému počtu dětí. Od 8 do 16 jsou ve svých kmenových třídách.
			Časový harmonogram je spíše orientační, aby byl dětem zachován řád a posloupnost činností. 

% TODO: do tabulky
7:00 – 8:45		Příchod dětí
8:00 – 9:15		Příchod do tříd a volná hra
9:15 – 10:30		Kontaktní kruh
			Svačina
			Hlavní společná činnost
10:30/11:00 – 12:00	Pobyt venku
12:15			Oběd
			Odpočívání/spaní
14:30			Svačina
15:00			Odpolední program

		\subsection{Příchod dětí do třídy}
			Jak jsem již zmínila dříve, děti, které navštěvují mateřskou školu již od 7 hodin, se sdružují vždy v jedné třídě, od 8 hodin se pak přemísťují do jejich kmenové třídy. Při vstupu do třídy se děti přezouvají a převlékají do oblečení určeného do školky, které se může ušpinit. Rodiče doprovodí dítě až do třídy, kde se všichni pozdraví s učiteli. Rodiče poté odcházejí. Učitel se snaží s dítětem promluvit, zeptat se ho, jak se má, apod.  Dítě má dále prostor na volnou hru.

		\subsection{Volná hra}
			Volnou hru nebo také volné činnosti si dítě může volit samo. Je na dítěti samotném, čím se zaměstná, jestli si bude hrát samo nebo někoho přizve, či se k někomu připojí. Projevuje svou vlastní aktivitu a učitel zde hraje roli podpůrnou a motivační, ale neurčuje, co má dítě dělat. V této školce se snaží o posilování sociálních vztahů mezi dětmi, učení vzájemné spolupráce mezi dětmi při hře i při řešení konfliktů na principu respektování druhého. Pokud to není nezbytně nutné, nechávají se dětem rozestavěné dětské stavby, aby se k nim mohly vrátit později. Důraz na volnou hru se projevuje v čase, který je pro volnou hru vyhrazen. Ráno je dán dětem velký prostor, okolo hodiny a čtvrt, podle příchodu do mateřské školy, volná hra převládá i při pobytu venku a další prostor je jí věnován při odpoledním programu. 

		\subsection{Kontaktní kruh}
			Kontaktní kruh trvá cca 15 minut a je důležitou každodenní společnou činností s jasně danou strukturou. Na začátku kruhu učitel zapaluje svíčku, poté se zeptá: „Kdo má rybu?“ To je otázka na látkovou hračku, kterou každé ráno potají dostane jedno dítě, a ostatní děti hádají, kdo ji má právě dnes. Poté se ryba posílá po kruhu a každý, kdo má rybu v ruce hovoří, ostatní naslouchají mluvícímu. Učitel určuje téma. Mluví se o tématech, která se váží k plánovaným aktivitám nebo aktivitám z minulého dne. Dává se prostor ale i těm nejmenším, ptá se tedy i na otázky, co dělaly děti o víkendu, aby pověděly ostatním o své oblíbené hračce, nebo všichni mají možnost říci, co vědí o jednom vybraném kamarádovi. Dále se pravidelně hraje na kytaru a zpívá písnička a nakonec se rozhoduje, kdo sfoukne svíčku. Na tom se děti domlouvají společně a vybrané dítě si k sobě může nebo nemusí někoho přibrat. 

		\subsection{Hlavní činnost}
			Hlavní činnost nebo také soustředění, práce, zaměstnání závisí na daném tématu. V této třídě se pracuje s tematickými celky a projektovým učením. Tato třída je ještě specifická svým dramatickým zaměřením. Je zde kladen větší důraz na prožívání aktivit a účastnění se jich. Součástí hlavní činnosti je i stavění dekorací, které podtrhují celkové téma, které děti doprovází třeba celý měsíc. Tyto dekorace jsou pak využívány k dalším činnostem, vše se prolíná se vším. Z organizačních důvodů jsou některé společné činnosti rozděleny: 1. skupina pracuje, 2. si hraje a pak se vystřídají nebo 1. skupina pracuje a 2. jde ven a pak se vystřídají, anebo jdou všichni nejdříve ven a poté pracují všichni společně. Učitelé si hodně přizpůsobují formu dané aktivitě. Během týdne se v aktivitách vystřídají všechny výchovy, od hudební, dramatické, výtvarné po tělesnou. Před každou delší činností však učitelé vždy nechávají děti „vyřádit“, aby se mohli lépe soustředit. Děti běhají cca 5 minut po třídě do rytmu bubnů či klavíru a na povely učitelů. 

		\subsection{Pobyt venku}
			Ven se chodí dle počasí téměř každý den a pobytu venku je vyhrazen čas 1 až 1 a půl hodiny. Při pobytu venku opět převládají volné činnosti dětí. Je jim dán velký prostor k seberealizaci. Třídy se střídají na dvou částech zahrady. Na jedné je k dispozici skluzavka, pískoviště a houpačky. Na druhé části zahrady je dřevěná věž a cesty pro jízdu na koloběžkách a odstrkovadlech. Mateřská škola má příjemné umístění nedaleko zeleně, a tak se daný čas využívá i k výletům do nedalekého lesa a hřiště v lese.

		\subsection{Strava a pitný režim}
			Jak jsem již uvedla dříve, mateřská škola má vlastní kuchyň a vaří vlastní stravu. Dětem je dovážen oběd až do tříd, kde všichni společně jedí. 
			Svačina se podává dvakrát denně. Na přípravě svačiny se děti sami podílejí, připravují talíře a skleničky pro ostatní a pomazánku na chleba si děti mažou sami nebo s dopomocí. 
			Na pitný režim je dáván velký důraz, po každém proběhnutí dětí podává asistent dětem pití, pije se i při obědě a u každé svačiny. 

		\subsection{Odpočívání/spaní}
			Děti mohou odejít domů již po obědě dle potřeb rodičů. Ty, co zůstávají, dodržují odpolední klid. Mladší děti mají 	od paní hospodářky připraveny matrace s pokrývkou a chodí spát. Před spaním se jim většinou čte kniha. Respektují zde přání dítěte, pro odpočinek dětí se snaží vytvářet příjemnou atmosféru. Pokud děti potřebují více soukromí, snaží se jim vytvořit „domeček“. Děti si také mohou půjčit ke spaní plyšovou či jinou hračku, která nedělá hluk. Starší děti dodržují klidový režim, mohou ležet a číst si nebo ležet nemusejí a věnují se jiným klidovým aktivitám, kreslení, puzzle, pexeso a další.  Předškoláci mají v tuto dobu předškolní přípravu. K odpolední svačině se vstává kolem 14h30. Děti, které spí, se nechávají spát a budí se podle dohody s rodiči.

		\subsection{Tělocvična}
			Tělocvična se nachází v přízemí budovy. Je to místnost s kobercem, vybavená klavírem a tělocvičným nářadím od švédské bedny, kladiny, po gymbally, míče, švihadla a další. Každá třída má v týdnu vyhrazenou celou jednu hodinu na pobyt v tělocvičně. Tato místnost je využívána pro další aktivity mateřské školy, jako jsou například divadelní představení. 

		\subsection{Odchod dětí z mateřské školy}
			Rodiče si mohou děti vyzvedávat po obědě od 12h30 do 13h15 a odpoledne od 14h30 kdykoliv do zavírací doby. Rodičům je dán velký prostor, kdy si mohou pro svoje děti přijít, podle potřeb a možností jejich pracovní doby. Rodiče si děti vyzvedávají ve třídě nebo na zahradě. Učitelé mají s rodiči dobré vztahy a všichni se znají, a tak učitelé vědí, který rodič patří ke kterému dítěti.

	\section{Srovnání režimu mateřské školy ve Francii a České republice}
\label{srovnani}
%TODO: tabulka
Srovnávací tabulka režimu dne

Francie
Česká republika
příhod dětí
8h50 - 9h10
7h - 9h15
čas na volnou hru
20min
1h15min
aktivity při volné hře
puzzle, malování, čtení
stavebnice, hračky, látky, molitanové kosty, malování, čtení
čas na řízenou aktivitu
5h30min
2-3h
řízené aktivity
"práce"-grafomotorika, stříhání, lepení, skládání, pregramatické činnosti
"hra" - výtvarné, hudební, dramatické, námětové hry, hry s pravidly
čas na rituály
10min
10-15min
rituály
docházka, datum, přivítací říkanka
ranní povídání/vyprávění
přestávka
pobyt venku
svačina
čas přestávky 
2x30 min
2x 15-20min
pobyt venku
2h
1h-3h
místo pobytu venku
dvůr
zahrada
strava
jen oběd
2x svačina, oběd
odpolední spaní
1h
1h30min
odchod dětí
16h30min
12h30 - 13h15, 14h30-17h

		Ze srovnávací tabulky jsou na první pohled zřetelné rozdíly v harmonogramu a hodinových dotací jednotlivých činností.

 		Z příchodu a odchodů dětí z mateřské školy a vlastní praxe vyplývá, že v České republice jsou otevírací časy mateřské školy přizpůsobené potřebám rodičů. Ráno mohou dát děti do školky již od brzkých hodin, aby mohli být včas práci a o děti bylo vhodně postaráno. I odpolední vyzvedávání je jim přizpůsobeno, mohou si své děti vyzvedávat již po obědě nebo po odpoledním spaní, na druhou stranu, je-li potřeba, děti mohou ve školce zůstat až do večerních hodin. Oproti tomu ve Francii příchod a odchod dětí kopíruje docházku na základní školu. Příchod do práce je ve Francii též pozdější, než je obvyklé v České republice. Ve Francii však zůstávají v práci do pozdějších večerních hodin, proto je velmi časté, že děti vyzvedávají chůvy a zůstávají s nimi do příchodu rodičů. Dalším markantním rozdílem je čas na volnou hru a řízenou aktivitu. Podle vzdělávacího plánu Francie je patrné, že děti jsou připravovány na školu. Na volnou hru je prostor jen při příchodu dětí do mateřské školy a při pobytu venku. Avšak pobyt venku je pro volnou hru omezující, vzhledem k nedostatku hraček, materiálu a prostoru. Všechny děti z mateřské školy jsou venku společně, dvůr je tedy relativně zaplněn. Nedostatek času pro volnou hru vyplývá z hodinové dotace, která je dána školským zákonem. Bulletin officiel (B.O.) uvádí 24 hodinový týden výuky. Děti tráví v mateřské škole během 4 dnů v týdnu celkově 30 hodin, z toho 24 hodin je věnováno výuce a přípravě na školu. Zbylý čas vychází na pobyt venku a polední pauzu na oběd a odpočinek. Oproti tomu je v České republice kladen velký důraz na dětské prožívání, vlastní kreativitu, socializaci a volné hře je ponechán daleko větší prostor. Dopoledne je jí věnována více jak hodina, záleží na času příchodu dítěte, volnost mají děti i při pobytu venku, kde mají větší prostor k pohybu i větší výběr materiálu a pomůcek ke hře. I v odpoledních hodinách je volné hře věnován dostatek prostoru. Dbá se i na přirozené a pravidelné střídání volných a řízených aktivit.

		Tato bakalářská práce se nevěnuje práci s dětmi a přístupu k nim, ale pro lepší obrázek o tom, jak to chodí za našimi hranicemi, je podle mého názoru velmi důležité též zmínit rozdíl v řízených aktivitách. Při francouzských ateliérech se „pracuje“, dětem je neustále připomínáno, že si nehrají, že mají správně sedět a soustředit se na svou „práci“. Činnost musí vždy dokončit, podepsat si ji, napsat datum (okopírovat datum z předlohy) a vlepit do svého sešitu, kam si vkládají všechny vlastní práce. V České republice mají řízené aktivity spíše formu „hry“. Děti si hrají, malují, tvoří, zkouší, apod. 
		Obsahová stránka řízených aktivit se zdá být v některých věcech podobná. Grafomotorické listy mají stejnou podobu, stavebnice a puzzle se najdou v obou zemích, jen ve Francii patří mezi řízené aktivity, v České republice si je děti vybírají během volné hry. Výtvarné techniky jsou stejné, tělesná výchova je též podobná, s větším důrazem na ofenzivní hru ve Francii, kdy 2 děti stojí proti sobě a snaží se druhého vytlačit z vymezeného prostoru, či ho převrátit z břicha na záda. To jsem v České republice nezažila. Dramatické výchově se ve Francii přikládá minimální význam. 

		Ke srovnání režimu dne byly použity dvě rozdílné třídy, jedna homogenní (F) a jedna heterogenní (ČR), je tedy těžké porovnávat přípravu předškoláků. Ale z toho, co jsem viděla v české mateřské škole, kde bylo 6 předškoláků a francouzské mateřské třídě předškoláků „Grande section“, je ve Francii dáván daleko větší důraz na rozvoj jazykových schopností, který s největší pravděpodobností vyplývá z nestejnosti psaného a mluveného jazyka, a dále na rozvoj psaní. Děti by při přechodu na základní školu již měly umět psát tiskacími písmeny a přečíst své jméno psacím písmem. 

		Stejným či společným aspektem je odpolední odpočinek. Jak v České republice tak ve Francii nejmenší děti mají vyhrazený čas okolo jedné až jedné a půl hodiny na spánek. Rozdíl je u starších dětí, u nás mají klidové činnosti, jako je čtení, malování, odpočinek, ve Francii starší děti tráví tento čas venku na dvoře za dohledu animátorů, kteří jím nabízejí aktivity sportovnějšího zaměření. Děti ovšem nejsou povinny se jich zúčastnit.

		Dalším srovnávaným faktorem je strava. Nerada bych zde dělala závěry, jsem přesvědčena, že zákaz svačin v mateřské škole ve Francii byl čistě individuální a zjednodušoval práci jak vedení, tak učitelům, aby nevznikaly zbytečné komplikace. Na dětech bylo zřetelné, že strava byla nedostatečná, v odpoledních hodinách byla vidět na dětech patrná únava, podrážděnost a apatie. To vše bylo podpořeno nedostatečným důrazem na pitný režim.

		Též hygienické zázemí je nesrovnatelné. Ve Francii je jedna hygienická místnost pro všechny třídy, otevřená, bez známky intimity, dost často tato místnost zaváněla močí. V České republice má každá třída své hygienické zázemí, jedná se o menší a intimnější prostor než ve Francii.
		Nakonec bych se chtěla zmínit o velmi důležitém aspektu rozdílnosti ve výuce. Tím je počet učitelů na třídu. Ve Francii je na jednu třídu o 30 dětech jedna učitelka na celý den. S nikým se nestřídá ani v průběhu dne ani v průběhu týdne. V České republice jsou standardem 2 učitelky na třídu, které se střídají, a část výuky se jim překrývá. Ranní a odpolední služby se jim pravidelně střídají. Co se týče prostorů, je poměr prostoru na dítě velmi rozdílný. Ve Francii je větší počet dětí „vtěsnán“ do jedné místnosti, kde tráví celý den. V České republice má většinou jedna třída dvě místnosti, děti mají tedy téměř dvojnásobek prostoru na o něco menší počet dětí.

		
		Časová dotace výuky, režim dne, přístup k dětem a metody výuky jsou po celé Francii stejné, ale strava a prostory se budou lišit podle možností a financí jednotlivých městských částí. Nelze tedy ze všech faktorů, které tu uvádím, vyvozovat jasné závěry.
