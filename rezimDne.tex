\chapter{REŽIM DNE}
\label{rezim}

	Druhý srovnávaný aspekt této práce je režim dne. Tato část práce vychází z vlastních osobních zkušeností autorky, která se v rámci studií účastnila povinných praxí jak v české, tak ve francouzksé mateřské škole. Byla použita metoda nestrukturovaného pozorování, která je vysvětlena v kapitole \ref{metody}. Níže je prezentována vždy jen jedna třída jedné mateřská školy z dané země. Nejde o bohatý výzkumný vzorek, z kterého by se daly vyvozovat obecné závěry. Cílem je přiblížit českému čtenáři organizační strukturu jednoho dne v předškolním zařízení obou zemí. Kraus stručně definuje režim jako\textit{\uv{přesně určený rozvrh života, práce, činnosti}}. \citep[s.~700]{Kraus}
	Pro tuto práci z toho vyplývá, že režimem dne je míněn přesný časový rozvrh běžného dne, který se pravidelně opakuje.
	Součástí této kapitoly je i materiální zázemí obou srovnávaných škol doplněné fotodokumentací, které autorka považuje za důležité zmínit z důvodu rozdílnosti podmínek, v kterých se předškolní vzdělávání odehrává.

	\section{Režim dne ve francouzské mateřské škole}

		Praxe ve Francii probíhala od 8. 11. 2010 do 19. 11. 2010. v mateřské škole Maintenon na adrese 3 rue des Glycines, 927 00, Colombes na předměstí Paříže. Tato mateřská škola spolupracovala s Academie Versaille pod Université de Cergy-Pontoise, která byla partnerskou univerzitou Univerzity Karlovy v rámci studentského projektu Erasmus. 


		\subsection{Průběh dne a jeho specifika}

			Mateřská škola byla otevřena od 8:50 do 16:45. Zajímavostí francouzských mateřských škol je 4 denní týden. Děti navštěvují mateřskou školu jen v pondělí, úterý, čtvrtek a pátek. Ve středu děti zůstavají doma nebo mají volnočasové aktivity a sporty. Některé školy tyto aktivity nabízejí, jiné ne. Tato mateřská škola patřila mezi ty, které aktivity nenabízejí. 
			
			Časový harmonogram je závazný. Zde je uváděn harmonogram třídy \uv{Grande section}, více ke třídě v kapitole \ref{tridaVybaveni} a \ref{trida}. Obdobný rozvrh byl vyvěšen v tištěné podobě na dveřích každé třídy. 

\begin{spacing}{1.0}
	\begin{table}[h!]
		\center
		\begin{tabular}{|l l|}
			\hline
			\rowcolor{grey!0}
			8:50 – 9:10 		& Příchod dětí a jejich uvítání 						\\
								& (volná hra, dokončování prací z minulého dne, úklid) 	\\
			9:10 – 9:20			& Rituály 												\\
								& (pozdravení se, datum, počasí, představení ateliérů) 	\\
			9:20 – 10:05		& Dílny 												\\
								& (grafomotorika/psaní, matematika, čtení) 				\\
			10:05 – 10:15		& Společný kruh (úklid, básničky/říkanky) 				\\
			10:15 – 10:45		& Přestávka 											\\
			10:45 – 11:30		& Lingvistické aktivity, společné čtení 				\\
			11:30 – 11:50		& Společný kruh (říkanky, matematické hry) 				\\
			11:50 – 13:30		& Oběd 													\\
			13:30 – 13:55		& Společný kruh (zpěv, hlasová cvičení, poslech) 		\\
			13:55 – 14:30		& Ateliéry 												\\
								&(motorika, výtvarná výchova, objevování světa) 		\\
			14:30 – 15:00		& Tělocvična 											\\
			15:00 – 15:30		& Přestávka 											\\
			15:30 – 16:10		& Video nebo promítání diapozitivů 						\\
			16:10 – 16:20		& Úklid třídy, zhodnocení dne 							\\
			16:20 – 16:30		& Odchod dětí 											\\
			\hline
		\end{tabular}
		\caption{ \textbf{Časový harmonogram ve francouzské MŠ}	
		}
		\label{tab:rezimDneFR}
	\end{table}
	\end{spacing}

		\subsection{Příchod dětí do mateřské školy}
		\label{prichod}
			Mateřská škola byla ráno otevřena od 8h50 do 9h10. Na chodbě před třídou mělo každé dítě svůj háček na pověšení oblečení a malou přihrádku na menší věci (Obr.~\ref{Obr9}). Děti se nepřezouvaly, zůstávaly celý den ve stejné obuvi, ve které přišly. Při vstupu do třídy vítala paní učitelka děti i jejich rodiče. S každým dítětem se poté snažila navázat kontakt. Kladla dětem otázky, jak se mají, co dělaly o víkendu apod. Každé dítě si poté našlo na stole cedulku se svým jménem a přileplo ji na menší tabuli se suchým zipem vedle velké tabule (Obr.~\ref{Obr10}). Tímto byla zjištěna docházka dětí, která je poté součástí ranního rituálu. Dále měli děti čas na volnou hru, malování, prohlížení knížek, skládání puzzle, dokončování výtvarných prací z minulého dne.
		
		\subsection{Rituály}
		\label{ritualy}
			K rannímu rituálu se děti usazovaly na lavičky před tabulí. Některé děti si z nedostatku míst sedaly na koberec. Jedno vybrané dítě mělo za úkol spočítat kartičky se jmény děvčat a chlapců a kolik dětí je přítomno celkově.Tato čísla zapsalo na tabuli na předepsané místo. Další dítě mělo na starost datum, nejdříve změnilo číslici dne a napsal na tabuli novou. Je.li potřeba, mění toto dítě i kartičku s názvem měsíce. Celá třída poté společně přečetla celé datum. (Obr.~\ref{Obr11},~\ref{Obr12}). Dále všichni společně zarecitovaly uvítací říkanku (v této třídě se jednalo o básničku s názvy dní a děti přitom ukazovaly na prstech ruky jejich pořadí). Jako poslední bod rituálu vyučující dětem vysvětlila, jaké aktivity je ten den čekají a podrobně je popsala. 

		\subsection{Pravidla chování}
		\label{pravidlaChovani}
			Zajímavostí této třídy byly obrázky s pravidly, které se měly ve třídě dodržovat. Byly zobrazeny na červeném a zeleném papíře velikosti A3 viz. Obr.~\ref{Obr6}) a byly viditelné již ode dveří třídy. Na pravidla se vyučující odkazovala téměř pokaždé, když byla některá z nich porušena. Dítě, které nějakým způsobem pravidla nedodrželo, bylo vyzváno, aby ukázalo, o které pravidlo se jedná a povědělo všem, jak by se mělo chovat. 

			\begin{spacing}{1.0}
			\begin{table}[h!]
				\center
				\begin{tabular}{|ll|ll|}
					\hline
					\rowcolor{grey!0}
				+	& Papír se vyhazuje do koše						& -	& Neprat se 			\\
					& Hlásit se 									&  	& Neběhat po třídě		\\
					& Uklízet po sobě materiál 						&	& Nestrkat se 			\\
					& Řadit se do řady 								&	& Nekřičet 				\\
					& Být potichu 									& 	& Neničit materiál 		\\
					& Udržovat stoly čisté 							& 	& Neschovávat věci 		\\
					& Říkat „Dobrý den“,							&	& Neříkat sprostá slova \\
					&  „Na shledanou“, „Děkuji“						&	& Nekrást				\\
					&												&	& Neobtěžovat kamarády 	\\
					\hline
				\end{tabular}
				\caption{ \textbf{Srnutí pravidel chovaní ve francouzské školce.}}
			\label{tab:pravidlaFR}
			\end{table}
			\end{spacing}

			\subsection{Společný kruh}
			V průběhu dne se konaly dvě seskupení u tabule. Tento čas byl zaměřen na básničky, říkanky, matematické hry, zpěv, hlasová cvičení, poslech a na učení se nových písmen nebo číslic. Co bude tématem daného dne se odvíjelo ode dne předešlého. Například se opakovala básnička či písnička nebo se přidávala nová sloka, učila se nová číslice či písmena, anebo se prohlížela a četla nějaká kniha. Pokud chtělo dítě něco říci, muselo se podle pravidel třídy přihlásit a počkat, až bude vyvoláno, podobně jako tomu je ve škole. Smí mluvit pouze jedno dítě, musí mluvit nahlas a ostatní děti ho nesmí vyrušovat. Některé děti se jen hlásily, protože chtěly být vyvolany, ale žádnou odpověď nevěděly. Hlášení se muselo striktně dodržovat. Oproti tomu, když se četla či prohlížela nová kniha, nechala paní učitelka děti mluvit více spontánně, aby se všechny mohly dostatečně vyjádřit.

		\subsection{Dílny}
			Z francouzského originálu \uv{ateliers}, v české terminologie se dá chápat jako specializovaný koutek, dílničky nebo dílny, dále tedy jen dílny. 
			Během dílen se sedí u stolů a každé dítě pracuje individuálně, nesmějí si pomáhat. Dětem bylo stále připomínaná, že si \uv{nehrají}, ale \uv{pracují}. Vzhledem k vysokému počtu dětí, byly rozděleny do 4 skupinek po 6-7. Každá skupina pracovala na jiném úkolu. Například jedna skupina vyplňovala grafomotorické listy, druhá skupina stříhala, sestavovala a lepila, třetí stavěla ze stavebnic a poslední měla prematematické činnosti. Každý den se témata dílen předala další skupině, takže na konci týdne všechny děti pracovaly na všech aktivitách. Tento způsob práce vyžaduje od vyučující nasazení, spolupráci a pozornost. Ta postupně obchází všechny stolky a pomáhá těm dětem, které to potřebují. Finální práce si děti sami podepisovaly. V připravených kelímcích byly kartičky s názvy dnů a měsíců, podle kterých děti opisovaly nebo kopírovaly datum, které ná práci muselo též figurovat. Hotové práce si poté děti lepily do sešitů, který tak byl základem jejich portfolia. Jednou až dvakrát za půl roku byly poskytnuty rodičům, aby se mohli podívat na čem děti pracují a jaké dělají pokroky \ref{Obr15}. Odpolední dílny měly již odpočinkovější nádech. K dispozici byly 4 počítače s prematematickými hrami, které byly u dětí velmi oblíbené. Dále byly v nabídce stavebnice Lego, tématická výtvarná činnost či stříhání a opětovné skládání částí lidského těla. Při výtvarné aktivitě byl dětem ukázán vzor, podle kterého měly malovat. Vyučující děti hodně korigovala, aby byl výtvor vzoru co nejpodobnější.


		\subsection{Pobyt venku}
		\label{prestavka}
			Děti měly během vyučování vyhrazeny dva třicetiminutové bloky na pobyt venku. Ven se chodilo na dvůr, kde se sešly všechny třídy najednou, nad kterými měly dozor vždy minimálně dvě učitelky. Každá učitelka měla dozor dvakrát do týdne. Čas pobytu venku byl flexibilní a přizpůsoboval se aktuálnímu počasí. Ven se však chodilo i za mírného deště. Dvůr školy má na jedné straně přístřešek, kde se děti mohly při špatném počasí schovat. Děti měli k dispozici tříkolky a odstrkovadla. Ta mohla však používat pouze třída, jejíž učitelka měla zrovna službu na dozor. K dispozici byly i míče. Děti se převážně honily, povídaly si ve dvojicích až trojicích, některé děti jen postávaly. Praxe se konala během podzimu, na zemi bylo spadané listí a tak si některé děti hraly s listím, jiné dostaly koště a pomáhaly listí shrabat. Několik dětí postávalo pod přístřeškem a čekalo, až přestávka skončí, protože jim byla zima z důvodu nedostatečného oblečení a nechtěly si kvůli tomu hrát. Přestože je na dvoře k dispozici prolézačka, děti na ni kvůli špatnému počasí nesměly (Obr.~\ref{Obr16},~\ref{Obr17}). 
			Konec pobytu na dvoře se oznamoval zazvoněním na zvoneček. Děti se pak řadily ke dveřím své třídy, kde si je vyzvedla jejich vyučující. 

		\subsection{Strava a pitný režim}
			Mateřská škola měla svou vlastní jídelnu. Jídlo se nechávalo dovážet, v jídelně se pouze ohřívalo. Stravování v jídelně nebylo povinné. Tradičně si ve Francii rodiče odvádějí děti na oběd domů a do školky se vracejí je 13:30. V této mateřské škole však většina dětí využívala možnosti poskytované stravy. 
			Další popis je specifikum této jedné mateřské školy. Oběd byl pro děti jediná strava během dne. Svačina se v této škole nepodávala vůbec a při pobytu venku bylo zakázáno cokoliv konzumovat. Dříve si děti nosily svačiny z domova, ale z důvodu údajné závisti dětí byl vedením mateřské školy vydán zákaz jakéhokoliv nošení potravin do školy. 
			Příjem tekutin byl povolen během celého dne. U umyvadla měly děti připravené kelímky a kdykoliv požádaly, mohly si samy natočit vodu z vodovodu a napít se. Učitelka občas děti upozornila, že se mohou napít, ale nebyl zde kladen větší důraz na dodržování pitného režimu.

		\subsection{Odpočívání/spaní}
		\label{spani}
			Po obědě chodily děti opět na dvůr, kde se o ně starali dva vychovatelé, většinou studenti volnočasových aktivit či budoucí učitelé sportu.
			Menší děti chodily po obědě spát. Na spaní byly vyhrazeny dvě speciální místnosti, kde byla dětem na zem rozložena lehátka s přikrývkou. Místnost byla menší a tak byla lehátka rozložena těsně vedle sebe. Okno je během odpočinku zatemněno a světla zhasnuta (Obr.~\ref{Obr18}).
			Každé dítě se muselo převléct na chodbě, často si přitom sedaly na studenou zem a své věci si dávaly do připravených košů. Děti spaly jen ve spodním prádle. 	

		\subsection{Odchod dětí z mateřské školy}
			Děti si rodiče vyzvedávaly u dveří třídy a vyučující osobně volal dítě, které má odcházet. Bez vědomí vyučujícího nesmělo žádné dítě odejít. U dveří ze školy stál pan ředitel, zdravil všechny rodiče a dohlížel na bezpečnost odchodů. 

		\subsection{Popis budovy mateřské školy}

			Mateřská škola byla součástí velké budovy, kde se nacházela i škola základní. Děti měly přístup na betonový dvůr, který měl jen na menších částech speciální mekký povrch. K dispozici zde byly dvě prolézačky a pískoviště. Děti měli možnost na dvoře jezdit odstrkovadlech, tříkolkách a hrát si s míči. 
			Mateřská škola měla 8 tříd, z toho 6 se nacházelo v přízemí budovy a dvě byla v prvním patře, v prostorách 	základní školy. Dále se zde nacházela knihovna, tělocvična, dvě ložnice na spaní, sborovna a jídelna. 

		\subsection{Popis třídy a materiálního vybavení}
		\label{tridaVybaveni}
			Třída, ve které se praxe konala, byla menší místnost s položenou dlažbou. Před tabulí byl menší koberec. Místnost měla dvoje dvěře, jedny vedly na školní chodbu a druhé přímo na dvůr.
			Na jedné straně třídy visela na zdi pověšená tabule, na které byly přilepené různé cedulky s návody, jak se píší číslice 1-10, menší cedulky s číslicemi od 1 do 30, cedulky s názvy dnů v týdnu a měsíců v roce.V rohu tabule visel popis dílen a v dolní části se psal počet dětí a aktuální datum (Obr.~\ref{Obr1}). Před tabulí byly do kruhu postaveny 3 lavice na sezení. Po obou stranách tabule byly umístěny stolky s dětskými počítači (Obr.~\ref{Obr2}). Uprostřed třídy bylo postaveno 5 stolů, každý s 6 židličkami. Na protilehlé straně třídy se nacházel čtenářský koutek s křesílkem, umyvadla, police na výtvarný materiál, bílá tabule, stůl pro vyučujíchiho, dětská kuchyňka s popisky věcí, které patří do kuchyně (Obr.~\ref{Obr3},~\ref{Obr4},~\ref{Obr5}). Na této straně byla na zdi pověšená písmena abecedy velikost A3 (Obr.~\ref{Obr6}). Čtenářský koutek i kuchyňka byly od třídy odděleny různými skříňkami a poličkami, které sloužily k uskladnění didaktických pomůcek a her či sešitů dětí (Obr.~\ref{Obr7}).

			Třída byla materiálně velmi dobře vybavena. Nacházelo se zde hodně didaktických pomůcek. Ve třídě se nacházelo vše potřebné, ovšem pro volnou hru a volný pohyb dětí mnoho místa nezbývalo. K té de dal využít menší prostor s kobercem před tabulí, dětska kuchyňka nebo prostor před vchodem na dvůr a ulička napříc třídou. Celkový prostor třídy se zdál pro 29 dětí velmi malý.

			Děti měly na výběr různé stolní hry, puzzle, knihy. Hraček, tak jak je známe z českých školek, jako jsou panenky, plyšáci, autička apod se zde nacházelo minimum. Pokud si děti donesly nějakou hračku z domova, musely ji na začátku hodiny odložit na poličku a vyzvednout si ji mohly na konci dne. 

		\subsection{Počet dětí a pedagogů ve třídě}
		\label{trida}

			Praxe se konala ve třídě “Grand section“ viz. \ref{tab:rozdeleniTridCR}. Jedná se o třídu homogenní, tzn.třídu, kterou navštěvují děti stejného věku. Ve třídě bylo zapsáno 29 dětí, mateřskou školu jich v průměru navštěvovalo 25. Děti trávily v mateřské škole celý den. Na tuto třídu byla jedna paní učitelka a to od pondělí do pátku po celou otevírací dobu mateřské školy. S nikým se nestřídala. 

		\subsection{Tělocvična}
			Tato mateřská škola měla velmi prostornou tělocvičnu rozdělenou, která byla společná pro školu základní. Část sloužila jakko sklad náčiní a druhé části se konala výuka. Nabídka náčiní byla pestrá a bohatá. Tělovýchovná chvilka byla podle rozvrhu zařazena do programu každý den na 30 minut. Tato chvilka byla však zařazena hned po dílnách a tak se pravidelně stávalo, že se kvůli prodloužení aktivit děti do tělocvičny vůbec nedostaly a navštívily ji v průměru jednou až dvakrát za týden. 

		\subsection{Hygienické zázemí}
		\label{zachody}
			Pro všechny třídy na patře se nachází jedna místnost se záchody, mušlemi a kruhovou fontánou, která slouží jako umyvadlo. Toalety jsou odděleny přepážkou (Obr.~\ref{Obr19}). Když děti potřebují, dovolí se paní učitelky a na záchod chodí samy. U menších dětí je doprovází asistent, je-li ve třídě. 
	

%!!!!!!!!!!!!!!!!!!!!!!!!!!!!!!!!!!!!!!!!!!!!!!!!!!!!!!!!!!!!!!!!!!!!!!!!!!!!!!!!!!!!!!!!!!!!!!!!!!!!!!!


\section{Režim dne v české mateřské škole}

		Praxe v České republice se konala ve fakultní mateřské škole Sluníčko pod střechou při Pedagogické fakulte UK, Mohylová 1964, 155 00, Praha 5 v období od 19. 10. 2008 do 23. 10. 2008. 

	\subsection{Průběh dne a jeho specifika}

			Mateřská škola je otevřena od 6:30 do 18:00 každý všední den. Časový harmonogram je spíše orientační, lze přizpůsobit momentální situaci i individuálním potřebám dětí. Mezi 6:30 a 7:30 se děti scházely v jedné společné třídě, poté si je vyučující odvedly do kmenových tříd. Večer do zavírací doby se děti, které odcházejí pozdě domů opět scházely v jedné třídě.

		\begin{spacing}{1.0}
		\begin{table}[h!]
			\center
			\begin{tabular}{|l l|}
				\rowcolor{white}
				\hline
			6:30 – 9:00				& Scházení dětí, ranní hry a činnosti dle volby dětí, 	\\ 
									& Individuální práce, individuální péče o děti\\
									& Jazykové chvilky, ranní kruh 	\\
									& Ranní cvičení, popř. relaxační cvičení, joga \\
			9:00 – 9:20				& Hygiena, svačina	\\
			9:20 – 11:45			& Plnění činností rozvíjející smyslové, manipulačně-	\\
									& technické, sebeobslužné, tělesné, estetické \\
									& a mravní stránky osobnosti dítěte (záměrné i spon- \\
									& tánní učení) ve skupinách i individuálně \\
									& Pobyt venku					\\
			11:45 – 13:00			& Hygiena, oběd, příprava na odpočinek					\\
			13:00 – 14:30			& Odpočinek dle věku dětí						\\
									& Náhradní nespací aktivity dle věku dětí 				\\
									& Péče speciálně pedagogická, logopedická				\\ 
									& Nadstandardní aktivity 							 \\
			14:30 – 15:00			& Tělovýchovná chvilka, hygiena, svačina 			\\
									& Nadstandardní aktivity 		\\
			15:00 – 18:00			& Odpolední hry a zájmové činnosti, \\
									& Individuální práce s dětmi\\
			\hline
			\end{tabular}
			\caption{ \textbf{Časový harmonogram v české MŠ}
			}
			\label{tab:rezimDneCR}
		\end{table}
		\end{spacing}

		\subsection{Příchod dětí do mateřské školy}
			
			Děti byly přijímány od 6:30 do 8:00. Rodiče pomohly dětem s přezutím a převlečením. V šatně mělo každé dítě svoji skříňku a přihrádku označenou specifickým symbolem. Rodiče jsou zodpovědní za osobní předání dítěte vyučujícímu. Při příchodu se vyučující s dítětem pozdravil a pomohl mu najít aktivitu nebo nechal dítě volně si vybrat, co bude dělat. Pozdní příchody nebo absence se hlasí telefonicky předem nebo do 8 hodiny. Příchod mimo domluvený čas je možné dohodnout s vyučujícím a přizpůsobit potřebám rodičů.


		\subsection{Volná hra}

			Volnou hru nebo také volné činnosti si dítě může volit samo. Je na dítěti samotném, čím se zaměstná, jestli si bude hrát samo nebo někoho přizve, či se k někomu připojí. Projevuje svou vlastní aktivitu a učitel zde hraje roli podpůrnou a motivační, ale neurčuje, co má dítě dělat. 

		\subsection{Ranní kruh}

			Ranní kruh se konal chvíli poté, co se ve třídě sešly všechny děti. Děti i vyučujísí si sedly vedle sebe do kruhu na koberec. Tvar kruhu zajišťuje všem rovnou pozici. Vyučujísí si s dětmi povídala o tom, co zažily o víkendu nebo si připomněly nějakou básničku. Na konci kruhu vysvětlila motivačním způsobem, co čeká děti za aktivitu. 

		\subsection{Řízené činnosti}
		 	V době před svačinou se v této mateřské škole věnoval spíše tělesným chvilkám a hrám. Po svačinně se konaly aktivity rozvíjející smyslové, manipulačně technické, estetické dovednosti apod. Jendalo se jak o vyplňování grafomotorických listů, tak výtvarné aktivity, skládání puzzle, stříhání, lepení atd. 

		\subsection{Pobyt venku}
			V této mateřské škole je pobytu venku věnován minimální čas 2 hodiny. Doba strávená venku je ovlivněna meteorogolickými podmínkami. Děti měly možnost využívat celý prostor rozlehlé zahrady, která byla bohatě vybavena. Nácházely se zde 4 pískoviště, prolézačky, houpačky, dřevěný domek, menší tabulepro kesbu křídou, zahradní domek. Děti měly k dispozici též hračky na písek a dětské zahradní nářadí. Mohly se starat i o menší bylinkovou zahrádku. Tato školka měla možnost využívat i nedalekého místního sportovního hřiště. Odchod zpět do tříd a řazení dětí u dvěří pavilonů byl využit k jednoduchým opakovaním probírané látky nebo řešením jednoduchých problémů. Dítě které vědělo odpověd bylo puštěno dovnitř a vyučující měla více času na jejich spočítání.

		\subsection{Strava a pitný režim}
			Mateřská škola měla svou vlastní kuchyň, která dovážela dětem jídlo přímo do tříd. Bylo zde dbáno na vyváženou stravu, jednu svačinu dopoledne, oběd a jednu svačinu odpoledne. Stravovací režim byl tedy nedílnou součástí programu dne. Na pitný režim byl dáván důraz. Po celý den bylo v konvi připravené pití a děti se mohly kdykoliv napít. Pítí se dětem podávalo i ke každému jídlu.

		\subsection{Odpočívání/spaní}
			Děti, které neodcházejí domů již po obedě, musely dodržovat odpolední klid. Mladším dětěm hospodářka připravovala matrace a s pokrývkou a děti chodily spát. Děti ze dvou tříd se scházeli ke spaní v jedné z nich. Před spaním se četla kniha. Respektují zde přání dítěte, pro odpočinek dětí se snaží vytvářet příjemnou atmosféru. Starší děti dodržovali klidový režim, mohli ležet a číst si nebo se věnovly jiným klidovým aktivitám jako je kreslení, puzzle, pexeso a další.  Předškoláci mají v tuto dobu předškolní přípravu. 


		\subsection{Odchod dětí z mateřské školy}
			Rodiče si mohly děti vyzvedávat po obědě od 12:30 do 13:00 a odpoledne od 14h30 kdykoliv do zavírací doby. Rodičům byl dán velký prostor, kdy si mohou pro svoje děti přijít, podle jejich potřeb a možností. Rodiče si děti vyzvedávají ve třídě nebo na zahradě. 

		\subsection{Popis budovy mateřské školy}

			Tuto mateřskou školu tvořily tři propojené jednopatrové pavilony , v každém pavilonu se nacházeli dvě třídy. v prostředním pavilonu se nacházela vlastní kuchyně. K objektu patřila velká zahrada, na které se nacházely 4 menší pískoviště, dřevěné domečky a prolézačky se skluzakou, kolotoč, moderní kruhová houpačka pro více dětí a lavičky. Na zahradě se nacházel i zahradní domek, kde se uskadňovalo dětské zahradní nářadí, hračky na zahradu a pískoviště, míče apod. Byla zde k nalezení i tabule a křídy na psaní či kreslení. V jedné části zahrady se společnými silami pěstovaly bylinky. Zvláštností bylo mlhoviště, místo, kde se při velkém vedru rozprašovala voda. 

%TODO FOTKYYYYYY !!!!!!!!!!!!!!!!!!

		\subsection{Popis třídy a materiálního vybavení}

			Před vstupem do třídy se nacházela prostorná šatna, kde mělo každé dítě svoji skříňku se značkou. Do třídy se vchází přes hygienické zázemí. Třída samotná byla prostorná a rozdělená na dvě části. V první části byly kruhové stolky s různě vysokými židličkami podle potřeb dětí. Podél zdí byly postaveny zavírací skříně na materiál a didaktické pomůcky a otevřené police. V této částí třídy se nacházel i stůl pro vyučujícího, magnetická tabule, tabule na výtvarné práce dětí a klavír. Na podlaze bylo položeno lino. Druhá část třídy byla menší a na zemi byl položen koberec. Na stěne byly zavešeny žebřiny a po obvodu byly krabice s hračkami a stavebnicemi. Z této části se dalo vstoupit do kabinetu na pomůcky.

		\subsection{Počet dětí a pedagogické zastoupení}

			V této třídě bylo zapsáno 27 dětí, v době praxe vzhledem k podzimnímu počasí docházelo v průměru 22. V průběhu dne se střídají dvě učitelky. Jedna ma ranní a druhá odpolední službu.
			Třída byla také homogenní, byly zde děti vě věku 4-5 let.
		
		\subsection{Tělocvična}
			Tělocvična byla samostatná místnost plně vybavena tělocvičným nářadím i moderními pomůckami. Na zemi byl položen koberec. Z důvodu malého a prostorem nevyhovujícího kabinetu na pomůcky,  byly míče zavěšeny v síti u stropu.

		\subsection{Hygienické zázemí}
			Hygienické zázemí se nacházelo hned za vstupem ze šatny. Jednalo se o menší příjemný a barevně laděný prostor. Záchody pro menší děti byly společné, větší děti je měly oddělené přepážkou pro větší soukromí. Nacházelo se zde dostatek umyvadel a dbalo se na číštění zubů po obědě. Záchody byly vždy společné pro dvě třídy. 


%\subsection{Pravidla chování}
			%Pravidla v této třídě nejsou nikde vyvěšena. Jsou zde chápána jako opatření, která napomáhají organizaci. Důležitým pravidlem je zazvonění na zvonek, při kterém se musí všichni ztišit a dávat pozor, co se bude dít. Ostatní pravidla jsou spíše připomínána ve chvíli, kdy nastane nějaký konflikt a vždy je dětem jasně vysvětleno, proč by se zrovna taková pravidla měla dodržovat. Dbá se i na to, aby určitá pravidla byla připomenuta dětmi samotnými. Tudíž ne z pozice učitel-dítě, ale z pozice rovný s rovným.
			%Hodně se zde pracuje se vzájemnou důvěrou, nechají děti dělat různé stavby při volné hře i mimo místa tomu určena, ovšem za určitých podmínek a vzájemně si důvěřují, že obě strany dohodu dodrží. Občas dostanou starší děti za úkol dohlédnout na ty mladší. Toto předávání funkcí a důležitosti na starší děti bylo pozitivně  a zodpovědně přijímáno.



	\section{Srovnání režimu mateřské školy ve Francii a České republice}
\label{srovnani}
%TODO: tabulka


\begin{table}[h]
	
	\begin{tabular}{|l|l|l|}
	\hline
	\rowcolor{grey}
								& \textbf{Francie}				& \textbf{Česká republika}	\\
	\hline
	\hline
	%=========================================================================
\rowcolor{grey!10}	 příchod dětí do MŠ			& 8:50 - 9:10				& 6:30 - 8:00			\\ 
\rowcolor{grey!50}	 čas na volnou hru 			& 20min 					&30min - 2hod 	\\ 
\rowcolor{grey!10}	 čas na rituály 			&10min 						&10-15min \\
\rowcolor{grey!50}	 čas na řízenou aktivitu	&5h30 						&2-3h   \\ %!!!!!!!!!!
\rowcolor{grey!10}	 čas na pobyt venku     	&2x30min během dne			&min.2h 	\\ 
\rowcolor{grey!10}								&a 1h po obědě				& \\ 
\rowcolor{grey!50}	 čas na stravu				&40min						&1h \\
\rowcolor{grey!10}	 čas na odpolední spaní 	&1h 						&1h30 	\\
\rowcolor{grey!50}	 odchod dětí z MŠ			&16h30						&12:30-13:00, 14:30-18:00	\\														 
	 \hline
	  
	%=========================================================================
	\end{tabular}
	\label{srovnanirezimdne}
	\caption{ \textbf{Srovnání časové dotace na aktivity ve Francii a České republice} Tabulka znázorňuje kolik času májí obě MŠ vyhrazeno na dané aktivity. %???????????????????
	}
\end{table}


	Na začátku kapitoly již bylo zmíněno, že se nejedná o bohatý výzkumný vzorek. K dispozici jsou pouze dvě třídy mateřské školy, jedna francouzská a jedna česká. Za těchto podmínek nelze stanovovat obecné závěry. V obou zemích se mohou některé aspekty měnit školu od školy, jiné však zůstavají jako vzor, který platí téměř všude. Následující popis srovnání lze považovat jako nahlédnutí do systému obou zemí a přiblížení organizační struktury a zázemí mateřských škol.


	Tabulka \ref{TABULKA!!!} je souhrnem časů v režimu dne mateřských škol. Již první bod, příchod dětí do MŠ ukazuje rozdíly. Ve Francii platí tento čas ve všech mateřských školách, může se lišit v odstupu několika minut. Čas příchodu do MŠ kopíruje čas příchodu a záčátek vyučování primární školy viz. \ref{tabulka}. Oproti tomu v České republice si čas příchodu upravuje každá mateřská škola. Některé školy, jako tato, vychází vstříc pracujícím rodičům a příjímá děti již od brzkých ranních hodin.

	Dalším sledovaným aspektem je čas věnovaný volné hře. V ranních hodinách mají děti ve Francii přibližně 20 minut volného čas. Tento čas se zdál autorce nedostatečný, většina dětí se sotva stihla rozkoukat a začít si samostatně nebo s kamarády hrát a už musely jít \uv{pracovat}. Pouze průraznější děti dokázaly tento relativně krátky čas naplno využít.  V České republice čas na volnou hru vyplývá z příchodu dětí. Děti, které jsou v MŠ od brzkých ranních hodin mají více volného času. Naopak děti, které přicházejí později, jsou na tom podobně jako děti ve Francii. 
	Další čas pro volnou hru je při pobytu venku. Celkový čas je v obou zemích téměř totožný, přestože je v režimu dne zařazen jindy a v jiných časových blocích. Ačkoliv ze sledování vyplýva, že celkový čas na volnou hru je témeř totožný, autorce se zdál aspekt materiálního vybavení a prostorového uspořádání, kde se volná hra uskutečnuje jako důležítý, více níže.

	Čas věnovaný ranní rituálům byl v obou zemích totožný. 

	








\begin{table}[h]
	
	\begin{tabular}{|l|l|l|}
	\hline
	\rowcolor{grey}
								& \textbf{Francie}				& \textbf{Česká republika}	\\
	\hline
	\hline
	%=========================================================================
\rowcolor{grey!10}	 aktivity při volné hře	&puzzle, malování,čtení 	&stavebnice, hračky, kostky,\\ 
\rowcolor{grey!10}	 						&							&kočárky, malování, čtení \\ 
\rowcolor{grey!50}	 druh řízené aktivity   &"práce"-gragomotorika, 	&"hra"-výtvarné, hudební, \\ 
\rowcolor{grey!50}	 						&stříhání, lepení, skládání,&pohybové, námětové hry, \\ 
\rowcolor{grey!50}	 						&pregramatické činnosti 	&, grafomotorika, \\
\rowcolor{grey!10}	 aktivity při rituálech &docházka, datum, říkanka 	&ranní povídání, vyprávění,\\ 
\rowcolor{grey!10}	 						&, vysvětlení dílen			& motivace k další činnosti\\ 
\rowcolor{grey!50}   venkovní aktivity 		&  %VYPLNIT					&%
	  
	%=========================================================================
	\end{tabular}
	\label{srovnanirezimdne}
	\caption{ \textbf{Srovnání aktivit realizovaných v MŠ Francie a České republiky}
	}
\end{table}



	 Problémem ovšem byl nedostatečný prostor ve třídě, kde by si děti mohly hrát.Rozdílem je ovšem prostor. V české mateřské škole má většina tříd dvě místnosti a děti mohou využívat celý prostor třídy a k dispozici mají i větší množství her a hraček. Volné hře dětí je dáván dostatek prostoru i času.

	. Zde se dají také nalézt markantní rozdíly, zejména z hlediska uspořádání venkovních prostor. Venkovní prostory ve Francii jsou převážně betonové dvory. Toto prostředí je ze subjektivního názoru autorky neútulné a děti neměly k dispozici dostatek her či prolézaček k vyžití. Zahrada v české MŠ byla oproti tomu plná herních prvků a děti si mohly vybírat z bohatší nábídky. Takto velké a vybavené zahrady však také nejsou v mateřských školách typické, přesto většina českých MŠ má menší či větší zahradu s herními prvky.

	 
V České republice měl spíše ráz příjemné chvilky, kdy si všichni povídaly a fungoval jako úvod dne. Ve Francii se už během této chvíle pracuje a vysvětlují se charakteristiky dílen.l


 		%Z příchodu a odchodů dětí z mateřské školy a vlastní praxe vyplývá, že v České republice jsou otevírací časy mateřské školy přizpůsobené potřebám rodičů. Ráno mohou dát děti do školky již od brzkých hodin, aby mohli být včas práci a o děti bylo vhodně postaráno. I odpolední vyzvedávání je jim přizpůsobeno, mohou si své děti vyzvedávat již po obědě nebo po odpoledním spaní, na druhou stranu, je-li potřeba, děti mohou ve školce zůstat až do večerních hodin. Oproti tomu ve Francii příchod a odchod dětí kopíruje docházku na základní školu. Příchod do práce je ve Francii též pozdější, než je obvyklé v České republice. Ve Francii však zůstávají v práci do pozdějších večerních hodin, proto je velmi časté, že děti vyzvedávají chůvy a zůstávají s nimi do příchodu rodičů. Dalším markantním rozdílem je čas na volnou hru a řízenou aktivitu. Podle vzdělávacího plánu Francie je patrné, že děti jsou připravovány na školu.


		Tato bakalářská práce se nevěnuje práci s dětmi a přístupu k nim, ale pro lepší obrázek o tom, jak to chodí za našimi hranicemi, je podle mého názoru velmi důležité též zmínit rozdíl v řízených aktivitách. Při francouzských ateliérech se „pracuje“, dětem je neustále připomínáno, že si nehrají, že mají správně sedět a soustředit se na svou „práci“. Činnost musí vždy dokončit, podepsat si ji, napsat datum (okopírovat datum z předlohy) a vlepit do svého sešitu, kam si vkládají všechny vlastní práce. V České republice mají řízené aktivity spíše formu „hry“. Děti si hrají, malují, tvoří, zkouší, apod. 


		Obsahová stránka řízených aktivit se zdá být v některých věcech podobná. Grafomotorické listy mají stejnou podobu, stavebnice a puzzle se najdou v obou zemích, jen ve Francii patří mezi řízené aktivity, v České republice si je děti vybírají během volné hry. Výtvarné techniky jsou stejné, tělesná výchova je též podobná, s větším důrazem na ofenzivní hru ve Francii, kdy 2 děti stojí proti sobě a snaží se druhého vytlačit z vymezeného prostoru, či ho převrátit z břicha na záda. To jsem v České republice nezažila. Dramatické výchově se ve Francii přikládá minimální význam. 

		Ke srovnání režimu dne byly použity dvě rozdílné třídy, jedna homogenní (F) a jedna heterogenní (ČR), je tedy těžké porovnávat přípravu předškoláků. Ale z toho, co jsem viděla v české mateřské škole, kde bylo 6 předškoláků a francouzské mateřské třídě předškoláků „Grande section“, je ve Francii dáván daleko větší důraz na rozvoj jazykových schopností, který s největší pravděpodobností vyplývá z nestejnosti psaného a mluveného jazyka, a dále na rozvoj psaní. Děti by při přechodu na základní školu již měly umět psát tiskacími písmeny a přečíst své jméno psacím písmem. 

		Stejným či společným aspektem je odpolední odpočinek. Jak v České republice tak ve Francii nejmenší děti mají vyhrazený čas okolo jedné až jedné a půl hodiny na spánek. Rozdíl je u starších dětí, u nás mají klidové činnosti, jako je čtení, malování, odpočinek, ve Francii starší děti tráví tento čas venku na dvoře za dohledu animátorů, kteří jím nabízejí aktivity sportovnějšího zaměření. Děti ovšem nejsou povinny se jich zúčastnit.

		Dalším srovnávaným faktorem je strava. Nerada bych zde dělala závěry, jsem přesvědčena, že zákaz svačin v mateřské škole ve Francii byl čistě individuální a zjednodušoval práci jak vedení, tak učitelům, aby nevznikaly zbytečné komplikace. Na dětech bylo zřetelné, že strava byla nedostatečná, v odpoledních hodinách byla vidět na dětech patrná únava, podrážděnost a apatie. To vše bylo podpořeno nedostatečným důrazem na pitný režim.

		Též hygienické zázemí je nesrovnatelné. Ve Francii je jedna hygienická místnost pro všechny třídy, otevřená, bez známky intimity, dost často tato místnost zaváněla močí. V České republice má každá třída své hygienické zázemí, jedná se o menší a intimnější prostor než ve Francii.
		Nakonec bych se chtěla zmínit o velmi důležitém aspektu rozdílnosti ve výuce. Tím je počet učitelů na třídu. Ve Francii je na jednu třídu o 30 dětech jedna učitelka na celý den. S nikým se nestřídá ani v průběhu dne ani v průběhu týdne. V České republice jsou standardem 2 učitelky na třídu, které se střídají, a část výuky se jim překrývá. Ranní a odpolední služby se jim pravidelně střídají. Co se týče prostorů, je poměr prostoru na dítě velmi rozdílný. Ve Francii je větší počet dětí „vtěsnán“ do jedné místnosti, kde tráví celý den. V České republice má většinou jedna třída dvě místnosti, děti mají tedy téměř dvojnásobek prostoru na o něco menší počet dětí.

		
		Časová dotace výuky, režim dne, přístup k dětem a metody výuky jsou po celé Francii stejné, ale strava a prostory se budou lišit podle možností a financí jednotlivých městských částí. Nelze tedy ze všech faktorů, které tu uvádím, vyvozovat jasné závěry.
