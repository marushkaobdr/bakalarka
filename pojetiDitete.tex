\chapter{POJETÍ DÍTĚTE VE FRANCII A ČESKÉ REPUBLICE}

Přestože se postavení mateřských škol ve vzdělávacím systému obou zemí formálně shoduje, v cílech mateřských škol a pohledu na dítě se obě země zásadně odlišují. Pojetím dítěte v rámci této práce je myšlen pohled na dítě v kontextu vzdělávacích cílů mateřských škol. 

Hlavním cílem ve francouzském předškolním vzdělávání je uvádět dítě do světa školy, tzn. směřovat práci v mateřské škole k přípravě na vstup do povinné školní docházky. Toto pojetí je tradiční ve většině románských zemí. Jedná se o školský model. (viz 1.1).

 Česká republika je tímto pojetím také ovlivněna, nicméně koncept české mateřské školy je mnohem širší. Cílem není pouze příprava na roli žáka, ale zejména celková příprava na život, která v sobě zahrnuje také přípravu na povinnou školní docházku. Důraz je v České republice kladen zejména na socializaci a radostné dětství s ostatními dětmi, tj. \uv{rosteme společně}. Nejedná se tedy čistě jen o školský nebo rodinný model, ale lze o něm hovořit jako o modelu smíšeném.

Tato pojetí jsou ovlivněna historickým vývojem mateřských škol a postavením dítěte v nich.

Historicky byla francouzská mateřská škola vnímána jako vzdělávací instituce. Tvoří první článek vzdělávací soustavy. Cykly, kterými dítě v předškolním věku prochází, jsou vnímány jako nezbytný obsah, na nějž se váže povinná školní docházka. 

Tradice české mateřské školy je také velmi dlouhá, ale ve své historii nebyla jednoznačně orientovaná jako příprava na školu. Její pozice jako vzdělávací instituce se vztahuje až ke školskému zákonu, kdy je zařazena jako první článek vzdělávací soustavy. 

 Po roce 1948 se pozice MŠ více blížila sociálnímu zařízení než skutečně výchovně vzdělávacímu. Mezi lety 1948 a 1989 je její vzdělávací charakter nezpochybnitelný. S přijetím dokumentu \uv{Další rozvoj výchovně vzdělávací soustavy} v roce 1976 byl posílen aspekt přípravy na školu. Cíl mateřských škol byl zúžen na přípravu pro povinné vzdělávání. Děti, které neabsolvovaly mateřskou školu, byly při zápisech do základní školy vyzvány k náhradnímu opatření, tzv. přípravných tříd, alespoň na dobu 3 měsíců, neboť dovednosti a znalosti získané před nástupem do 1. třídy byly východiskem, na němž 1. třída \uv{startovala}. Po roce 1989 byl krátkodobě zpochybněn vzdělávací charakter ve prospěch pozice sociální, avšak na konci devadesátých let, zvláště pak školským zákonem v roce 2004 se její pozice zakotvila a posílila. CITACE UHLIROVA

 %TODO citacae Uhlirova
